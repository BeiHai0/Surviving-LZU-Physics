\section*{第五章~~总结与展望}

\setcounter{section}{5} \setcounter{subsection}{0}
\setcounter{table}{0} \setcounter{figure}{0} \setcounter{equation}{0}
\addcontentsline{toc}{section}{第五章~~总结与展望}

暗物质作为现代宇宙学与粒子物理交汇处最重要的问题之一,已成为连接宏观宇宙结构与微观基本粒子的关键线索。从1930年代Zwicky首次提出“隐藏质量”假说,到今日多波段、多手段观测支持下ΛCDM模型的建立,暗物质研究走过了近一个世纪的探索历程。当前,星系旋转曲线、引力透镜效应与宇宙微波背景辐射等多重观测证据已牢固确立了暗物质存在的事实。然而,它的基本属性、粒子本质以及相互作用机制仍未被揭示。

理论上,WIMP、轴子等多种候选粒子模型在标准模型之外提供了丰富的可能性,但在LHC、XENON、LUX-ZEPLIN、AMS-02等一系列前沿实验的努力下,尚无确凿的探测信号被确认。与此同时,ATIC、PAMELA、DAMPE 和 AMS 等间接探测实验对正电子、反质子等能谱结构的观测则引发了关于天体源与暗物质湮灭机制的激烈讨论,也暴露出理论建模的不确定性与观测解释的多义性。

展望未来,暗物质研究正迈向更高灵敏度与多探针协同发展的阶段。实验方面,下一代液氙探测器(如DARWIN)、空间谱仪(如HERD)以及地下超低本底实验(如PANDAX-III)将显著提升对WIMP和轻暗物质的探测能力。理论方面,更加重视宇宙学-粒子物理-引力三者融合的跨尺度建模,也将成为理解暗物质本质的关键突破口。此外,若未来某一实验或观测方向取得决定性结果,其影响将不仅局限于暗物质本身,更可能推动标准模型和广义相对论的根本性修正。

因此,尽管当前我们尚未“看见”暗物质,但它正以前所未有的方式引领基础物理学走向更深层的理解,揭示关于宇宙、物质和引力的基本奥秘。

\newpage