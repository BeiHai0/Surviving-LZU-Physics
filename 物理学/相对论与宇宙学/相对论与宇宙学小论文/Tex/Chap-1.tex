\pagestyle{fancy} % 正文页眉页脚设置
\pagenumbering{arabic}\setcounter{page}{1} % 页码从1开始
\lhead{\song \fontsize{9pt}{13pt}\selectfont 《相对论与宇宙学》课程小论文} % 页眉左要求是“兰州大学本科生毕业论文”,小五号字体对应9pt
\chead{} % 页眉中间为空
\rhead{\song \fontsize{9pt}{13pt}\selectfont \selectfont 暗物质存在的证据与可能的解释} % 页眉右边是论文题目
\lfoot{} % 页脚左边为空
\cfoot{\thepage} % 页脚中间显示页码,从正文开始计数
\rfoot{} % 页脚右边为空


\section*{第一章~~引言}
\setcounter{section}{1} \setcounter{subsection}{0}
\addcontentsline{toc}{section}{第一章~~引言}

在现代宇宙学和粒子物理学的发展中,暗物质(Dark Matter)问题被视为最根本而深刻的未解之谜之一。尽管它无法通过电磁相互作用被直接观测,我们却可以通过大量天文现象间接推断出其存在:从星系旋转曲线的异常,到引力透镜效应的质量分布不符,再到宇宙微波背景辐射和大尺度结构的形成。暗物质在宇宙演化中扮演着不可或缺的角色。

暗物质的概念最早可以追溯到1930年代,当时Fritz Zwicky发现,可见星系的质量不足以提供星系团内所需的引力约束。随后,1970年代Vera Rubin对螺旋星系的系统观测进一步确立了“质量不守恒”的现象,并引发对宇宙中“不可见物质”的广泛研究。此后,暗物质逐渐被纳入ΛCDM(Λ Cold Dark Matter)模型,成为标准宇宙学框架中不可或缺的一环。\cite{bertone2018history}

然而,迄今为止,暗物质的本质仍不明确。尽管从引力效应可以推断出其存在,我们却尚未直接探测到暗物质粒子。其成分、相互作用类型乃至是否真为“物质”本身,仍是理论物理和实验物理的重大前沿课题。对此,物理学界提出了多种理论模型,如WIMP(弱相互作用大质量粒子)、轴子、超对称模型和额外维等,并开展了大量直接与间接探测实验。

本文旨在对暗物质研究进行简要综述,围绕观测证据与理论模型两条主线展开。第二章将介绍三类主要的暗物质观测证据,包括星系旋转曲线、引力透镜效应以及宇宙微波背景辐射;第三章将介绍当前主流的暗物质候选者,讨论它们的物理动机、特点与局限;第四章则简要介绍正在进行中的实验努力及其最新进展;第五章将讨论当前研究所面临的挑战及未来可能的发展方向。

通过综述现有观测结果与理论探索,本文希望为理解暗物质的研究路径提供清晰的结构与概览,并凸显该问题在现代物理中的核心地位与挑战性。

\newpage

%{\chuhao 初号字.}\\
%{\xiaochuhao 小初号字.}\\
%{\yihao 一号字.}\\
%{\erhao 二号字.}\\
%{\xiaoerhao 小二号字.}\\
%{\sanhao 三号字.}\\
%{\sihao 四号字.}\\
%{\xiaosihao 小四号字.}\\
%{\wuhao 五号字.}\\
%{\xiaowuhao 小五号字.}\\
%{\liuhao 六号字.}\\
%{\qihao 七号字.}\\
%\par 公式

%| 中文字号 | 大小(pt)  |
%| ---- | ------- |
%| 初号   | 42pt    |
%| 小初号  | 36pt    |
%| 一号   | 26pt    |
%| 小一号  | 24pt    |
%| 二号   | 22pt    |
%| 小二号  | 18pt    |
%| 三号   | 16pt    |
%| 小三号  | 15pt    |
%| 四号   | 14pt    |
%| 小四号  | 12pt    |
%| 五号   | 10.5pt  |
%| 小五号  | 9pt |
%| 六号   | 7.5pt   |
%| 小六号  | 6.5pt   |

