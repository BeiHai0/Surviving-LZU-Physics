  % UTF-8 encoding
\documentclass[9pt, dvipsnames]{beamer} %
% Beamer 设置
\usetheme[secheader]{Boadilla} % 使用的 Beamer 主题: Boadilla
\usecolortheme{beaver} % 使用的 Beamer 颜色:beaver
% 字体设置
\usefonttheme{professionalfonts} % professional 字体
% 其他 Package
\usepackage{times}
\usepackage{amsmath}
\usepackage{verbatim}
\usepackage{anyfontsize}
\usepackage{subcaption} % 子图片
\usepackage{graphicx} % 图片
\usepackage[export]{adjustbox}
\setbeamertemplate{caption}[numbered]
\newcounter{saveenumi}
\resetcounteronoverlays{saveenumi}
\usepackage[multidot]{grffile} % 允许文件名带多个点
\usepackage{tabularx} % 表格
\usepackage{tikz}
\usepackage{ctex}
\usepackage{multicol}
\usepackage{mathtools} % 提供 \mathcolorbox
\usepackage{xcolor}    % 提供颜色支持
\AtBeginSection[]
{
	\begin{frame}{Index}
		\transfade%淡入淡出效果
		\tableofcontents[sectionstyle=show/shaded,subsectionstyle=show/shaded/hide]
		\addtocounter{framenumber}{-1}  %目录页不计算页码
	\end{frame}
}
%%%%%%%%%%%%%%%%%%%

\title{Nonlinear electrodynamics coupled with gravity } % 标题
%\author{林照翔  \\ 科研与实践III导师:赵力} % 作者
\date{\today} % 如果 Date 参数为空,自动显示当前日期

\begin{document}
\setbeamerfont{headline}{size=\Tiny}
\everymath{\displaystyle}

    % 标题页
\begin{frame}
    \titlepage % 根据上面信息生成标题
\end{frame}

\begin{frame}
    \frametitle{\textbf{Index}}
    \begin{multicols}{2}
    \tableofcontents
    \end{multicols}
\end{frame}

\section{Introduction}

\begin{frame}{引言}
    非线性电动力学(Nonlinear Electrodynamics, NLED)是经典电动力学的一个重要扩展,旨在解决线性麦克斯韦理论中的一些基本问题,例如点电荷的无限自能和奇异场强。自20世纪30年代Born和Infeld提出非线性电动力学模型以来,这一领域经历了多次复兴和发展。近年来,随着对宇宙加速膨胀和黑洞物理的深入研究,NLED在广义相对论和宇宙学中的应用引起了广泛关注。

在广义相对论中,NLED与引力的耦合为解决奇点问题提供了新的思路。例如,Born-Infeld电动力学在与爱因斯坦方程耦合时,能够产生无奇点的黑洞解。这些解被称为“正则黑洞”,在某些情况下表现出与Reissner-Nordström解类似的渐近行为。然而,NLED在引力耦合中的应用仍面临诸多挑战,例如能量条件的违反和解的稳定性问题。

在宇宙学中,NLED的应用主要集中在解释宇宙的加速膨胀和避免初始奇点。研究表明,非线性电磁场可能在宇宙早期阶段发挥重要作用,甚至能够诱导宇宙从收缩状态反弹到膨胀状态。此外,NLED在f(T)引力理论中的应用也显示出其在描述宇宙演化中的潜力。f(T)引力理论通过引入挠率标量T来修改引力作用,为研究宇宙加速膨胀提供了新的框架。

本文探讨了NLED在平坦时空中、与引力耦合的黑洞解以及在f(T)引力理论中的应用。这些研究不仅展示了NLED在解决经典电动力学和广义相对论中奇点问题的潜力,还揭示了其在宇宙学中的重要应用前景。通过对这些研究的分析,我们希望进一步理解NLED在现代物理中的角色,并为未来的研究提供新的方向。
\end{frame}

\section{Aspects of a novel nonlinear electrodynamics in flat spacetime and in a gravity-coupled scenario}

\subsection{非线性电动力学模型和场方程}

\begin{frame}{麦克斯韦拉氏量和BI模型拉氏量}

麦克斯韦的拉氏量(真空、无源,三维表述)

$$
F
=\frac{1 }{2 } (B^2-E^2)
$$

$$
L_{\mathrm{maxwell}}
=-F
$$

BI 模型的拉氏量

$$
L_{\mathrm{BI}}
=\frac{2 }{\beta } \left(1-\sqrt{1+\beta F} \right) 
$$

$\beta $ 是任意常数。   弱场极限 $\beta F\ll 1 $ 下,BI 模型的拉氏量可近似为:

$$
L_{\mathrm{BI}}
=\frac{2 }{\beta } \left(1-\sqrt{1+\beta F} \right)
\approx -F + \frac{1 }{4 } \beta F^2 - \frac{1 }{8 } \beta^2 F^3 +\mathcal{O}\left(\beta^3 F^4 \right) 
$$

当 $\beta\to 0 $,BI 模型的拉氏量与线性麦克斯韦的拉氏量相同。

\end{frame}

\begin{frame}{新 NLE 模型拉氏量}
一种新型 NLE 模型的拉氏量

$$
L_{\mathrm{general}}(F)
=-\frac{\left(aF+1 \right)^m }{\delta(bF+1)^n } \left(\beta F \right)^p
$$

$m,n,p $ 是无量纲常数,$a,b,\beta,\delta $ 是长度平方量纲的任意常数。在弱场极限下,拉氏量可近似为:

$$
L_{\mathrm{general}}(F)
=-\frac{\left(aF+1 \right)^m }{\delta(bF+1)^n } \left(\beta F \right)^p
\approx -c\left[F^p + c_1 F^{p+1} +c_2 F^{p+2}  + \mathcal{O}\left(c_3 F^{p+3} \right) \right]
$$

$p=1 $ 时得到麦克斯韦的拉氏量。通过后面分析取 $m=1,n=m+1,a=-3b $,得到含有两个参数且遵守麦克斯韦极限的拉氏量:

$$
\boxed{
L(F)
=\frac{\gamma(3\eta F - 1 )F }{(1+\eta F)^2 }
}
$$

其中,$\gamma=\beta/\delta $ 和 $\eta $ 是任意参数。

当 $\eta F\ll 1 $,即弱场极限下,拉氏量近似为:

$$
L(F)
=\frac{\gamma(3\eta F - 1 )F }{(1+\eta F)^2 } 
\approx -\gamma F + 5\gamma \eta F^2 -9\gamma \eta^2 F^3 + \gamma\mathcal{O}\left(\eta^3F^4 \right) 
$$

\end{frame}

\begin{frame}{新 NLE 模型的物理对应}
利用电位移矢量 $\vec{D} $ 与 $\vec{E} $ 的关系 $\vec{D}=\partial L/\partial \vec{E} $,可由拉氏量式得到:

$$
\vec{D}
=\gamma\frac{1-7\eta F }{(1+\eta F)^3 } \vec{E}
$$

$D_i = \varepsilon_i^{~~ j } E_j $,介电常数张量

$$
\boxed{
\varepsilon_{ij} = \gamma \frac{1-7\eta F }{(1+\eta F)^3 }\delta_{ij}
}
$$

磁场 $\vec{H}=-\partial L/\partial \vec{B} $ 结合拉氏量有

$$
\vec{H}
=\gamma \frac{1-7\eta F }{(1+\eta F)^3 } \vec{B}
$$

磁感应强度 $B_i=\mu_i^{~~j}H_j $,磁导率张量的逆 $\left(\mu^{-1} \right)_{ij} $

$$
\boxed{
\left(\mu^{-1} \right)_{ij}
=\gamma \frac{1-7\eta F }{(1+\eta F)^3 } \delta_{ij}
}
$$

可以认为新 NLE 拉氏量由这种特殊的介质生成。
\end{frame}

\begin{frame}{场方程}
    平坦时空中拉氏量给出欧拉-拉格朗日运动方程:

    $$
    \partial_\mu\left(L_F F^{\mu\nu} \right) = 0
    $$
    
    其中,
    
    $$
    L_F
    \equiv \frac{\partial L }{\partial F } 
    =\frac{\gamma(-1+7\eta F) }{(1+\eta F)^3 }
    $$
    
    $F^{\mu \nu} $ 是麦克斯韦场强张量。可以回到无源麦克斯韦方程:

    $$
    \boxed{
    \nabla\cdot\vec{D} = 0,\quad
    \frac{\partial \vec{D} }{\partial t } - \nabla\times\vec{H}= \vec{0}
    }
    $$
    
    由 Bianchi identity $\partial_\mu \tilde{F}^{\mu \nu}=0 $,$\tilde{F}^{\mu\nu} $ 是场强张量的对偶,可得
    
    $$
    \boxed{
    \nabla\cdot\vec{B} = \vec{0},\quad 
    \frac{\partial \vec{B} }{\partial t } + \nabla\times\vec{E} = \vec{0}
    }
    $$
\end{frame}

\begin{frame}{点电荷电场}
    考虑静电极限(electrostatic limit) $\vec{B}=\vec{H}=\vec{0} $,对点电荷

    $$
    \nabla\cdot\vec{D} = e\delta(\vec{r})
    $$
    
    解:
    
    $$
    \vec{D}
    =\frac{e }{4\pi r^3 } \vec{r}
    $$
    
    结合 $\vec{D},\vec{E} $ 关系和 $F=-E^2/2 $ 可得
    
    $$
    \boxed{
    E+\frac{7 }{2 } \eta E^3
    =\frac{e }{4\gamma \pi r^2 } \left(1-\frac{\eta  }{2 } E^2 \right)^3 
    }
    $$
    
    上式限制 $F>-1/\eta $;弱场极限 $\eta F\ll 1 $,$E(r) $ 可按 $\eta $ 展开
    
    $$
    E
    =E_{(0)} + \eta E_{(1)} + \eta^2 E_{(3)} + \mathcal{O}\left(\eta^3 \right)
    $$
    
    $E_{(1)},E_{(2)} $ 分别代表对电场 $E_{(0)} $ 的一阶和二阶修正。
\end{frame}

\begin{frame}
    比较系数可得
    
    $$
    E_{(0)}
    =\frac{e }{4\pi\gamma r^2 }
    $$
    
    $$
    E_{(1)} 
    =-\frac{7 }{2 } E_{(0)}^3 - \frac{e }{4\pi \gamma r^2 } E_{(0)}^2
    $$
    
    $$
    E_{(2)}
    =-\frac{21 }{2 } E_{(0)}^2 E_{(1)} + \frac{e }{4\pi\gamma r^2 } \left(-3E_{(0)}E_{(1)} + \frac{3 }{4 } E_{(0)}^4 \right)
    $$
    
    弱场极限下
    
    $$
    E
    \approx \frac{e }{4\pi\gamma r^2 } - 5\eta\left(\frac{e }{4\pi\gamma r^2 }  \right)^3 + \frac{273 }{4 } \eta^2\left(\frac{e }{4\pi\gamma r^2 }  \right)^5 + \mathcal{O}\left(\eta^3 \right)
    $$
    
    对于很小的 $r$ 和任意的 $\eta$,电场最大值
    
    $$
    \boxed{
    E_{\max}
    =\sqrt{\frac{2 }{\eta } }
    }
    $$
    
    NLE 模型中\textcolor{red}{电场有限}。当 $\eta\to 0 $,电场发散。
    
\end{frame}

\subsection{点电荷的能量}

\begin{frame}{点电荷的能量}
    平坦时空中希尔伯特应力-能量张量(Hilbert stress-energy tensor)是一个对称张量且是洛伦兹不变量,其定义为

    $$
    T_{\mu\nu}^H
    \equiv -\frac{2 }{\sqrt{-g} } \left(\frac{\partial \left(\sqrt{-g}L(F) \right) }{\partial g^{\mu\nu} }  \right)\bigg|_{g=\eta}
    =\eta_{\mu\nu}L(F) - L_FF_\mu^\alpha F_{\nu\alpha}
    $$

    结合新NLE模型 $L(F)=\frac{\gamma(3\eta F-1)F }{(1+\eta F)^2 } $ 可得电能密度
    
    $$
    \rho
    =-T_t^t
    =-L_FE^2-L(F)
    =\frac{\gamma E^2\left[1+\frac{3 }{2 } \eta E^2 \left(4+\frac{\eta }{2 } E^2 \right) \right] }{2\left(1-\frac{\eta }{2 } E^2 \right)^3 } 
    $$
    
    总电能
    
    $$
    \epsilon
    =4\pi\int_{0}^{+\infty} \rho(r)r^2\mathrm{d}r
    $$ 
    
    转化为对 $E $ 的积分
    
    $$
    \epsilon
    =\frac{e^{3/2} }{\sqrt{4\pi\gamma} } \int_{0}^{\sqrt{\frac{2 }{\eta  } }} \frac{\sqrt{\left(2-\eta E^2 \right)\left[4+3\eta E^2\left(8+\eta E^2 \right) \right]\left[4+\eta E^2\left(52+21\eta E^2 \right) \right]} }{16\sqrt{E}\left(2+7\eta E^2 \right)^{5/2} } \mathrm{d}E
    $$
    
    \textcolor{red}{总能量有限}。当 $\eta\to 0$,点电荷自能发散。
\end{frame}

\subsection{真空双折射}

\begin{frame}{真空双折射}
        考虑平面电磁波 $(\vec{e},\vec{b}) $ 沿 $z $ 轴在两片平行电容板间传播,$x $ 轴方向有背景匀强电场。外电场 $\bar{\vec{E}}=(\bar{E},0,0) $,总电场 $\vec{E}=\vec{e}+\bar{\vec{E}},\vec{B}=\vec{b} $,设 $\vec{e} $ 远小于 $\bar{\vec{E}} $,拉氏量

    $$
    L\left(\vec{e}+\bar{\vec{E}},\vec{b} \right)
    =\gamma\frac{\left\{\frac{3 }{2 } \eta\left[\vec{b}^2-\left(\vec{e}+\bar{\vec{E}} \right)^2 \right] - 1 \right\}\left[\vec{b}^2-\left(\vec{e}+\bar{\vec{E}} \right)^2 \right] }{2\left\{1+\frac{\eta }{2 } \left[\vec{b}^2-\left(\vec{e} + \bar{\vec{E}}^2 \right) \right] \right\}^2 } 
    $$
    
    忽略高阶项
    
    $$
    L^{(2)}\left(\vec{e}+\bar{\vec{E}},\vec{b}\right)
    =\frac{\gamma\eta\left(5+\frac{7 }{2 } \eta \bar{\vec{E}}^2 \right) }{\left(1-\frac{\eta }{2 } \bar{\vec{E}}^2 \right)^4 }\left(\vec{e}\cdot\bar{\vec{E}} \right)^2 - \frac{1 }{2 } \gamma \frac{\left(1+\frac{7 }{2 } \eta \bar{\vec{E}}^2 \right) }{\left(1-\frac{\eta }{2 } \bar{\vec{E}}^2 \right)^3 }\left(\vec{b}^2-\vec{e}^2 \right) 
    $$
    
    电位移矢量和磁场强度
    
    $$
    d_i = \frac{\partial L^{(2)} }{\partial e_i } = \left(\alpha\delta_i^j+\beta\bar{E}_i\bar{E}^j \right)e_j,\quad
    h_i
    =-\frac{\partial L^{(2)} }{\partial b_i } 
    =\alpha\delta_i^j b_j 
    $$

    $$
    \beta = \frac{2\gamma \eta\left(5+\frac{7 }{2 } \eta \bar{\vec{E}}^2 \right) }{\left(1-\frac{\eta }{2 } \bar{\vec{E}}^2 \right)^4 },\quad 
    \alpha = \gamma \frac{\left(1+\frac{7 }{2 }\eta\bar{\vec{E}}^2 \right)  }{\left(1-\frac{\eta }{2 } \bar{\vec{E}}^2 \right)^3 }
    $$
\end{frame}

\begin{frame}    
    结合关系 $d_i=\varepsilon_i^j e_j,h_i=\left(\mu^{-1} \right)_i^j b_j $ 得到
    $
    \varepsilon_{ij} = \alpha\delta_{ij} + \beta\bar{E}_i\bar{E}_j ,\quad
    \left(\mu^{-1} \right)_{ij} = \alpha\delta_{ij} 
    $
    
    平面波麦克斯韦方程
    
    $$
    k_i d^i = k_i b^i = 0,\quad
    \vec{k}\times\vec{e} = \omega\vec{b},\quad
    \vec{k}\times\vec{h} = -\omega\vec{d} 
    $$
    
    
    $$
    \left\{\varepsilon^{ijk}\varepsilon_{lmn} \left(\mu^{-1} \right)_k^l k_j k^m+\omega^2\epsilon_n^i \right\}e^n = 0
    $$
    
    $\varepsilon_{ijk} $ 是反对称张量。矩阵形式
    
    $$
    \Lambda \vec{e} = 0 
    $$
    
    $$
    \Lambda
    \equiv \begin{bmatrix}
    -k^2 \alpha+\omega^2\left(\alpha+\beta\bar{E} \right) &0 &0 \\
    0 &-k^2\alpha+\omega^2\alpha &0 \\
    0 &0 &\omega^2\alpha
    \end{bmatrix} 
    $$
    
    由 $\mathrm{det}(\Lambda)=0 $ 可知\textcolor{red}{电场有两种模式}。两种模式定义了色散关系。折射率定义为 $n\equiv k/\omega $,因此有两种不同的折射率
    
    $$
    n_\parallel = \sqrt{1+\frac{\beta }{\alpha } \bar{E}^2},\quad
    n_\perp = 1 
    $$
    
    \textcolor{red}{不同偏振的电磁波有不同的速度} $v_\parallel=n_\parallel^{-1},v_\perp=1 $,这里平行和垂直极化是相对于背景匀强电场而言的。这种真空双折射现象是由模型的非线性性造成的。当 $\eta\to 0$ 时,$\beta\to 0$,真空双折射现象也消失。
\end{frame}

\subsection{拉格朗日量的因果性和幺正性条件}

\begin{frame}{拉格朗日量的因果性和幺正性条件}
    只有当因果性条件和幺正性条件都成立时,NLED 模型才是合理的。
    若拉氏量满足不等式:

    $$
    \boxed{
    L_F \leqslant 0,
    L_{FF}\geqslant 0,
    L_F + 2F L_{FF} \leqslant 0
    }
    $$
    
    则背景场上元激发的群速度不超过真空光速,且动能非负。电场部分取 $B=0 $,拉氏量 $\displaystyle{L(F)=-\frac{\left(aF+1 \right)^m }{\delta(bF+1)^n } \left(\beta F \right)^p }$ ,前两个不等式给出
    
    $$
    n\geqslant m+1,a\leqslant 0,b\geqslant 0
    $$
    
    在此基础上,第三个不等式自动满足。磁场部分取 $E=0 $,类似可得
    
    $$
    n\geqslant m+1,a\geqslant 0,b\leqslant 0
    $$
    
    $n=m+1,a=-3b $ 时的因果性和幺正性条件
    
    $$
    L_F
    =\frac{\gamma\left(-1+7\eta F \right) }{\left(1+\eta F \right)^3 },
    L_{FF}
    =\frac{2\gamma \eta\left(5-7\eta F \right) }{\left(1+\eta F \right)^4 } 
    $$
    
    $$
    L_F + 2FL_{FF} 
    =\gamma \frac{-\eta F\left(26-21\eta F \right) }{\left(1+\eta F \right)^4 }
    $$
\end{frame}

\begin{frame}
    仅电场部分,取 $B=0 $,因果性和幺正性条件三个不等式给出
    
    $$
    -\frac{2+7\eta E^2 }{\left(2-\eta E^2 \right)^3 } \leqslant 0,
    \frac{10+7\eta E^2 }{\left(2-\eta E^2 \right)^4 } \geqslant 0,
    \frac{-4-\eta E^2\left(521\eta E^2 \right) }{\left(2-\eta E^2 \right)^4 } \leqslant 0
    $$
    
    
    所有情况都有 $E<\sqrt{2/\eta} $,$E_{\max}=\sqrt{\eta/2} $。因此,\textcolor{red}{对于电场所有允许的取值,因果性和幺正性条件都得到满足。}
    
    分析磁场,取 $E=0 $,前两个不等式给出
    
    $$
    \left(-1+\frac{7 }{2 } \eta B^2 \right)\leqslant 0,
    \left(5-\frac{7 }{2 } \eta B^2 \right)\geqslant 0
    $$
    
    可以得到 $F<1/7\eta $。第三不等式给出
    
    $$
    -2+\eta B^2\left(26-\frac{21 }{2 } \eta B^2 \right)\leqslant 0
    $$
    
    得到 $(13-2\sqrt{37})/21<\eta F<(13+2\sqrt{37})/21 $。因此,\textcolor{red}{只有磁场取一定范围内的值时因果性和幺正性条件才得到满足。}
\end{frame}

\subsection{新 NLE 与广义相对论的耦合}

\begin{frame}{新 NLE 与广义相对论的耦合}
    通过作用量把拉氏量 $L(F) $ 与引力进行最小耦合

    $$
    I
    =\int \mathrm{d}^4 x\sqrt{-g}\left(\frac{R }{\kappa } + L(F) \right) 
    $$
    
    其中 $R $ 为里奇标量。变分可得运动方程
    
    $$
    \nabla_\mu\left(\frac{\partial L }{\partial F } F^{\mu\nu} \right)
    =0,\quad
    R_{\mu\nu} - \frac{1 }{2 } g_{\mu\nu} R
    =\kappa T_{\mu\nu}
    $$
    
    其中 $R_{\mu\nu} $ 为里奇张量,$T_{\mu\nu} $ 为 Hilbert 能量-动量张量,在弯曲时空表达为
    
    $$
    T_\mu^\nu
    =L\delta_\mu^\nu - L_F F_{\mu\lambda} F^{\nu \lambda}
    $$
    
    考虑球对称静态时空,线元
    
    $$
    \mathrm{d}s^2
    =-f(r)\mathrm{d}t^2 + \frac{1 }{f(r) } \mathrm{d}r^2 + r^2\left(\mathrm{d}\theta^2+\sin^2\theta\mathrm{d}\phi^2 \right)
    $$
    
    假设 $F_{tr},F_{\theta\phi} $ 在 $F_{\mu\nu} $ 中非零,$F_{tr}=-F_{rt} $ 代表径向电场,$F_{\theta\phi}=-F_{\phi\theta} $ 代表径向磁场。应力能动-张量非零分量
    
    $$
    T_t^t = T_r^r
    =L(F) - L_F F_{tr}F^{tr},
    T_\theta^\theta
    =T_\phi^\phi
    =L(F) - L_F F_{\theta\phi}F^{\theta\phi}
    $$
    
    下面只关注纯磁场解和纯电场解。
\end{frame}

\begin{frame}{磁场正则黑洞解 }
    纯磁场解来自 $F_{tr=0} $,非零麦克斯韦张量分量 $F_{\theta\phi}=-q_m \sin\theta $,$q_m $ 为常数,可理解为一个磁单极子的总荷量,导致径向磁场 $B_r=q_m/r^2 $,麦克斯韦不变量 $F=q_m^2/2r^4 $;$r=0 $ 处奇异

    磁单极子能-动张量
    
    $$
    T_t^t
    =T_r^r
    =\frac{\gamma q_m^2\left(3\eta q_m^2-2r^4 \right) }{\left(2r^4+\eta q_m^2 \right)^2 }
    $$
    
    $$
    T_\theta^\theta
    =T_\phi^\phi
    =\frac{\gamma q_m^2\left(4 r^8-2\eta q_m^2 r^4+3\eta^2q_m^4 \right) }{\left(2r^4+\eta q_m^2 \right)^3 }
    $$
    
    由线元可得爱因斯坦张量
    
    $$
    G_\mu^\nu
    =\mathrm{diag}\left[\frac{f' }{r } + \frac{f-1 }{r^2 } ,\frac{f' }{r } + \frac{f-1 }{r^2 } , \frac{f'' }{2 } + \frac{f' }{r } , \frac{f'' }{2 } + \frac{f' }{r }  \right]
    $$
    
    $' $ 代表度规函数 $f(r) $ 的径向微分。爱因斯坦非线性麦克斯韦方程 $tt $ 或 $rr $ 分量简化为
    
    $$
    \frac{f' }{r } + \frac{f-1 }{r^2 } 
    =\kappa \frac{\gamma q_m^2\left(3\eta q_m^2-2r^4 \right) }{\left(2r^4+\eta q_m^2 \right)^2 }
    $$
    
    解上面方程可得度规函数
\end{frame}

\begin{frame}
    $$
    f(r)
    =1+\frac{c_0 }{r } + \frac{\kappa \gamma q_m^2 r^2 }{2r^4+\eta q_m^2 } 
    $$
    
    $c_0 $ 是积分常数。取
    
    $$
    c_0=0,
    \gamma=-\frac{2b_0^2 }{\kappa q_m^2} ,
    \eta=\frac{2g^4 }{q_m^2 }
    $$
    
    $b_0,g $ 是长度量纲常数。线元可改写为
    
    $$
    \mathrm{d}s^2
    =-\left(1-\frac{b_0^2r^2 }{r^4+g^4 }  \right)\mathrm{d}t^2 + \left(1-\frac{b_0^2r^2 }{r^4+g^4 }  \right)^{-1}\mathrm{d}r^2 + r^2\left(\mathrm{d}\theta^2+\sin^2\theta \mathrm{d}\phi^2 \right)
    $$
    
    当 $r$ 趋于无穷大,时空度规渐近平坦
    
    $$
    g_{tt}\to -1,g_{rr}\to 1 \quad \mathrm{as}\quad r\to\infty
    $$
    
    对于很小的 $r$,其行为与 de-Sitter 时空相似
    
    $$
    g_{tt}\to -\left(1-c^2r^2 \right),\quad g^{rr} \to \left(1-c^2r^2 \right)\quad \mathrm{as}\quad r\to 0
    $$
    
    $g_{tt}=0$ 给出黑洞视界的位置。若度规参数满足 $0<g<0.5 b_0^2 $,则上面几何代表一系列双视界黑洞;当 $g^2=0.5b_0^2$,得到单视界黑洞;若 $g^2>0.5b_0^2$,则黑洞没有视界。特别地,当 $g^2=0$,黑洞有一个视界。
\end{frame}

\begin{frame}{曲率张量和不变量的正则性}
    可通过黎曼和里奇张量各分量是否发散来判断时空的正则性、奇异性。坐标基底,非零黎曼曲率张量分量

    $$
    R^0_{~~ 110} 
    =-\frac{b_0^2\left(3r^8 - 12 g^4 r^4 + g^8 \right) }{\left(r^4+g^4 \right)^3 } 
    $$
    
    $$
    R^0_{~~220} = R^0_{~~330} = R^2_{~~112} = R^3_{~~113}
    =\frac{b_0^2\left(r^4-g^4 \right) }{\left(r^4+g^4 \right)^2 }
    $$
    
    $$
    R^3_{~~223}
    =-\frac{b_0^2 }{r^4+g^4 } 
    $$
    
    非零 Ricci 张量分量
    
    $$
    R_{00} = -R_{11} = -\frac{b_0^2\left(r^8-12g^4r^4+3g^8 \right) }{\left(r^4+g^4 \right)^3 } 
    $$
    
    $$
    R_{22} = R_{33} = \frac{b_0^2 \left(3g^4-r^4 \right) }{\left(r^4+g^4 \right)^2 }
    $$
    
    当 $r\to 0 $,两个张量的分量都有限;当 $r\to \infty $,所有分量趋于零,这保证时空是渐进平坦的。
\end{frame}

\begin{frame}
    只有当曲率标量都正则时时空才是正则的。因此考察三个标量不变量:
    
    Ricci scalar
    
    $$
    R = g^{\mu\nu}R_{\mu\nu}
    =\frac{4b_0^2\left(3g^8-5g^4r^4 \right) }{\left(r^4+g^4 \right)^3 }
    $$
    
    Ricci contraction
    
    $$
    R_{\mu\nu} R^{\mu\nu}
    =\frac{4b_0^2 \left(r^{16}-14g^4r^{12}+74g^8r^8-30g^{12}r^4+9g^{16} \right) }{\left(r^4+g^4 \right)^6 }
    $$
    
    Kretschmann scalar
    
    $$
    K
    =R_{\mu\nu\lambda\delta} R^{\mu\nu\lambda\delta}
    =\frac{8\left(3g^{16}-10g^{12}r^4+74 g^8 r^8-34g^4r^{12}+7r^{16} \right)b_0^4 }{\left(r^4+g^4 \right)^6 }
    $$
    
    在 $r\to 0$ 时,三个不变量也都有限。因此\textcolor{red}{上述几何是正则的}。
\end{frame}

\begin{frame}{能量条件}
    定义 $\rho=-T_t^t,\tau=t_r^r,p=T_\theta^\theta=T_\phi^\phi $

    $$
    \rho=-\tau
    =\frac{b_0^2\left(3g^4-r^4 \right) }{\kappa \left(r^4+g^4 \right)^2 } 
    $$
    
    $$
    p
    =-\frac{b_0^2\left(3g^8-12g^4r^4+r^8 \right) }{\kappa \left(r^4+g^4 \right)^3 }
    $$
    
    零能量条件(Null Energy Condition)要求 $\rho+\tau \geqslant 0,\rho+p\geqslant 0 $,第一个在 $\rho+\tau=0 $ 时自动满足;
    
    $$
    \rho+p
    =\frac{2b_0^2 r^4\left(r^4-7g^4 \right) }{\kappa \left(r^4+g^4 \right)^3 }
    $$
    
    若第二个不等式得到满足,则只能 $r^4\leqslant 7g^4 $,即\textcolor{red}{并非所有 $r$ 的取值都是合理的。}
\end{frame}

\begin{frame}{裸电奇点解}
    纯电场解 $B=0,F=-E^2/2 $,类似可计算 Ricci scalar

    $$
    R
    =g_{\mu\nu} R^{\mu\nu}
    =\frac{4 }{r^2 } - \frac{4f(r) }{r^2 } - \frac{8f'(r) }{r } - 2f''(r) 
    $$

    通过变量替换 $\tilde{E}=E^2$ 可以写为
    
    $$
    R
    =-8\kappa \left(L-FL_F \right)
    =-8\gamma\kappa \eta\tilde{E}^2\left(10+3\eta\tilde{E} \right) \bigg/ \left(-2+\eta\tilde{E} \right)^3 
    $$
    
    Ricci contraction
    
    $$
    R_{\mu\nu} R^{\mu\nu}
    =8\left[\frac{f'(r) }{r } + \frac{f(r)-1 }{r^2 }  \right]^2 + 8 \left[\frac{f'(r) }{r } + \frac{f''(r) }{2 }  \right]^2
    $$
    
    $$
    \begin{aligned}
        R_{\mu\nu} R^{\mu\nu}
        &=8\kappa\left[\left(L - 2FL_F \right)^2 + L^2 \right] \\
        &=\frac{16\kappa \gamma^2 \tilde{E}^2 \left\{16+\eta \tilde{E}\left[112+\eta\tilde{E}\left(296+3\eta \tilde{E}\left(20+3\eta\tilde{E} \right) \right) \right] \right\} }{\left(-2+\eta\tilde{E} \right)^6 }
    \end{aligned}
    $$ 
    在 $\tilde{E}=2/\eta $ 或 $r=0$ 处三个标量发散,或者说 \textcolor{red}{$r=0$ 是个时空奇点}。可以计算,Kretschmann 标量 $R_{\mu\nu\lambda\delta}R^{\mu\nu\lambda\delta}  $ 在 $r=0$ 处也发散。因此,\textcolor{red}{由拉氏量控制的非线性电单极子与爱因斯坦引力的最小耦合导致了裸电奇点。}
\end{frame}

\subsection{总结}

\begin{frame}{总结}

    新型非线性电动力学模型的有效性:
\begin{itemize}
    \item 该NLE模型引入了两个具有维度的参数,能够在弱场极限下表现出类似于Born-Infeld拉格朗日量的行为。

    \item 模型成功消除了点电荷的自能发散问题,确保了电场在某一最大值下的有限性,从而避免了传统线性电动力学中的问题。
    
    \item 在外部均匀电场的影响下,模型展示了真空双折射现象,表明其在电磁现象中的新颖性。
\end{itemize}

因果性和幺正性条件:

\begin{itemize}
    \item 研究表明,在所有背景电场下,模型满足因果性和单一性条件。然而,对于磁场,因果性和幺正性条件仅在有限的磁场范围内成立,这表明模型在某些情况下可能存在局限性。
\end{itemize}

引力耦合的影响:

\begin{itemize}
    \item 探讨了将NLE拉格朗日量与爱因斯坦广义相对论(GR)最小耦合的可能性,提出了通过引入引力场来寻找不同的时空解。

    \item 结果显示,耦合后的模型可以产生常规黑洞或裸奇点,具体取决于源是非线性磁单极子还是电荷。
\end{itemize}


时空和场奇点的挑战:

\begin{itemize}
    \item 尽管NLE模型在某些情况下能够生成常规解,但仍然存在未解决的奇点问题。源场或几何中的奇点问题需要进一步研究,以实现时空和场奇点的同时解决。
\end{itemize}
\end{frame}

\section{Nonlinear Electrodynamics in f(T) Gravity and Generalized Second Law of Thermodynamics}



\subsection{f(T)引力和NLED基础}

\begin{frame}{f(T)引力基础}
    $f(T)$ 引力是一种修改引力理论,它是对爱因斯坦广义相对论的推广。与广义相对论基于黎曼几何不同,$f(T)$ 引力基于 \textbf{Teleparallel 几何}(Teleparallel Geometry),并通过修改 \textbf{挠率标量}(Torsion Scalar)$T$ 来构建引力理论。

    在广义相对论中,引力被描述为时空的弯曲,基于黎曼几何,使用 \textbf{曲率标量} $R$ 作为基本量。而在 Teleparallel 引力中,引力被描述为时空的 \textbf{挠率}(Torsion),而不是曲率。Teleparallel 引力的基本量是 \textbf{挠率标量} $T$,它通过 \textbf{魏森博克联络}(Weitzenböck Connection)定义。

\end{frame}

\begin{frame}

    $f(T)$ 引力中的基本元素是四分量场 $h_a(x^\mu)$,其中英文字母标记切空间,希腊字母标记时空。$h_a=h_a^\mu\partial_\mu$,分量满足

    $$
    h_\mu^a h_b^\mu = \delta_b^a,\quad h_\mu^a h_a^\nu = \delta_\mu^\nu.
    $$

    其与度规张量 $g_{\mu\nu}$ 的关系为 $ g_{\mu\nu}=\eta_{ab}h_\mu^ah_\nu^b$,其中 $\eta_{ab}=\mathrm{diag}(1,-1,-1,-1)$ 是切空间中的闵氏度规。借助魏森博克联络 $\Gamma_{~~\mu\nu}^{\lambda}=h_a^\lambda \partial_\nu h_\mu^a$,扭率张量 $T^\rho_{~~\mu\nu}$ 和张量 $S_\rho^{~~\mu\nu}$ 可定义为

    $$
    T^{\lambda}_{~~\nu\mu} = \Gamma^\lambda_{~~\nu\mu} - \Gamma^\lambda_{~~\mu\nu} = h_a^\lambda\left(\partial_\nu h_\mu^a - \partial_\mu h_\nu^a \right),
    $$

    $$
    S_\rho^{~~\mu\nu} = \frac{1 }{2 } \left(K^{\mu\nu}_{~~~~\rho} + \delta_\rho^\mu T^{\theta\nu}_{~~~~\theta} - \delta_\rho^\nu T^{\theta\mu}_{~~~~\theta} \right),
    $$

    其中,$K^{\mu\nu}_{~~~~\rho}=-\frac{1}{2}(T^{\mu\nu}_{~~~~\rho}-T^{\nu\mu}_{~~~~\rho}-T^{\mu\nu}-T_\rho^{~~\mu\nu}) $

    $f(T)$ 引力的作用量

    $$
    S = \frac{1}{2\kappa^2}\int \mathrm{d}^4 x [ef(T)+L_m]
    $$

    其中,$e=\sqrt{-g},\kappa^2=8\pi G$,$G$ 是引力常量,$L_m$ 是宇宙中物质拉格朗日密度。
    
\end{frame}

\begin{frame}
    对作用量变分可得场方程

    $$
    \left[e^{-1}\partial_\mu \left(e S_a^{~~\mu\nu} + h_a^\lambda T^\rho_{~~\mu\lambda} S_\rho^{~~\nu\mu} \right) \right] f_T + S_a^{~~\mu\nu} \partial_\mu(T) f_{TT} + \frac{1 }{4 } h_a^\nu f
    =\frac{1 }{2 } \kappa^2 h_a^\rho T_\rho^\nu
    $$
    
    其中,$f_T=\mathrm{d}f/\mathrm{d}T,f_{TT}=\mathrm{d}^2f/\mathrm{d}T^2$,$T_\rho^\nu$ 是理想流体的能-动张量。平坦 FRW 宇宙线元

    $$
    \mathrm{d}s^2
    =\mathrm{d}t^2 - a^2(t)\left(\mathrm{d}x^2+\mathrm{d}y^2+\mathrm{d}z^2 \right),
    $$

    其中,\textcolor{red}{$a$ 是依赖于时间的尺度因子,表示某时刻相对参考时刻宇宙的线性尺度变化。}相应 $h_\mu^a=\mathrm{diag}(1,a,a,a)$。$f(T)$ 引力中,\textcolor{red}{修改 Friedmann 方程描述了均匀且各向同性宇宙的演化}:

    $$
    \boxed{
    12 H^2 f_T + f = 2\kappa^2 \rho_t,
    }
    $$

    $$
    \boxed{
    48H^2\dot{H} f_{TT} - \left(12 H^2 + 4\dot{H} \right) f_T - f= 2\kappa^2 p_t,
    }
    $$

    其中,\textcolor{red}{$H=\dot{a}/a$ 是哈勃参数,描述了宇宙的膨胀速度;$\rho_t,p_t$ 分别是宇宙的总能量密度和压力。}
\end{frame}

\begin{frame}{非线性电动力学基础}
    对任意物理量 $Y$,体积空间平均值定义为

    $$
    \bar{Y}
    =\lim_{V\to V_0} \frac{1 }{V } \int Y\sqrt{-g} \mathrm{d}^3 x,
    $$

    其中,$g$ 是度规行列式,$\displaystyle{V=\int\sqrt{-g}\mathrm{d}^3 x } $,电场和磁场的平均值

    $$
    \bar{E}_i = 0,\quad
    \bar{B}_i = 0,\quad
    \overline{E_i B_i} = 0,\quad
    \overline{E_i E_j} = -\frac{1 }{3 } E^2 g_{ij},\quad
    \overline{B_i B_j} = -\frac{1 }{3 } B^2 g_{ij},
    $$

    利用电磁场不变量 $F,F^*$ 来表达拉氏量,保留至二阶

    $$
    \mathcal{L}=-\frac{1 }{4 } F + \omega F^2 + \eta_0 F^{*2},
    $$

    其中,$F=F_{\mu\nu} F^{\mu\nu}=2(B^2-E^2),F^*=F_{\mu\nu}^*F^{\mu\nu}=-4\vec{E}\cdot\vec{B}$,$\omega_0,\eta_0$ 是任意常数。

    相应能-动张量

    $$
    T_{\mu\nu}
    =-4\mathcal{L}_F F_\mu^{~~\alpha} F_{\alpha\nu} + \left(F^* \mathcal{L}_{F^*} - \mathcal{L} \right) g_{\mu\nu},
    $$

\end{frame}

\begin{frame}
    结合电磁场平均值以及理想流体 $T_{\mu\nu}=(\rho+p)u_\mu u_\nu-p g_{\mu\nu}$,可得能量密度 $\rho$ 和压力 $p$

    $$
    \rho = -\mathcal{L} - 4E^2 \mathcal{L}_F,
    $$

    $$
    p = \mathcal{L} + \frac{4 }{3 } \left(E^2-2B^2 \right)\mathcal{L}_F,
    $$

    考虑等离子体中迅速衰减至零的电场,进一步有

    $$
    \rho_B = \frac{1 }{2 } B^2\left(1-8\omega B^2 \right),
    $$


    $$
    p_B = \frac{1 }{6 } B^2 \left(1-40\omega_0 B^2 \right),
    $$

    当 $\omega_0=\eta_0=0$ 时,拉氏量退化为线性麦克斯韦拉氏量,能动张量也退化

    $$
    \mathcal{L} = -\frac{1 }{4 } F,\quad
    T_{\mu\nu} = F_\mu^{~~\alpha} F_{\alpha\nu} + \frac{1 }{4 } F g_{\mu\nu}.
    $$

    对于非线性过程

    $$
    \rho = 3p = \frac{1 }{2 } \left(E^2 + B^2 \right),
    $$

    这表明\textcolor{red}{宇宙充满普通辐射(ordinary radiations),具有正压力。}

\end{frame}

\subsection{宇宙学参数和热力学}

\begin{frame}{宇宙学参数}
    修改 Friedmann 场方程可写为

    $$
    \frac{3 H^2 }{\kappa^2 } = \rho_t,\quad
    -\frac{2\dot{H} }{\kappa^2 }  = \rho_t + p_t,
    $$
    
    其中,$\rho_t=\rho_m+\rho_B+\rho_T,p_t=p_m+p_B+p_T$,下标 $m,B,T$ 分别代表物质,磁场和扭率的贡献。

    $$
    \rho_T = \frac{1 }{2\kappa^2 } \left(-12 H^2 f_T - f + 6H^2 \right),
    $$

    $$
    p_T = -\frac{1 }{2\kappa^2 } \left[48\dot{H}H^2 f_{TT} - \left(12 H^2 + 4\dot{H} \right) f_T - f + 6 H^2 + 4\dot{H} \right],
    $$

    为了方便,取 $p_m=0$,能量守恒方程

    $$
    \dot{\rho}_m + 3H \rho_m = 0,
    $$

    $$
    \dot{\rho}_B + 3H\left(\rho_B + p_B \right) = 0,
    $$

    $$
    \dot{\rho}_T + 3H\left(\rho_T + p_T \right) = 0,
    $$

\end{frame}

\begin{frame}
    第一条方程解得

    $$
    \rho_m = \rho_{m0} a^{-3},
    $$

    其中,$a_{m0}$ 是任意常量。第二条方程解得

    $$
    B=\frac{B_0}{a^2},
    $$

    其中,$B_0$ 是任意常数。这表明\textcolor{red}{磁场能量密度的演化随宇宙的膨胀而衰减。}

    状态方程(equation of state, EoS)参数

    $$
    \begin{aligned}
        \omega_t
        &=\left\{-\frac{1 }{\kappa^2 } \left[48\dot{H}H^2 f_{TT} - \left(12 H^2 + 4\dot{H} \right) f_T - f + 6 H^2 + 4\dot{H} \right] + \frac{B^2 }{6 } \left(1-40\omega_0 B^2 \right) \right\} \\
        &\times \left\{\rho_{m0} a^{-3} + \frac{1 }{2\kappa^2 } \left[6 H^2 - f-12 H^2 f_T \right] + \frac{B^2 }{2 } \left[1-8\omega_0 B^2 \right] \right\}^{-1}
        \end{aligned}
    $$

    减速参数 $q$ 是宇宙膨胀加速度的度量,其由下式给出

    $$
    q = -1 - \frac{\dot{H} }{H^2 } .
    $$

    \textcolor{red}{负的 $q$ 意味着宇宙膨胀在加速,正的 $q$ 意味着宇宙膨胀在减速。}

\end{frame}

\begin{frame}
    当前情况下 $q_t=\frac{1}{2}(1+3\omega_t)$,于是
    $$
    \begin{aligned}
        2q_t
        &=1 + 3\left[-\frac{1 }{\kappa^2 } \left(48\dot{H} H^2 f_{TT} - \left(12 H^2 + 4\dot{H} \right) f_T - f + 6H^2 + 4\dot{H} \right) + \frac{B^2 }{6 } \left(1-40\omega_0 B^2 \right) \right] \\
        &\times \left[\rho_{m0}a^{-3} + \frac{1 }{2 } \left(6H^2-f-12H^2 f_T \right) + \frac{B^2 }{2 } \left(1-8\omega_0 B^2 \right) \right]^{-1}
    \end{aligned}
    $$

    这是 $f(T)$ 的EoS(Equation of State),可以通过几个 $f(T)$ 模型来检查这些宇宙学参数的行为。
\end{frame}

\begin{frame}{广义热力学第二定律(GSLT)}
    GLST说,\textcolor{red}{在视界里和视界上的总熵不随时间减少。}

    由热力学第一定律有克劳修斯关系 $-\mathrm{d}E=T_X\mathrm{d}S_X$,其中 $S_X=A/(4G)$ 是 Bekenstein 熵,$A=4\pi R_X^2$ 是视界面积,$X$ 是任意视界,$T_X=1/(2\pi R_X)$ 是霍金温度。Miao 等人发现在 $f(T)$ 引力中热力学第一定律被违背,这导致额外的熵增项 $S_P$;而在 $f_{TT}$ 很小时,热力学第一定律成立,这时熵 $S_X=(Af_T)/(4G)$,而与 $S_P$ 无关。下面采用更一般的方法来研究磁 $f(T)$ 引力框架下的 GSLT。熵对时间微分

    $$
    \frac{\mathrm{d}S_X }{\mathrm{d}t } + \frac{\mathrm{d}S_P }{\mathrm{d}t } = \frac{\pi R_X }{G } \left(2\dot{R}_X f_T + R_X\dot{T} f_{TT} \right).
    $$

    利用吉布斯方程找到视界熵正常熵(normal entropy) $S_I$ 的变化率

    $$
    \frac{\mathrm{d}S_I }{\mathrm{d}t } 
    =\frac{1 }{T_X } \left(\frac{\mathrm{d}E_I }{\mathrm{d}t } + p_t\frac{\mathrm{d}V }{\mathrm{d}t }  \right),
    $$

    其中,$E_I=\rho_t V,V=4\pi R_X^3/3$ 是视界体积。计算得

    $$
    \frac{\mathrm{d}S_I }{\mathrm{d}t } 
    =\frac{4\pi R_X^2 }{T_X } \left(\dot{R}_X - H R_X \right)\left(\rho_t+p_t \right).
    $$
    
\end{frame}

\begin{frame}
    总熵对时间的微分

    $$
    \begin{aligned}
        &\frac{\mathrm{d}S_X }{\mathrm{d}t } + \frac{\mathrm{d}S_P }{\mathrm{d}t } + \frac{\mathrm{d}S_I }{\mathrm{d}t } \\
        =&\frac{\pi R_X }{G } \bigg\{2\dot{R}_X f_T + R_X\dot{T}f_{TT} \\
        &+ 8\pi G R_X^2 \left[\rho_{m0}a^{-3} + \frac{1 }{\kappa^2 } \left(4\dot{H} T f_{TT} + 2\dot{H}\left(f_T-1 \right) \right) + \frac{2B_0^2 }{3a^4 } \left(1-\frac{16\omega_0 B_0^2 }{a^4 }  \right) \right] \\
        &\times \left[\dot{R}_X - H R_X \right] \bigg\}
    \end{aligned}
    $$

    \textcolor{red}{GSLT给出 $(\dot{S}_X+\dot{S}_I+\dot{S}_P)\geqslant 0$},下面讨论两种常用的宇宙学视界:哈勃视界(Hubble Horizon)和事件视界(Event Horizon)。

\end{frame}

\begin{frame}{哈勃视界(Hubble Horizon)}

    哈勃视界定义为光在宇宙膨胀过程中能够到达的最远距离。

    假设FRW宇宙热力学系统的边界被处于平衡状态的表观视界(apparent horizon)占据。对于平坦的 FRW,它退化为半径为 $R_H$ 的哈勃视界

    $$
    R_H = \frac{1 }{H } ,\quad
    \dot{R}_H = -\frac{\dot{H} }{H^2 }
    $$

    把任意视界 $X$ 替换为 $H$ 得

    $$
    \begin{aligned}
        &\frac{\mathrm{d}S_X }{\mathrm{d}t } + \frac{\mathrm{d}S_P }{\mathrm{d}t } + \frac{\mathrm{d}S_I }{\mathrm{d}t } \\
        =&-\frac{\pi }{G H} \bigg\{\frac{2\dot{H} }{H^2 } f_T + 12\dot{H}f_{TT} +\frac{8\pi G }{H^2 }\left(1+\frac{\dot{H} }{H^2 }  \right) \\
        &\times \left[\rho_{m0}a^{-3} + \frac{1 }{\kappa^2 } \left(4\dot{H}T f_{TT} + 2\dot{H}\left(f_T-1 \right) \right) + \frac{2 B_0^2 }{3a^4 } \left(1-\frac{16\omega_0 B_0^2 }{a^4 }  \right) \right] \bigg\}.
        \end{aligned}
    $$

    这是宇宙中所有流体(包括尘埃物质、磁场和挠率贡献)在哈勃视界处的总熵的变化率。
    
\end{frame}

\begin{frame}{事件视界(Event Horizon)}

    在宇宙学中,事件视界是指从一个观察者的视角出发,能够观测到的宇宙范围。

    事件视界半径定义为

    $$
    R_E = a\int_0^{\infty} \frac{\mathrm{d}t }{a } ,\quad
    \dot{R}_E = H R_E - 1.
    $$

    把任意视界 $X$ 替换为 $E$ 得

    $$
    \begin{aligned}
        &\frac{\mathrm{d}S_E}{\mathrm{d}t} + \frac{\mathrm{d}S_I}{\mathrm{d}t} + \frac{\mathrm{d}S_P}{\mathrm{d}t} \\
        =&\frac{\pi}{G} \left(a\int_t^\infty \frac{\mathrm{d}t}{a}\right)\bigg[2\left(\dot{a}\int_t^\infty -1 \right) - 12H\dot{H} \left(a\int_t^\infty \frac{\mathrm{d}t}{a}\right) \\
        &+8\pi G\times \left(a\int_t^\infty\frac{\mathrm{d}t}{a}\right)^2\left(\left(\dot{a}\int_t^\infty\frac{\mathrm{d}t}{a}-1\right)-H\left(a\int_t^\infty\frac{\mathrm{d}t}{a}\right)\right) \\
        &\times \left\{\rho_{m0}a^{-3} + \frac{1}{\kappa^2} \left(4\dot{H}T f_{TT} + 2\dot{H}(f_T-1)\right) + \frac{2B_0^2}{3a^4}\left(1-\frac{16\omega_0B_0^2}{a^4}\right) \right\} \bigg]. 
    \end{aligned}
    $$

    这表示在平衡态下宇宙事件视界处总熵的变化率。

\end{frame}

\subsection{f(T)模型的一个例子}

\begin{frame}{f(T)模型的一个例子}
    考虑如下\textcolor{red}{极点形式的尺度因子}

    $$
    a(t)
    =a_0\left(t_s-t \right)^{-h},\quad
    h>0,\quad t_s\geqslant t
    $$

    哈勃参数 $H$,扭率(torsion)标量 $T$,$\dot{{H}}$ 分别为

    $$
    H = \frac{h }{t_s - t } ,\quad
    T = -\frac{6h^2 }{\left(t_s-t \right)^2 } ,\quad
    \dot{H} = \frac{h }{\left(t_s - t \right)^2 }.
    $$

    把上式代入修改 Friedmann 方程可得

    $$
    \begin{aligned}
        f(T)
        =&c_1\left(-\frac{T }{6h^2 }  \right)^{1/2} + \frac{2\kappa^2 \rho_{m0} }{a_0^3\left(3h+1 \right) } \left(-\frac{6h^2 }{T }  \right)^{3h/2} + \frac{\kappa^2 B_0^2 }{a_0^4\left(4h+1 \right) } \left(-\frac{6h^2 }{T }  \right)^{2h} \\
        &- \frac{8\kappa^2 B_0^4 \omega_0 }{a_0^8 \left(8h+1 \right) } \left(-\frac{6h^2 }{T }  \right)^{4h},
    \end{aligned}
    $$

    其中,$c_1$ 由边界条件确定。

\end{frame}

\begin{frame}
    取 $z=a_0/a - 1$,两个 EoS 参数

    $$
    \omega_t
    =\frac{20\kappa^2 B_0^4 \omega_0 }{9h^2a_0^8 } \left(1+z \right)^{(8h+2)/h} - \frac{\kappa^2 B_0^2 }{18h^2 a_0^4 } (1+z)^{(4h+2)/h} - \frac{2(3h+2) }{3h } ,
    $$

    $$
    q_t
    =\frac{10\kappa^2 B_0^4 \omega_0 }{3h^2 a_0^8 } (1+z)^{(8h+2)/h} - \frac{\kappa^2 B_0^2 }{12 h^2 a_0^4 } (1+z)^{(4h+2)/h} - \frac{5h+4 }{2h } ,
    $$

    \begin{figure}
        \centering
        \includegraphics[width=0.6\textwidth]{fig/fig11.png}
        \caption{$\omega_t$ 和 $q_t$ 与 $z$ 的关系}
    \end{figure}

    \textcolor{red}{该模型满足在高红移下 $f_T\to 0$ 的条件,符合宇宙加速膨胀的要求。}

    \textcolor{red}{对于较小的红移值 $z$,$q_t<0$,宇宙处于加速膨胀状态,这与现实一致,也与原始核合成(primordial nucleosynthesis)和宇宙微波背景约束一致。}


    
\end{frame}

\begin{frame}

    $$
    \begin{aligned}
        f_{TT}
        &=\frac{(1+z)^{4/h} }{36h^4 } \bigg[\frac{3h(3h+2)\kappa^2\rho_{m0} }{2a_0^3(3h+1) }(1+z)^3 + \frac{2h(2h+1)\kappa^2 B_0^2 }{a_0^4(4h+1) } (1+z)^4 \\
        &- \frac{32h(4h+1)\kappa^2 B_0^4 \omega_0 }{a_0^8 (8h+1) } (1+z)^8 - \frac{c_1 }{4(1+z)^{1/h} }   \bigg].
    \end{aligned}
    $$
    \begin{figure}
        \centering
        \includegraphics[width=0.8\textwidth]{fig/fig12.png}
        \caption{$f_{TT}$}
    \end{figure}   
\end{frame}

\begin{frame}{Hubble Horizon}
    $$
    \begin{aligned}
        &\frac{\mathrm{d}S_H }{\mathrm{d}t } + \frac{\mathrm{d}S_I }{\mathrm{d}t } \\
        =&-\frac{\pi(1+z)^{3/h} }{Gh^3 } \bigg[\frac{3h(3h+4)\kappa^2\rho_{m0} }{2a_0^3(3h+1) }(1+z)^3 + \frac{4(h+1)\kappa^2 B_0^2 }{3a_0^4(4h+1) } (1+z)^4 \\
        &- \frac{64h(2h+1)\kappa^2 B_0^4 \omega_0 }{3a_0^8 (8h+1) } (1+z)^8- \frac{c_1 }{4h(1+z)^{1/h} } \bigg] + \frac{2\pi }{G h^3 } (1+h)(1+z)^{1/h}.
    \end{aligned}
    $$

    \textcolor{red}{对于极点型尺度因子,在哈勃视界处,当 $z>8.2$ 或 $z<0$ 时 GSLT 成立。}

\end{frame}

\begin{frame}{Event Horizon}
    $$
    \begin{aligned}
        &\frac{\mathrm{d}S_E }{\mathrm{d}t } + \frac{\mathrm{d}S_I }{\mathrm{d}t } \\
        =&-\frac{\pi(1+z)^{3/h} }{Gh(1+h)^2 } \bigg[\frac{(3h+4)\kappa^2\rho_{m0} }{2a_0^3(3h+1) }(1+z)^3 + \frac{4(h+1)\kappa^2 B_0^2 }{3a_0^4(4h+1) } (1+z)^4 \\
        &- \frac{64(2h+1)\kappa^2 B_0^4 \omega_0 }{3a_0^8 (8h+1) } (1+z)^8 - \frac{c_1 }{4h\kappa^2(1+z)^{1/h} } \bigg] + \frac{2\pi h }{G (1+h)^3 } (1+z)^{1/h}
    \end{aligned}
    $$

    \begin{figure}
        \centering
        \includegraphics[width=0.5\textwidth]{fig/fig3.png}
        \caption{$\dot{S}_H+\dot{S}_I$,left for Hubble Horizon,right for Event Horizon}
    \end{figure}

    \textcolor{red}{对于极点型尺度因子,事件视界处,有同样的结果,当 $z>8.2$ 或 $z<0$ 时 GSLT 成立。}

\end{frame}

\begin{frame}{另一种形式的尺度因子}

    构造\textcolor{red}{幂律型尺度因子} $a(t)=a_0(t_s-t)^h$,与之对应的 $f(T)$ 模型
    
    $$
    \begin{aligned}
        f(T)
        &=c_2\left(-\frac{T }{6h^2 }  \right)^{1/2} + \frac{2\kappa^2\rho_{m0} }{a_0^3(1-3h) } \left(-\frac{T }{6h^2 }  \right)^{3h/2} + \frac{\kappa^2 B_0^2 }{a_0^4(1-4h) } \left(-\frac{T }{6h^2 }  \right)^{2h} \\
        &- \frac{8\kappa^2 B_0^4 \omega_0 }{a_0^8(1-8h) } \left(-\frac{T }{6h^2 }  \right)^{4h},
    \end{aligned}
    $$
    \begin{figure}
        \centering
        \includegraphics[width=0.6\textwidth]{fig/fig14.png}
        \caption{另一种尺度因子的 $\omega_t$ 和 $q_t$}
    \end{figure}
\end{frame}
    
\begin{frame}
    \begin{figure}
        \centering
        \includegraphics[width=0.5\textwidth]{fig/fig15.png}
        \caption{另一种尺度因子的 $f_{TT}$}
    \end{figure}
    \begin{figure}
        \centering
        \includegraphics[width=0.5\textwidth]{fig/fig16.png}
        \caption{另一种尺度因子的 $\dot{S}_H+\dot{S}_I$,left for Hubble Horizon,right for Event Horizon}
    \end{figure}

    \textcolor{red}{对于幂律型尺度因子,GSLT 在 Hubble 视界处的有效性取决于参数 $h$ 的取值。当 $h=5$,GSLT 对所有红移值 $z$ 都成立;而当 $h=2,3$,GSLT 只在特定红移值范围内成立。而对于事件视界,GSLT 对所有的 $h$ 值和红移值 $z$ 都成立。}

\end{frame}

\subsection{总结}

\begin{frame}
    文章在包含暗能量、尘埃物质和磁场贡献的 FRW 宇宙中,在 f(T) 引力框架下研究了 NLED。采用平均手段来保留 NLED 中时空的各向同性。在这种情况下,评估了宇宙总能量密度和压力的 EoS 和减速参数。开发了哈勃和事件视界的总熵的时间导数,以使用视界熵和吉布斯方程来研究 GSLT 的合法性(validity)。使用极点和幂律形式的尺度因子构建了 f(T) 模型。讨论了一些特定模型参数的图形行为。文章的结果总结如下:


    \begin{itemize}
        \item 第一个由极点尺度因子构建的 f(T) 模型的宇宙学参数代表一个在 z≤5.6 时加速膨胀的 phamtom dominated 的宇宙。对于更高的 z 值,膨胀率降低,磁场主导扭率贡献,代表减速膨胀的宇宙。
        \item 作了哈勃和事件视界的总熵的时间导数关于 $z$ 的图,以讨论 GSLT 对满足条件 $f_{TT} \ll 1 $  的模型的合法性。对于这两个视界,GSLT 对 $z >8.2$ 和 $z <0$ 都成立。
        \item 使用精确幂律比例因子构建第二个 $f(T)$ 模型。$\omega_t$ 和 $q_t$ 关系 $z$ 的关系图表明与第一个模型相同的行为。
        \item 第二个模型也满足条件 $f_{TT}\ll 1$,以借助热力学第一定律讨论 GSLT。图6显示了哈勃视界总熵的时间导数在 $h = 2,3$ 时 $z$ 的一定范围内的正行为,而 $h = 5$ 表示 GSLT 对所有 $z$ 值的合法性。对于事件视界,GSLT 对所有 $h$ 和 $z$ 值都合法。
        \item 值得一提的是,仅对于磁宇宙,当 $z \geqslant -0.1$ 时,事件视界的总熵的时间变化率保持正值,在此范围之外变为负值。另一方面,在我们的例子中,对于具有幂律比例因子的视界,它对于磁 f(T) 框架中的所有 $z$ 值都保持在正区域。哈勃视界在两种情况下都表现出总熵的时间导数的相似行为。对于更高的红移值,宇宙学参数表明,与磁场相比,扭转贡献变得微弱。它指向宇宙的早期减速阶段。
    \end{itemize}
\end{frame}

\section{Nonlinear electrodynamics and black holes}

\subsection{NLED 形式}

\begin{frame}{电磁场不变量}
    假设非线性电磁场可由电磁势 $A_\mu$ 描述

    $$
    F_{\mu\nu}
    =2A_{[\mu,\nu]}
    $$

    $F_{\mu\nu}$ 有一个不变量和一个伪不变量

    $$
    F = \frac{1 }{4 } F_{\alpha\beta}F^{\alpha\beta},\quad
    \tilde{G} = \frac{1 }{4 } F_{\alpha\beta} \tilde{F}^{\alpha\beta} 
    $$

    其中 $\tilde{F}^{\alpha\beta}=\left(\mathrm{i}/2\sqrt{-g} \right)\varepsilon^{\alpha\beta\gamma\delta}F_{\gamma\delta}$ 是 $F^{\alpha\beta}$ 的对偶。若 NLED 拉氏量在洛伦兹群作用下不变,则它依赖于 $F$ 和 $\tilde{G}$;同时,在弱场极限下应与经典线性理论相同。
\end{frame}

\begin{frame}{$(F,\tilde{G})$ 和 $(P,\tilde{Q})$ 框架}

    两个框架可通过勒让德变换联系

    $$
    P^{\alpha\beta}
    =2\frac{\partial L }{\partial F_{\alpha\beta} } 
    =\frac{\partial L }{\partial F } F^{\alpha\beta} + \frac{\partial L }{\partial \tilde{G} } \tilde{F}^{\alpha\beta}
    $$

$$
H
=\frac{1 }{2 } P^{\alpha\beta} F_{\alpha\beta} - L\left(F,G^2 \right)
$$

与 $P^{\alpha\beta} $ 有关的不变量:

$$
P = \frac{1 }{4 } P_{\alpha\beta} P^{\alpha\beta},\quad
\tilde{Q} = \frac{1 }{4 } P_{\alpha\beta}\tilde{P}^{\alpha\beta}
$$

哈密顿方程

$$
F^{\alpha\beta}
=2\frac{\partial H }{\partial P_{\alpha\beta} } 
=\frac{\partial H }{\partial P } P^{\alpha\beta} + \frac{\partial H }{\partial Q } \tilde{P}^{\alpha\beta}
$$
    
\end{frame}

\begin{frame}
    NLED 与引力耦合作用量:

    $$
    S
    =\int \mathrm{d}^4 x\sqrt{-g} \left\{R(16\pi)^{-1}-L \right\}
    $$
    
    $R $ 是曲率标量;$g:=\mathrm{det}\left|g_{\mu\nu} \right| $
    
    $$
    L
    =\frac{1 }{2 } P^{\mu\nu}P_{\mu\nu} - H\left(P,\tilde{Q} \right)
    $$

    能动张量和曲率标量

$$
4\pi T_{\mu\nu}
=H_{,P} P_{\mu\alpha} P_\nu^\alpha - g_{\mu\nu}\left(2P H_{,P} + \tilde{Q} H_{,\tilde{Q}} - H \right)
$$

$$
R
=8\left(P H_{,P} + \tilde{Q} H_{,\tilde{Q}} - H \right) 
$$

其中,$\partial H/\partial P = H_{,P} $ 

Born-Infeld 非线性电动力学由结构函数 $H(P,\tilde{Q}) $ 给出:

$$
H = b^2\left(1-\sqrt{1-2P/b^2+\tilde{Q}^2/b^4} \right) 
$$

其中,$b $ 是最大场强,是 BI 理论中的参数。
\end{frame}

\begin{frame}{NLED能量条件}

    类时矢量 $V^\alpha,V_\alpha V^\alpha<1 $ ,local energy density 非负 $T_{\mu\nu}V^\mu V^\nu\geqslant 0 $;local energy flow 矢量是非类空的要求 $T_{\alpha\beta}T_\gamma^\alpha V^\beta V^\gamma\leqslant 0 $

    $$
    H_{,P}>0,\quad
    \left(P H_{,P} + \tilde{Q} H_{,\tilde{Q}} - H \right) \geqslant 0
    $$
    
    strong energy condition(SEC) $R_{\mu\nu}V^\nu V^\nu\geqslant 0 $,结合爱因斯坦方程得
    
    $$
    R_{\mu\nu}V^\mu V^\nu 
    =8\left(T_{\mu\nu} V^\mu V^\nu + \frac{T }{2 }  \right) \geqslant 0
    $$
    
\end{frame}

\subsection{NLED 黑洞}

\begin{frame}{NLED 黑洞}
    SSS(静态球对称)线元

    $$
    \mathrm{d}s^2 = -\psi\mathrm{d}t^2 + \psi^{-1} \mathrm{d}r^2 + r^2\left(\mathrm{d}\theta^2+\sin^2\theta\mathrm{d}\phi^2 \right)
    $$
    
    HI发现一种解为
    
    $$
    \psi
    =1-\frac{8\pi }{r } \int_{0}^{r} \left(\sqrt{r^4+1} - r^2 \right)\mathrm{d}r
    $$
    
    上面的解有正则奇点;另一个正则解 $D_{,r}=1/r,E_{,r}=r^2/\left(r^4+1 \right) $
    
    度规函数
    
    $$
    \psi_{HI}
    =1-\frac{k }{r } + \frac{8\pi\gamma }{r } \int_{0}^{r} \left[r^2\ln\left(\frac{r^4 }{1+r^4 }  \right) \right] \mathrm{d}r
    $$

    
\end{frame}

\begin{frame}
    对 SSS 线元,PT 发现

    $$
    \psi_{PT}
    =1-\frac{d }{r } + \frac{8\pi }{r } \int_{0}^{r} H(x) x^2\mathrm{d}x
    $$

    点电荷电磁场

$$
F_{\mu\nu}
=-\frac{e }{r^2 } \frac{\partial H(P,0) }{\partial P } 2\delta_\mu^{[0}\delta_{\nu}^{r]}
$$

其中 $P=-e^2/4r^2$;若积分存在且有限,则当 $r\to 0$ 时电磁场张量是有限的;此外,当 $r$ 很大时,$H(P,0)\approx P$。这些条件保证了此解有较好的行为。

\end{frame}

\begin{frame}{BI 黑洞和 EBIon}

    SSS 时空的 \textcolor{red}{EBI 解}由度规函数 $\psi_{BI}(r)$ 给出
$$
\Psi_{BI}(r)
=1-\frac{2m }{r } + \frac{2 }{3 } b^2\left(r^2-\sqrt{r^4+a^4} \right) + \frac{4g^2 }{3r } G(r)
$$

$$
G'(r)
=-\left(r^4+a^4 \right)^{-1/2}
$$

其中,$m$ 是质量参数,$g$ 是磁荷,$a^4=g^2/b^2$,$b$ 是 BI 模型参数。电磁场非零分量为

$$
F_{rt}
=g\left(r^4+a^4 \right)^{-1/2},\quad
P_{rt}
=\frac{g }{r^2 }
$$

\textcolor{red}{黑洞解}为

$$
G(r)
=\int_{r}^{\infty} \frac{\mathrm{d}s }{\sqrt{s^4+a^4} } 
=\frac{1 }{2a } \mathbb{F} \left[\arccos\left(\frac{r^2-a^2 }{r^2+a^2 }  \right) , \frac{1 }{\sqrt{2} }  \right]
$$

其中,$\mathbb{F}$ 是第一类椭圆积分。\textcolor{red}{这个解在 $r=0$ 处发散。}另一方面,\textcolor{red}{粒子解}为

$$
G(r)
=\int_{0}^{r} \frac{-\mathrm{d}s }{\sqrt{s^4+a^4} } 
=-\frac{1 }{2a } \mathbb{F}\left[\arccos\left(\frac{a^2-r^2 }{a^2+r^2 } , \frac{1 }{\sqrt{2} }  \right) \right]
$$

\textcolor{red}{这个解在 $r=0$ 处有限。}

\end{frame}

\begin{frame}
    二者的联系为
$$
\int_r ^{\infty} \frac{\mathrm{d}s }{\sqrt{s^4+a^4} } + \int_0^r \frac{\mathrm{d}s }{\sqrt{s^4+a^4} } 
=\frac{1 }{a } \mathrm{K}\left[\frac{1 }{2 }  \right] 
$$

其中,$\mathrm{K}\left[\frac{1 }{2 } \right]$ 是第一类完全椭圆积分。\textcolor{red}{在 $r$ 很大时,解趋于 RN 解;当 BI 模型参数 $b\to\infty$,得到线性电磁场 RN 解;在无电荷极限 $b=0$ 下,得到史瓦西黑洞解。}

\end{frame}

\begin{frame}{BI黑洞中测试粒子的轨迹}

    测试粒子运动过程中有两个守恒量:能量 $E$ 和角动量 $l$;若把粒子运动限制在赤道面 $\theta=\pi/2$ 上,则可用有效势进行分析。

    对于有质量测试粒子,其轨迹由洛伦兹方程给出

    $$
    \frac{\mathrm{d}^2 x^\nu }{\mathrm{d}\tau^2 } + \Gamma_{\alpha\beta}^\nu \frac{\mathrm{d}x^\alpha }{\mathrm{d}\tau } \frac{\mathrm{d}x^\beta }{\mathrm{d}\tau } 
    =-\frac{\varepsilon }{\mu } F_{\sigma}^{\nu} \mathrm{d}x^\sigma \mathrm{d}\tau
    $$

    其中,$\varepsilon$ 是电荷量,$\mu$ 是质量,$\tau$ 是沿轨迹的仿射参数。利用两个守恒量可得

    $$
    \dot{r}^2 + \psi\left(\frac{l^2 }{r^2 } + 1 \right) - \left\{E+\frac{\varepsilon g }{\mu } \sqrt{\frac{b }{4g } } \mathbb{F} \left[\arccos\left(\frac{r^2-g/b }{r^2+g/b }  , \frac{1 }{\sqrt{2} }  \right) \right] \right\}^2=0
    $$

    与 $\dot{r}^2/2+U_{\mathrm{eff}}(E,l,r)=0$ 比较可得有效势。对于\textcolor{red}{光子},类似有

    $$
    \dot{r}_{ph}
    =\sqrt{E^2 - \frac{\Psi_{BI} l^2 }{r^2 } \left(1+\frac{a^4 }{r^4 }  \right)^{-1}} 
    $$

    \textcolor{red}{可以发现 $(\mathrm{d}r/\mathrm{d}t)_{\mathrm{photon}}<(\mathrm{d}r/\mathrm{d}t)_{\mathrm{grav}}$,即非线性效应导致光的传播速度小于引力波的传播速度。这可以解释为光子在电介质中传播导致的。}
    
\end{frame}

\subsection{NLED 黑洞热力学}

\begin{frame}{NLED 黑洞热力学}
    考虑静态球对称黑洞,热力学第一定律给出

    $$
    \delta M_{\Delta} = \frac{\kappa }{8\pi } \delta a_{\Delta} + \Phi_\Delta \delta Q_\Delta
    $$

    其中,$\kappa$ 为视界处的表面引力,$M_\Delta$ 为视界质量,$a$ 为视界面积,$Q$ 为电荷,$\Phi$ 为电势;另一方面,Smarr 公式给出

    $$
    M_\Delta = \frac{\kappa a_\Delta }{4\pi } + \Phi_\Delta Q_\Delta
    $$

    Rasheed 发现,对于非线性电动力学,上面公式不适用。但可以认为

    $$
    \boxed{
    M_\Delta = \frac{\kappa a_\Delta }{4\pi } + \Phi_\Delta Q_\Delta + V\left(a_\Delta , Q_\Delta , P_\Delta \right)
    }
    $$

    其中,$V$ 是由视界参数决定的未知势。

\end{frame}

\begin{frame}{Bardeen 黑洞 Smarr 公式}
    Bardeen模型是爱因斯坦场方程与一种特定非线性电动力学耦合的准确解,其拉氏量为

    $$
    \mathcal{L}(F)
    =\frac{2 }{2 sg^2 } \left(\frac{2g^2 F }{1+\sqrt{2g^2 F} }  \right)^{5/2}
    $$

    其中,$g$ 是磁荷,$F$ 是电磁场不变量,$s=g/m$;对于 SSS 时空,相应度规函数为

    $$
    \psi_B
    =1-\frac{2 m(r) }{r } 
    =1-\frac{2mr^2 }{\left(r^2+g^2 \right)^{3/2} }
    $$

    Bardeen 解不含电荷,视界质量只依赖于视界面积

    $$
    M_\Delta
    =\frac{1 }{8\pi } \int\kappa \mathrm{d}a
    =\int \left(1-m' \right)\mathrm{d}r
    $$

    \textcolor{red}{视界质量的正号性给出 $m(r)\leqslant r$,且 $\psi_B\geqslant 0$,这导致 $(r^2+g^2)^3\geqslant 4m^2 r^4$;$g^2=16m^2/27$ 对应极端黑洞;$g^2<16m^2/27$ 时存在内外事件视界。}
\end{frame}

\begin{frame}
    Bardeen 黑洞的势 $V$ 是不确定的,除非把一个积分常数设为零,于是
    
    $$
    V = m r^3 \frac{2g^2 - r^2 }{\left(g^2 + r^2 \right)^{3/2} }
    $$

    代入 Smarr 公式得到

    $$
    M_\Delta = \frac{r }{2 } - \frac{m r^3 }{\left(r^2+g^2 \right)^{3/2} } 
    $$

    注意到,Bardeen 黑洞的视界质量只依赖于视界面积,这是因为磁荷被认为是视界的不变参数。
    \begin{figure}
        \centering
        \includegraphics[width=0.5\textwidth]{fig/fig8.png}
        \caption{极端 Bardeen 黑洞视界质量 $M_\Delta$ 和表面引力 $\kappa$ 与视界半径 $r_\Delta$ 的关系}
      \end{figure}
\end{frame}

\subsection{孤立视界框架和质量关系}

\begin{frame}{孤立视界框架和质量关系}
    ADM模型中,带毛黑洞(hairy black hole)可以被视为普通黑洞和孤立子的束缚态。下面的公式将有色黑洞解的视界质量与相应理论的孤立子解的ADM质量联系起来

    $$
    M_{\mathrm{sol}}^{(n)} = M_{\mathrm{ADM}}^{(n)} - M_\Delta^{(n)} 
    $$

    若EBI理论给出两个确定解:黑洞解和孤子解,那么即使 EBI 黑洞是无色的(is not a coloured one),我们也应当采用 ACS 模型,认为 $b$ 是一个自由参数。因此这里 $n$ 应该替换为 BI 参数 $b$(分立或连续)。 

    EBI 解中,视界质量和 ADM 质量是视界半径 $r_\Delta$ 的函数,它们分别为

    $$
    M_\Delta^{(b)}(r_\Delta)
    =\frac{r_\Delta }{2 } + \frac{b^2 r_\Delta }{3 } \left(r_\Delta^2 - \sqrt{r_\Delta^4 + a^4} \right) - \frac{2g^2 }{3 } \int_0^{r_\Delta} \frac{\mathrm{d}s }{\sqrt{a^4+s^4} } 
    $$

    $$
    M_{\mathrm{ADM}}^{(b)}(r_\Delta)
    =\frac{r_\Delta }{2 } + \frac{b^2 r_\Delta }{3 } \left(r_\Delta^2 - \sqrt{r_\Delta^4 + a^4} \right) + \frac{2g^2 }{3 } \int_{r_\Delta}^{\infty} \frac{\mathrm{d}s }{\sqrt{a^4+s^4} }
    $$

    由于大多数 ACS 特征得到了满足,可以认为,当保持电荷不变而变化 BI 参数 $b$ 时,EBI 理论的静态部分可以用 Ashtekar 等人提出的有色黑洞(colored black hole)的启发式(heuristic)模型来描述。

\end{frame}

\begin{frame}
    \begin{figure}
        \centering
        \includegraphics[width=0.4\textwidth]{fig/fig9.png}
        \caption{ADM 质量作为视界半径 $r_\Delta$ 的函数}
    \end{figure}

    \begin{figure}
        \centering
        \includegraphics[width=0.4\textwidth]{fig/fig10.png}
        \caption{视界质量作为视界半径 $r_\Delta$ 的函数}
    \end{figure}
\end{frame}

\subsection{NLED 黑洞的稳定性}

\begin{frame}{稳定性条件}
    稳定性条件是一些对拉氏量及其微分 $L(F),L_F,L_{FF}$ 的约束。\textcolor{red}{变量替换 $y=\sqrt{2g^2 F}$} 后可表述为
    $$
    L(y)>0,\quad
    L(y)_{,y}>0,\quad
    L(y)_{,yy}>0 
    $$

    $$
    f(y)
    \equiv yL_{,yy} / L_{,y} > 0,\quad
    f(y) N(y) < 3    
    $$

    其中,$N(y)$ 为 SSS 线元度规函数。BI 拉氏量满足稳定性条件。把 BI 拉氏量写作 $y$ 的函数,取 $\tilde{G}=0$ 得

    $$
    L(y)
    =b^2\left(\sqrt{1+\frac{y^2 }{b^2 g^2 } } - 1 \right) > 0 
    $$

    其他不等式
    
    $$
    L_{,y} = \frac{y }{g^2 } \left(1 + \frac{y^2 }{b^2 g^2 }  \right)^{-1/2} > 0
    $$

    $$
    L_{,yy} = \frac{1 }{g^2 } \left(1 + \frac{y^2 }{b^2 g^2 }  \right)^{-3/2} > 0
    $$

    $$
    f(y) = y\frac{L_{,yy} }{L_{,y} } = \left(1 + \frac{y^2 }{b^2 g^2 }  \right)^{-1} > 0
    $$

\end{frame}

\begin{frame}
    对于所有 $y$,以上不等式均成立。考虑不等式 $f(y)N(y)<3$,由于 $0<f(y)\leqslant 1$,则其化简为 $\psi_{BI}(y)<3$

    对于\textcolor{red}{黑洞}情况,度规函数

    $$
    \psi_{BI}(y)
    =1-\frac{2m\sqrt{y} }{g } + \frac{2b^2 g^2 }{3y } \left(1-\sqrt{1+\frac{y^2 }{b^2 g^2 } } \right) + \frac{2\sqrt{gby} }{3 } \mathbb{F}\left[\arccos\left(\frac{gb-y }{gb+y } \right), \frac{1 }{\sqrt{2} }   \right] 
    $$

    \textcolor{red}{在 $0<y<y_\Delta=8.35$ 范围内,$0<\psi_{BI}(y)\leqslant 1$,不等式得以满足。}

    对于 EBI 方程\textcolor{red}{类粒子解},度规函数

    $$
    \psi_{BI}(y)
    =1-\frac{2m\sqrt{y} }{g } + \frac{2b^2 g^2 }{3y } \left(1-\sqrt{1+\frac{y^2 }{b^2 g^2 } } \right) - \frac{2\sqrt{gby} }{3 } \mathbb{F}\left[\arccos\left(\frac{y-gb }{gb+y } \right), \frac{1 }{\sqrt{2} } \right]
    $$

    $N(y=0)=1$,而 $N(y)$ 单调递减,\textcolor{red}{因此 $N=\psi<3$ 总能满足。}

    \textcolor{red}{当 $m=0$ 原点处 $N(r)$ 有限。因此,EBI 的黑洞解和类粒子解都是稳定的。}

\end{frame}

\section{Conclusion}
\begin{frame}{总结}
    本文探讨了其在平坦时空中、与引力耦合的黑洞解以及在f(T)引力理论中的应用。这些研究展示了NLED在解决经典电动力学和广义相对论中奇点问题的潜力,同时也揭示了其在宇宙学中的重要应用前景。

首先,Anjan Kar的研究提出了一种新的NLED模型,该模型在平坦时空中表现出有限的点电荷电场和自能,并在与引力耦合时能够产生正则黑洞或裸奇点解。这一模型在弱场极限下与Born-Infeld电动力学相似,但在强场极限下表现出独特的性质。研究表明,该模型在平坦时空中能够消除点电荷的无限自能问题,并在与引力耦合时产生无奇点的黑洞解。然而,该模型在磁性场的因果性和能量条件方面仍面临一些挑战。

其次,探讨了NLED在f(T)引力理论中的应用,特别是在宇宙加速膨胀和广义第二定律(GSLT)的有效性方面。研究表明,NLED在f(T)引力理论中能够产生与宇宙加速膨胀一致的解,并在某些情况下满足GSLT。通过构造具体的f(T)模型,研究发现NLED在宇宙早期阶段能够避免初始奇点,并在晚期阶段表现出标准的辐射相。

最后,回顾了NLED在黑洞物理中的应用,特别是Born-Infeld电动力学与引力耦合的解。研究表明,Born-Infeld理论能够产生无奇点的黑洞解,并在某些情况下表现出与Reissner-Nordström解类似的渐近行为。然而,这些解在能量条件和稳定性方面仍存在争议。此外,研究还探讨了NLED黑洞的热力学性质,发现其满足黑洞力学的第一定律,但Smarr公式需要进行修正。

综上所述,NLED在解决经典电动力学和广义相对论中的奇点问题方面表现出巨大潜力,但在引力耦合和宇宙学应用中仍面临诸多挑战。未来的研究需要进一步探索NLED在不同物理背景下的行为,并解决其在能量条件和稳定性方面的潜在问题。
\end{frame}
    
\begin{frame}[noframenumbering]
    \centering
    {\fontsize{40}{50}\selectfont Thank You!}
\end{frame}

\end{document}