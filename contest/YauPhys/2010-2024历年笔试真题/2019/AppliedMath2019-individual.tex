\documentclass[10pt]{article}
\usepackage[margin=1in]{geometry}
\usepackage{dsfont}
\usepackage{amsmath, amsthm, amssymb, mathrsfs}
\usepackage[abbrev,non-sorted-cites]{amsrefs}
\usepackage[all]{xy}
\usepackage{CJK}
\usepackage{grffile}
\usepackage{graphicx}
\ProvidesPackage{stackrel}
\newtheorem{thm}{Theorem}[section]
\newtheorem{cor}[thm]{Corollary}
\newtheorem{lem}[thm]{Lemma}
\newtheorem{ex}[thm]{Example}
\newtheorem{definition}[thm]{Definition}
\newtheorem{rmk}[thm]{Remark}
\newtheorem{obs}[thm]{Observation}
\newtheorem{prop}[thm]{Proposition}
\newtheorem{conj}[thm]{Conjecture}
\newtheorem{question}[thm]{Question}
\newtheorem{prob}[thm]{Problem}
\renewcommand{\thefootnote}{\arabic{footnote}}

\newcommand{\de}{\partial}
\newcommand{\db}{\overline{\partial}}
\newcommand{\ov}{\overline}
\newcommand{\ddbar}{\sqrt{-1}\de\db}
\newcommand{\vp}{\varphi}
\newcommand{\ti}{\tilde}
\newcommand{\ve}{\varepsilon}
\newcommand{\Ric}{\mathrm{Ric}}
\newcommand{\tr}[2]{\textrm{tr}_{#1} #2}
\renewcommand{\leq}{\leqslant}
\renewcommand{\geq}{\geqslant}

\newcommand{\thalf}{\textstyle{\frac{1}{2}}}


\begin{document}


\begin{center}
S.-T. Yau College Student Mathematics Contests 2019

\vspace{0.1cm}

\Large {\bf Applied and Computational Math}

\vspace{0.1cm}

\large {\bf Individual (4 problems)}

\vspace{0.1cm}
\end{center}

\begin{enumerate}

\item[1)](20 points)

    Given a set $\mathcal X$, $m\in\mathbb{N}$ and a hypothesis space $\mathcal{H}$, define
    \begin{equation*}
    \Pi_{\mathcal{H}}(m) = \max_{\{x_1,x_2,\ldots,x_m\}\subseteq\mathcal X} |\{(h(x_1),h(x_2),\ldots,h(x_m))|h\in\mathcal{H}\}|
    \end{equation*}
    where $|\mathcal S|$ denotes the cardinality of the set $\mathcal S$. The VC dimension of $\mathcal H$ is
    \begin{equation*}
    \mathrm{VC} (\mathcal{H}) = \max\{m:\Pi_{\mathcal{H}}(m)=2^m\}.
    \end{equation*}
     $(i)$ Let $\mathcal X=\mathbf R$. If $a\leq b$, define $h(x;a,b) = 1$ if $x\in[a,b]$ and $h(x)=-1$ if $x\notin[a,b]$. Find the VC dimension of the hypothesis space $\mathcal{H}=\{h(x;a,b)|a,b\in\mathbf{R},a\leq b\}$.\\
    $(ii)$ Let $\mathcal X = \mathbf R^d$, $\mathcal{H}$ to be the set of linear classifiers,
    i.e.\ $\mathcal{H} = \{f(x)|f(x)=\mathrm{sign}(w^\top x+b), w\in\mathbf{R}^d,b\in\mathbf{R}\}$ where $\mathrm{sign}(x) = 1$ if $x>0$,
    $\mathrm{sign}(x)=-1$ if $x<0$ and $\mathrm{sign}(x)=0$ if $x=0$. Show that the VC dimension of $\mathcal{H}$ is $d+1$.

\item[2)](25 points)


    Consider Richardson's difference scheme for the heat equation $u_{t} = u_{xx}$ :
    \begin{equation*}
    \frac{1}{2k}\left( u(x, t+k) - u(x, t-k) \right)
    = \frac{1}{h^{2}}\left( u(x-h, t) - 2 u(x, t) + u(x+h, t) \right).
    \end{equation*}
      $(i)$ Show that this scheme has second-order truncation error.\\
      $(ii)$ Use either ODE principles or von Neumann analysis to show that this scheme is unconditionally unstable.\\
      $(iii)$ Demonstrate a minor modification of the left-side of Richardson's scheme that yields a familiar unconditionally stable scheme and prove it.
\item[3)](25 points) \label{ques:differeoftheta}

Let $\emptyset\neq K$ be a closed convex set in $\mathbf{R}^n$, i.e., $K$ is a closed set and for any $x,y\in K$ and $\lambda\in (0,1)$, $\lambda x+(1-\lambda)y\in K$.
For any $z\in \mathbf{R}^n$, let $\Pi_{K}(z)$ denote the metric projection of $z$ onto $K$, which is the unique optimal solution of following problem:

\begin{equation}\label{eq:Projector}
\min \displaystyle \frac{1}{2}\|y-z\|_2^2,\quad{\rm s.t.}\quad  y\in K.
\end{equation}
Show that\\
$(i)$ the point $y\in K$ solves (\ref{eq:Projector}) if and only if
\begin{equation*}\label{eq:Projecterinequlity} (z-y)^T(d-y)   \leq 0, \quad \forall\, d\in K;
\end{equation*}
$(ii)$  for any $ y,z\in \mathbf{R}^n$,
$$
\|\Pi_{K}(y)-\Pi_{K}(z)\|_2\le \|y-z\|_2;
$$
$(iii)$ $\Theta(\cdot)$ is continuously differentiable with its gradient given by
\[\nabla \Theta(z)=z-\Pi_{K}(z),\]
where for any $z\in \mathbf{R}^n$,
$\Theta(z):=\thalf \|z-\Pi_{K}(z)\|_2^{2}$.

\item[4)](25 points) The scientists FitzHugh (1961) and Nagumo, Arimoto, Yoshizawa (1962) derived a mathematical model to characterize the behavior of a neuron under the externally injected current $I$:
\begin{equation*}
\begin{cases}
\frac{dV}{dt}&=V-\frac{1}{3}V^3-W+I,\\
\frac{dW}{dt}&=\frac{1}{\tau}\big(V+a-bW \big),
\end{cases}
\end{equation*}
where the variable $V$ describes the membrane potential of the neuron, the variable $W$ describes the current arising from opening and closing of ion channels on the neuron’s membrane. The variables $\tau$, $a$ and $b$ are parameters with typical values:  $a=0.7$,$b=0.8$ and $\tau=13$.

$(i)$ For a small positive constant current $I$, how the neuron
behaves.

$(ii)$ For a large positive constant current $I$, how the neuron
behaves.

$(iii)$ Suppose one injects a pulse current with different magnitude
at some time $t_0$, i.e., $I=I_0\delta(t-t_0)$, where $I_0$
describes the magnitude of the pulse, analyze the dynamical behavior
of the neuron when $I_0$ is small or large.






\end{enumerate}






\end{document}
