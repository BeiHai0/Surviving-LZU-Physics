\documentclass[10pt]{article}
\usepackage[margin=1in]{geometry}
\usepackage{dsfont}
\usepackage{amsmath, amsthm, amssymb, mathrsfs}
\usepackage[abbrev,non-sorted-cites]{amsrefs}
\usepackage[all]{xy}
\usepackage{CJK}
\usepackage{grffile}
\usepackage{graphicx}
\ProvidesPackage{stackrel}
\newtheorem{thm}{Theorem}[section]
\newtheorem{cor}[thm]{Corollary}
\newtheorem{lem}[thm]{Lemma}
\newtheorem{ex}[thm]{Example}
\newtheorem{definition}[thm]{Definition}
\newtheorem{rmk}[thm]{Remark}
\newtheorem{obs}[thm]{Observation}
\newtheorem{prop}[thm]{Proposition}
\newtheorem{conj}[thm]{Conjecture}
\newtheorem{question}[thm]{Question}
\newtheorem{prob}[thm]{Problem}
\renewcommand{\thefootnote}{\arabic{footnote}}

\newcommand{\de}{\partial}
\newcommand{\db}{\overline{\partial}}
\newcommand{\ov}{\overline}
\newcommand{\ddbar}{\sqrt{-1}\de\db}
\newcommand{\vp}{\varphi}
\newcommand{\ti}{\tilde}
\newcommand{\ve}{\varepsilon}
\newcommand{\Ric}{\mathrm{Ric}}
\newcommand{\tr}[2]{\textrm{tr}_{#1} #2}
\renewcommand{\leq}{\leqslant}
\renewcommand{\geq}{\geqslant}

\newcommand{\thalf}{\textstyle{\frac{1}{2}}}


\begin{document}



\begin{center}
S.-T. Yau College Student Mathematics Contests 2019\\

\vspace{0.1cm}

\Large {\bf Probability and Statistics}

\vspace{0.1cm}

\large {\bf Team (4 problems)}

\vspace{0.1cm}


\end{center}

\begin{enumerate}
\item[1)] Suppose $(X_n)_{n\ge 1}$ is a sequence of i.i.d. random variables and the common law is exponential with parameter one. Show that
\[\mathbb{P}\left[\limsup_{n\to\infty}\frac{X_n}{\log n}=1\right]=1.\]

\item[2)] Let $(X_n)_{n\ge 1}$ be i.i.d. real random variables and set $S_n=\sum_{i=1}^n X_i$ for $n\ge 1$. Suppose that for some constant $c\in\mathbb{R}$ we have
$S_n/n\to c$ as $n\to\infty$ almost surely.
Show that $X_1$ has a finite first moment and $\mathbb{E}[X_1]=c$.

\item[3)]Consider uniform permutation of $\{1, 2, \ldots, n\}$ and denote by $X_n$ the number of cycles in the permutation. Find a sequence of reals $(a_n)_{n\ge 1}$ such that
\[\lim_{n\to\infty}\frac{\mathbb{E}[X_n]}{a_n}=1,\]
and justify your answer.
\item[4)]The Erd\"{o}s-R\'{e}nyi random graph $G(n, p)$ with parameters $n\ge 1$ and $p\in [0,1]$ is the random graph whose vertex set is $V=\{1, 2, \ldots, n\}$ and where for each pair $i\neq j\in V$ the edge $i\leftrightarrow j$ is present with probability $p$ independently of all the other pairs.
\begin{enumerate}
\item[(a)] For $\epsilon>0$, if $p_n\ge (1+\epsilon)\frac{\log n}{n}$, then
\[\mathbb{P}[G(n, p_n) \text{ has an isolated vertex}]\to 0,\quad \text{as }n\to\infty.\]
\item[(b)] For $\epsilon>0$, if $p_n\le (1-\epsilon)\frac{\log n}{n}$, then
\[\mathbb{P}[G(n, p_n) \text{ has an isolated vertex}]\to 1,\quad \text{as }n\to\infty.\]
\end{enumerate}



\end{enumerate}





\end{document}
