\section*{第五章~~总结与展望}

\setcounter{section}{5} \setcounter{subsection}{0}
\setcounter{table}{0} \setcounter{figure}{0} \setcounter{equation}{0}
\addcontentsline{toc}{section}{第五章~~总结与展望}

\subsection{当前面临的主要挑战}

(1)实验尚无正面信号
目前所有直接、间接以及加速器探测手段均未发现确凿的暗物质信号。这一事实可能暗示:

暗物质粒子与普通物质的相互作用极其微弱;

暗物质质量远超或远低于现有实验的灵敏度范围;

暗物质并非粒子形式,现有模型设想可能存在根本偏差。

(2)理论模型高度自由
虽然WIMP模型曾因其“WIMP奇迹”而备受推崇,但随着实验排除区域不断扩大,其参数空间越来越受限;轴子虽然仍具吸引力,但其质量范围跨越十个数量级,探测难度极大;而诸如 MACHO 或修改引力理论的方案也存在严重的不完备性或观测矛盾。这一现状提示我们暗物质的理论空间可能远比过去设想的更为丰富。

(3)多重观测之间的张力
在某些观测中存在暗物质模型无法完全解释的现象。例如:

银河旋转曲线在小尺度星系中存在偏离 ΛCDM 模型的行为;

暗物质质量分布与弱引力透镜观测存在偏差;

宇宙早期结构形成速率可能高于 WIMP 模型预测。

\subsection{未来发展方向}

(1)高灵敏度实验的推进
新一代暗物质实验正朝着更高目标推进。例如:

DARWIN(欧洲):计划使用 40 吨液氙,探索接近“中微子底噪”极限;

Tonne-scale Axion Experiments:如 MADMAX 和 IAXO,将极大拓展轴子质量窗口;

升级版 LHC 或未来高能对撞机(如 FCC、CEPC)可能开启新的能标,发现潜在的新粒子。

(2)天文观测的深度融合
未来望远镜(如 Vera Rubin Observatory、Euclid、Nancy Grace Roman Space Telescope)将对大尺度结构、弱透镜、星系动力学提供高精度数据,这些信息对于反演暗物质的分布与性质至关重要。

此外,中微子望远镜、引力波探测器与21厘米宇宙学等新兴手段可能间接贡献暗物质的研究。

(3)跨尺度理论模型的发展
近年来涌现出一些综合粒子物理、宇宙学与引力修正的“混合模型”,例如:

多成分暗物质模型(multi-component DM);

自相互作用暗物质(SIDM);

暗光子或隐藏规范对称性;

热起源与非热起源的混合生成机制。

这些模型虽然更复杂,但更贴合观测现象,体现出现代物理研究从“简洁优美”向“复杂有效”演化的趋势。

\subsection{总结}

本文从三类主要观测证据出发,回顾了支持暗物质存在的星系旋转曲线、引力透镜效应与宇宙微波背景数据;随后介绍了主流暗物质模型,包括WIMP、轴子、MACHO与修改引力理论,并综述了当前探测实验的最新进展。尽管多种实验已极大压缩理论模型空间,暗物质的微观本质仍未揭晓。未来研究将在更高灵敏度实验、跨尺度模型建构与多波段天文观测中继续深化。暗物质问题不仅牵涉现代宇宙学的核心构架,也在挑战人类对“存在”的基本认知,其最终揭示可能开启物理学的新篇章。

\newpage