\pagenumbering{Roman} % 设置页码编号为大写罗马数字。LZU毕业论文要求中英文摘要页用罗马数字单独连续编号。
\pagestyle{plain}{%
\fancyhf{} % clear all header and footer fields
\fancyhead[C]{\CJKfamily{song}\wuhao }
\fancyfoot[C,C]{\wuhao--~\thepage~--}
\renewcommand{\headrulewidth}{0pt} % 不显示页眉横线
\renewcommand{\footrulewidth}{0pt}} % 不显示页脚横线


\fontsize{12pt}{20pt}\selectfont % 摘要内容小四号对应12pt;行距20pt

\setcounter{page}{1}
\begin{center} % 居中
	{\fontsize{16pt}{13pt}\selectfont\bf 暗物质存在的证据与可能的解释}
\end{center}
\begin{center}
{\fontsize{15.75pt}{13pt}\selectfont{\bf 摘~要}} \vspace{1.0cm}
\end{center}
\addcontentsline{toc}{section}{{摘~要}}
暗物质是现代宇宙学中最深刻的未解之谜之一。尽管它无法被电磁波直接探测,但天文学观测表明,它对星系旋转、引力透镜和宇宙微波背景辐射等现象产生了重要影响。本文首先介绍了支持暗物质存在的三类主要观测证据,包括星系旋转曲线的异常、引力透镜效应和宇宙微波背景辐射;随后,本文简要评述了当前主流的理论模型,包括WIMP、轴子、超对称模型以及额外维理论。尽管尚未直接探测到暗物质粒子,但多种实验正在进行中。通过综述观测数据与理论发展,本文展示了暗物质研究在现代物理中的核心地位与挑战。\\
\textbf{\hei 关键词}: 暗物质;  星系旋转曲线 ;  弱相互作用大质量粒子; 宇宙微波背景辐射;轴子

\newpage
\fontsize{12pt}{18pt}\selectfont 
\begin{center} {\fontsize{15.75pt}{13pt}\selectfont\bf
Evidence for Dark Matter and Its Possible Explanations}\end{center}
\begin{center}
{\large Abstract} \vspace{1.0cm}
\end{center}
\addcontentsline{toc}{section}{{\large Abstract}}
Dark matter is one of the most profound unsolved mysteries in modern cosmology. Although it cannot be directly detected via electromagnetic interactions, astronomical observations have shown that it plays a crucial role in phenomena such as galaxy rotation, gravitational lensing, and the cosmic microwave background (CMB). This paper first introduces three main types of observational evidence supporting the existence of dark matter, including the anomalous rotation curves of galaxies, gravitational lensing effects, and the CMB. It then briefly reviews current leading theoretical models, such as Weakly Interacting Massive Particles (WIMPs), axions, supersymmetric models, and extra-dimensional theories. Although dark matter particles have not yet been directly detected, numerous experiments are currently underway. By reviewing both observational data and theoretical developments, this paper highlights the central importance and ongoing challenges of dark matter research in modern physics. \\
\textbf{Key words}: Dark matter; Galaxy rotation curves; Weakly Interacting Massive Particles (WIMPs); Cosmic Microwave Background; Axions

\newpage
