\pagenumbering{Roman}
\pagestyle{plain}{%
\fancyhf{} % clear all header and footer fields
\fancyhead[C]{\CJKfamily{song}\wuhao }
\fancyfoot[C,C]{\wuhao--~\thepage~--}
\renewcommand{\headrulewidth}{0pt}
\renewcommand{\footrulewidth}{0pt}}


\fontsize{12pt}{18pt}\selectfont \setcounter{page}{1}
\begin{center} 
	{\fontsize{15.75pt}{13pt}\selectfont\bf 暗物质存在的证据与可能的解释}
\end{center}
\begin{center}
{\fontsize{15.75pt}{13pt}\selectfont{\bf 摘~要}} \vspace{1.0cm}
\end{center}
\addcontentsline{toc}{section}{{摘~要}}
暗物质是现代宇宙学中最深刻的未解之谜之一。尽管它无法被电磁波直接探测,但天文学观测表明,它对星系旋转、引力透镜和宇宙微波背景辐射等现象产生了重要影响。本文首先介绍了支持暗物质存在的三类主要观测证据,包括星系旋转曲线的异常、强引力透镜效应和宇宙大尺度结构的形成;随后,本文简要评述了当前主流的理论模型,包括WIMP、轴子、MACHO以及修改引力理论等。尽管尚未直接探测到暗物质粒子,但多种实验正在进行中。通过综述观测数据与理论发展,本文展示了暗物质研究在现代物理中的核心地位与挑战。\\
\textbf{\hei 关键词}: 暗物质;  星系旋转曲线; 引力透镜 ;  弱相互作用大质量粒子; 宇宙微波背景辐射

\newpage
\fontsize{12pt}{18pt}\selectfont 
\begin{center} {\fontsize{15.75pt}{13pt}\selectfont\bf
Evidence for Dark Matter and Its Possible Explanations}\end{center}
\begin{center}
{\large Abstract} \vspace{1.0cm}
\end{center}
\addcontentsline{toc}{section}{{\large Abstract}}
Dark matter is one of the most profound mysteries in modern cosmology. Although it cannot be directly detected via electromagnetic interactions, astronomical observations suggest that it plays a crucial role in various phenomena, such as galaxy rotation curves, gravitational lensing, and the cosmic microwave background. This paper first reviews three main types of observational evidence for dark matter: anomalous galactic rotation, strong lensing effects, and the formation of large-scale structures. Then, it briefly discusses several leading theoretical models, including WIMPs, axions, MACHOs, and modified gravity. While dark matter particles have not yet been directly detected, a number of experiments are underway. By surveying both observational and theoretical developments, this work highlights the central role and ongoing challenges of dark matter research in modern physics..\\
\textbf{Key words}: Dark matter; Galaxy rotation curves; Gravitational lensing; Weakly Interacting Massive Particles (WIMPs); Cosmic microwave background (CMB)

\newpage
