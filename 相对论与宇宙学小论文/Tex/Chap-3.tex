\section*{第三章~~暗物质的候选者}
\setcounter{section}{3} \setcounter{subsection}{0} 
\setcounter{table}{0} \setcounter{figure}{0} \setcounter{equation}{0}
\addcontentsline{toc}{section}{第三章~~暗物质的候选者}

要成为暗物质的候选粒子,粒子必须满足一些一般性的约束限制:

\begin{itemize}
    \item 它必须在宇宙尺度上稳定,这样它才能到今天仍然存在;
    \item 它不参与强或电磁相互作用;
    \item 所有的候选者加在一起必须有合适的残留丰度;
    \item 重子物质不能在暗物质组分中占有太大比例。
\end{itemize}

\subsection{WIMP:弱相互作用大质量粒子(Weakly Interacting Massive Particles)}

根据目前流行的宇宙学模型,早期宇宙处于高温状态,各种粒子通过快速交换能量处于热平衡态粒子数密度满足玻尔兹曼分布,宇宙的温度随着其膨胀而逐渐下降,直到今天的温度为3K 左右。在温度下降的过程中,与热力学平衡体系相互作用较弱的稳定粒子会在某个时期脱离热平衡,其粒子数密度不再随温度而快速下降,成为剩余丰度。这些剩余粒子最终形成今天宇宙中存在的各类基本元素。这一热退耦机制很好地解释了宇宙中的轻元素比如氦的丰度。暗物质作为一种弱相互作用的粒子,其丰度也有可能是来源于热力学退耦。研究表明,如果暗物质的相互作用强度和质量在电弱能标附近,可以自然解释宇宙中的暗物质丰度。这一类候选者通常被称为弱相互作用大质量粒子(WIMP)。\cite{周宇峰2011暗物质问题简介}

\subsubsection{WIMP“奇迹”}

在WIMP类暗物质丰度的计算中,由于存在各种数量级上差别巨大的物理量如 CMB温度、哈勃常数、普朗克常数等的偶然相消,导致了最后要求的暗物质质量在电弱质量标度附近,而很多新物理理论模型中预期的暗物质质量也正好在电弱质量标度附近。这种巧合被称为 WIMP “奇迹”(WIMP miracle)。\cite{刘佳2009暗物质的理论研究进展}

假设稳定的暗物质粒子为 $X$,它和标准模型粒子 $Y$ 通过 $X\bar{X} \leftrightarrow Y\bar{Y}$ 进行反应。这个过程可以由玻尔兹曼方程来描述

\begin{align}
\frac{\mathrm{d}n_X }{\mathrm{d}t } + 2 H n_X 
=-\braket{\sigma_{X\bar{X}}\left|v \right|} \left(n_X^2 - n_{X,\mathrm{eq}}^2 \right),
\end{align}

其中,$n_X$ 是暗物质粒子的数密度;$n_{X,\mathrm{eq}}$ 是处于热平衡的粒子数密度;$H\equiv \dot{R}/R=\left(8\pi G \rho/3 \right)^{1/2}$ 是宇宙的膨胀速度;$\braket{\sigma_{X\bar{X}}\left|v \right|}$ 是暗物质湮灭截面的热平均值。当宇宙早期的温度远大于 $m_X$ 时,$X\bar{X}$ 的产生和湮灭过程达到动态平衡,所以暗物质和标准模型粒子一样大量存在。随着宇宙的膨胀,宇宙温度逐渐降低。当温度降至 $m_X$ 以下时,标准模型粒子 $Y$ 不能有效碰撞产生暗物质粒子 $X$,这是因为假设 $Y$ 按能量分布是玻尔兹曼分布,当温度低于 $m_X$ 时,能量大于 $m_X$ 的粒子是指数下降的。而暗物质粒子的湮灭过程仍在继续,所以 $X$ 粒子数密度在迅速降低。一般来讲,$X$ 是非相对论粒子,此时 $X$ 粒子的平衡密度为

\begin{align}
n_{X,\mathrm{eq}}
=g_X \left(\frac{m_X T}{2\pi }  \right)^{3/2} \mathrm{e}^{-m_X/T},
\end{align}

其中 $g_X$ 是 $X$ 拥有的内部自由度的数目。当宇宙降温至某个温度 $T_f$ 以下时,暗物质湮灭过程发生的速度小于宇宙膨胀速度,即 $n \Braket{\sigma_{X\bar{X}}\left|v \right|} \leqslant H .$ 可认为暗物质再膨胀的几率很小,其数密度基本只受宇宙膨胀的影响,即达到热退耦(thermal freeze-out),相应的温度 $T_f$ 称为退耦温度。

通过求解玻尔兹曼方程,可以近似得到今天宇宙中暗物质残留密度

\begin{align}
\Omega_X h^2 \sim g_*^{-1/2} x_f \frac{1.17\times 10^{-10} }{a+3b/x_f }, 
\end{align}

其中 $a,b$ 来自暗物质湮灭截面的非相对论展开 $\Braket{\sigma_{\bar{X}X}\left|v \right|}=a+b\Braket{v^2}+\mathcal{O}\left(v^4 \right)$;与宇宙温度相关的变量 $x_f$ 的定义是 $x_f\equiv m_X/T$;$g_*$ 是一个外部自由度(在标准模型中,在 $T\sim 1~\mathrm{TeV}$ 时,$g^*\sim 120$;在 $T\sim 1~\mathrm{GeV}$ 时,$g^*\sim 65$)。

如果要求上面假设的过程能够解释今天观测到的暗物质残留密度,即 $\Omega_X h^2 \sim 0.11$,则可以对暗物质的湮灭截面有个大概的估计(只考虑湮灭截面里与速度无关的部分,并假设 $g_*\sim 100,x_f\sim 20$)

\begin{align}
\Braket{\sigma_{X\bar{X}}\left|v \right|}
\sim g_*^{-1/2} x_f \frac{1.17 \times 10^{-10} ~\mathrm{GeV}^{-2} }{\Omega_x h^2 } \sim 10^{-9}~\mathrm{GeV}^{-2}, 
\end{align}

而通常一种质量为 $m_X$,相互作用耦合常数为 $\alpha$ 的粒子的湮灭截面可以近似写为(假设 $\alpha\sim 0,01,m_X\sim 300~\mathrm{GeV}$)

\begin{align}
\sigma_{X\bar{X}} \sim \frac{\alpha^2 }{m_X^2 } \sim 10^{-9}~\mathrm{GeV}^{-2},
\end{align}

这就是说,如果暗物质质量在 GeV-TeV 范围,且电弱相互作用的量级 $\alpha\sim 0.01$,那么它的数密度就符合目前实验观测。这就是所谓的WIMP“奇迹”。

\subsection{轴子(Axion)}

自上世纪七十年代以来,物理学家在夸克模型假设下在用规范场理论来描写强相互作用方面作了巨大努力,建立了称为量子色动力学的理论。量子色动力学可以很好地解释一些现象,但也存在一些困难。其中一个困难就是破坏了宇称守恒和时间反演不变性,这与实验事实矛盾。为了摆脱这一困境,理论物理学家提出了两种可能的路径:一种是假设夸克的质量为零,另一种是假设有一种轻的、长寿命的、不带电的玻色子存在,这种玻色子就是轴子(Axion)。我们知道,夸克是是自然界所有物质的最基本的组成单元,其质量不可能为零。因此轴子的假设可能更为合理。\cite{张家铨1984寻找轴子的实验现状}

\subsubsection{轴子的基本性质}

根据理论上的推断,轴子可能的自旋值和宇称为 $0^-,1^+,2^-,3^+,\cdots$,它的质量为

\begin{align}
m_a
\approx 25 N \left(x+\frac{1 }{x }  \right)~\mathrm{keV},
\end{align}

其中 $N$ 是夸克二重态的数目,其至少为 $2$;$x$ 是理论中引入的一个自由参变量。当自由参变量 $x=1$ 时,$x+1/x$ 有最小值 $2$,因此轴子的质量 $m_a$ 至少是 $100~\mathrm{keV} .$

轴子的寿命与它的质量有密切的关系。当轴子的质量小于两倍电子质量时,它只能通过放射两个 $\gamma$ 光子的方式衰变

\begin{align}
a \to \gamma + \gamma,
\end{align}

它的寿命可以表示为

\begin{align}
\tau_{a\to 2\gamma}
\approx 0.4 Z^{-1} \left(\frac{100~\mathrm{keV} }{m_a }  \right)^5~\mathrm{s},
\end{align}

其中 $Z\equiv m_u/m_d \approx 0.56$,$m_u$ 是上夸克的质量,$m_d$ 是下夸克的质量。若 $m_a\approx 100~\mathrm{keV}$,则它的寿命大约是 $0.7~\mathrm{s} .$

若轴子的质量大于两边电子质量,那么除了放射两个 $\gamma$ 光子的衰变方式之外,衰变方式

\begin{align}
a \to e^+ + e^-,
\end{align}

也是允许的。这样轴子的寿命就会短很多。理论上给出轴子通过放射 $e^+ e^-$ 偶衰变方式的寿命为

\begin{align}
\tau_{a\to e^+ e^-}
=\frac{8 \pi x^2 f_\phi^2}{m_e^2\left(m_a^2-4m_e^2 \right)^{1/2} } ,
\end{align}

其中 $f_\phi\equiv \left(\sqrt{2} G_F \right)^{-1/2}$,$G_F$ 是费米耦合常数;$m_e$ 是电子质量。若取 $x\approx 1$,那么当轴子质量 $m_a$ 为几 MeV 时,轴子通过放射 $e^+ e^-$ 偶衰变方式的寿命 $\tau_{a\to e^+ e^-}$ 大约时 $10^{-8} - 10^{-9}~\mathrm{s} .$

\subsubsection{轴子作为暗物质粒子候选者}

轴子能够作为暗物质的原因主要包括以下几点\cite{duffy2009axions}:

1. 解决强 CP 问题的机制:轴子源自佩奇-奎恩(PQ)机制,作为一种伪-诺恩-戈尔登(pseudo-Nambu-Goldstone)玻色子出现,能够自然解释强相互作用中的 CP 破缺问题(强 CP 问题)。其存在由粒子物理理论预测,具有很强的理论基础。

2. 早期宇宙中的产生机制:在宇宙早期,通过“真空重整机制”、弦和壁的衰变等过程,轴子以冷暗物质的形式在宇宙中生成。这些冷轴子没有与其他粒子充分热平衡,因此表现出无温度的、冷暗物质的特性,符合暗物质的要求。

3. 极长的相干长度和低能量尺度:轴子具有极小的质量(在微电子伏特范围),对应非常低的动能和长的相干长度,使得它们在银河系尺度上表现为高度相干的场,这符合暗物质的分布特征。

4. 与引力和其他暗物质特性兼容:轴子作为无电荷、无磁矩、互动弱的粒子,不与普通物质发生强烈相互作用,符合暗物质不与电磁辐射直接相互作用的特性。

5. 密度和宇宙演化的一致性:轴子在宇宙中的粮食密度和结构形成中扮演角色,能够支持银河系中的暗物质候选粒子的角色,并且与宇宙学观察(如宇宙微波背景辐射、星系旋转曲线等)观察结果相一致。

因此,轴子凭借其产生机制、物理性质和宇宙分布的兼容性,被认为是极佳的暗物质候选粒子。

\subsection{超对称模型(Supersymmetry Model)}

\subsubsection{标准模型(Standard Model)}

目前,基本粒子之间的相互作用是由标准模型来描述的。这样的理论基于量子力学基本原理、狭义相对论、杨-米尔斯局域规范作用和最小作用量原理。

量子力学基本原理提供了量子化概念,狭义相对论和杨-米尔斯局域规范作用分别对应全局时空对称性和局域内稟对称性,最小作用量原理指导如何从满足以上对称性的拉式量得到运动方程。

上面提到的对称性分为全局的和局域的。全局变换对称性是指对场做常数变化(变化量不是时间空间的函数)情况下,作用量不变。局域变换对称性是指对场做局域变化(变化量可以是是时间空间的函数)情况下,作用量不变。

标准模型具有的对称性为 $P\otimes G$,其中 $P$ 表示庞加莱群,即四维时空的洛伦兹变换加上时空平移变换。在标准模型中,认为庞加莱对称性是全局的。如果要将其局域化,则涉及到广义相对论和引力。由于粒子需要满足庞加莱不变性,于是庞加莱群决定了可以存在的粒子的种类,即每种粒子可以按群的基本表示归类。

后面的 G 表示内稟对称性,在标准模型中,$G=\mathrm{SU(3)_c \otimes SU(2)_L \otimes U(1)_Y }$,并且是局域对称性。局域对称性 G描述了标准模型里的强、弱以及电磁相互作用。在标准模型中,根据以上对称性写出拉式量,并根据最小作用量原理就可以得到各种粒子的运动方程。

\subsubsection{超对称模型——超出标准模型的新物理模型}

超对称模型是近年来最为流行的一种超出标准模型的新物理模型。超对称模型与标准模型的唯一不同在于将前面的庞加莱群 $P$ 换为更大的时空对称性 $S$,超对称代数。超对称代数包含了庞加莱群,并且加入了玻色子和费米子的反对易变换操作。同样的,在这里 $S$ 是作为全局对称性处理的,如果考虑局域化的超对称代数 $S$,则是超引力(Supergravity)的研究范围.由于 $S$ 是一个更大的时空对称性,超对称包含的粒子数比标准模型要多,每一个粒子都有其超对称伴子(玻色子的超对称伴子是一个费米子,反之亦然)。

\subsubsection{超对称粒子(Lightest Supersymmetry Particle)作为暗物质候选者}

超对称的引入能够解决在 Higgs 粒子质量的圈
图修正计算中遇到的等级问题,并且可以使各种规范
作用常数在高能标处达到统一。

在早些时候,为了
解决超对称模型里质子衰变的问题引入了 $R$ 宇称,其定义为 $R\equiv (-1)^{3B+L+2S}$,其中 $B,L,S$ 分别为重子数、轻子数和自旋。所有的标准模型粒子
和超对称粒子分别具有偶宇称 $R=+1$ 和奇宇称 $R=-1$,因此所有的超对称粒子必须成对产生或
消灭,这样最轻的超对称粒子(lightest supersymmetry particle, LSP)就是稳定的,可以作为暗物质的候选者。

超对称粒子的质量谱是由超对称破缺的细节来决定的。但是,在超对称粒子列表里能够成为暗物质候
选者的粒子并不多。

在最小超对称模型里,具有电中性、色中性的粒子是四种 neutralino(中性规范玻色子以及 Higgs 粒子的超对称伴子),三种sneutrino(中微子的超对称伴子)以及 gravitino(引力子graviton的超对称伴子)。

在不加入右手中微子的情况下,sneutrino不能做为暗物质候选者,因为它大大超过了目前直接探测的限制,已经被目前直接探测实验排除。

Gravitino是合适的暗物质候选者它与其他粒子的相互作用极其微弱,不被目前已有的实验限制,但是因此对它的探测也相当困难。

Neutralino是超对称中研究最广泛的粒子。在超对称模型中,四种标准模型的超对称伴子(bino、wino和两个中性higgsino)混合后组成的四个质量本征态被称为neutralino,其中最轻的neutralino即是暗物质候选者。

\subsection{额外维(Extra Dimensions)}

额外维模型假设真实的时空并非通常的3+1
维,而是具有更高的维度。这些额外的维度很小,因此
不为我们所知。

最早的 ADD模型里就包括由 n 个小于毫米量级的平坦的额外维度。在 ADD模型里引力在全空间里传播,而标准模型粒子则限制在四维膜(四维空间在高维空间中的切片)上。这样的设置下,在长距离的情况下,引力行为和以前相同。但是,在距离和额外维的大小差不多时,引力的行为就会出现偏离,因此得出的普朗克能标比以前低。通过适当的参数选取,可以使普朗克能标降到 TeV 量级,这样就解决了前面提到的 Higgs粒子质量的圈图修正计算中遇到的等级问题。

在稍后提出的 Randall-Sundrum 模型里,依旧引入的是极小尺度的额外维,不过它们具有很大的空间曲率和 ADD 模型相同,RS模型中的引力在全空间里传播,而标准模型粒子则限制在四维膜上。在 RS 模型中,所有基本的物理参数都可以是四维普朗克能标量级大小,但是由于空间曲率引起的卷曲因子,这些物理参数在标准模型四维膜上都被压低到 TeV 量级。这同样解决了Higgs粒子质量的等级问题。

普朗克能标降到 TeV 量级的另一个重要意义是解释了为什么我们观测到的引力作用强度比其他三种力强、弱、电磁力要小很多。由于牛顿引力系数与普朗克质量平方成反比,当普朗克质量为 TeV 量级时,引力与弱电相互作用强度相当。这样使得在额外维模型里面,自然界四种力强、弱、电磁、引力的相互作用强度是差不多的。那么为什么我们观测到的引力比其他三种力要小呢? 因为时空存在更多维度,而引力存在于所有维度.我们作为四维时空的观测者只看到一部分的引力,所以强度变小了。用形象的话来说,引力“漏”到其他维度里面去了。

由于以上两个模型只有引力在全空间里传播,因此引入的新粒子主要是一些激发态的引力子 graviton.

普遍的额外维模型(universal extra dimensions,
UED)则假设平坦的额外维,但是所有的标准模型粒
子都在空间的所有维度中传播。在 UED 模型中,
额外维的大小由于实验限制以及理论原因通常取为 $R^{-1} \sim \mathrm{TeV}^{-1} .$ 标准模型粒子在额外维中传播可能带有动量,于是在四维膜上表现为新的重粒子,即 Kaluza-Klein(KK) 激发态。由于额外维上的周期边界条件,在额外维的动量是量子化的.所以,KK 激发态在树图层次上的质量为

\begin{align}
m_{X^{(n)}}^2
=\frac{n^2 }{R^2 } + m_{X^{(0)}}^2,
\end{align}

其中,$X^{(n)} $ 是标准模型粒子的第 $n$ 个 KK 激发态;$R^{-1}\sim \mathrm{TeV}^{-1} $ 是额外维的大小;$X^{(0)}$ 是标准模型粒子,也是 KK 态的基态。在这个模型中,粒子的 KK激发态一般在 TeV,这个能标与目前 PAMELA和 ATIC 实验感兴趣的暗物质质量范围很接近,也是LHC感兴趣的能量区间。

最轻的 KK粒子的稳定性通常由 KK 宇称守恒保证。通常一个额外维假设被紧致为一个圆环 $S$(即 $r=0$ 的点和 $r=R$ 的点是同一个点),两个额外维会紧致为两个一维圆环的直积,即圆环面。在这种紧致下,额外维的动量守恒使得 KK 数($n$) 是守恒的。但是,在实际模型中为了构建手征费米子,额外维取为 $S/Z_2$($x\leftrightarrow -x$ 的对径捏合)或者 $T^2/Z_2$ 迹形(orbifold)。这个设置破坏了 KK 数($n$)的守恒,但是仍然存在 KK宇称守恒,即 $n$ 的奇偶性守恒仍保留下来。在 KK 宇称下为奇的质量最轻的粒子(lightest KK particle),比如光子的第一 KK 激发态,是稳定的,可以作为暗物质的候选者。

\newpage