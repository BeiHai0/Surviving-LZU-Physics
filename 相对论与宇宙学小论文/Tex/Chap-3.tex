\section*{第三章~~暗物质的理论模型}
\setcounter{section}{3} \setcounter{subsection}{0} 
\setcounter{table}{0} \setcounter{figure}{0} \setcounter{equation}{0}
\addcontentsline{toc}{section}{第三章~~暗物质的理论模型}

\subsection{WIMP:弱相互作用大质量粒子(Weakly Interacting Massive Particles)}

WIMP 是最受广泛关注的暗物质候选粒子之一,假设其质量在 1 GeV 至 TeV 量级,且只通过弱核力和引力与普通物质相互作用。WIMP 模型的物理动机主要来自对标准模型的拓展,例如超对称理论(SUSY)中的中微子或Kaluza-Klein 粒子等。

WIMP 模型的最大优点是具有所谓的“WIMP 奇迹”:若WIMP在早期宇宙中以热平衡存在,其通过湮灭反应的散射截面约为

$$
\Braket{\sigma v} \sim 3\times 10^{-26}~\mathrm{cm^3/s}
$$

那么其热遗留密度正好与现今宇宙的暗物质密度相符。这种“巧合”被认为是一种强有力的动机。

WIMP 的探测方式主要包括:

直接探测:WIMP 与探测器中原子核发生弹性散射(如 XENONnT、LUX-ZEPLIN);

间接探测:WIMP 湮灭产生的 γ 射线、反质子、中微子等(如 Fermi-LAT、AMS-02);

对撞机实验:如 LHC 搜索能量缺失或新粒子迹象。

尽管目前尚无正面信号,WIMP 仍是最成熟的暗物质理论之一,众多实验正在不断提高灵敏度以检验其参数空间。

\subsection{轴子(Axion)}

轴子最初并非为了解释暗物质而提出,而是为解决强相互作用中的 强 CP 问题(为何量子色动力学不违反 CP 守恒)。Peccei–Quinn 理论引入一种自发对称性破缺,伴随而生的 Goldstone 粒子即为轴子。

若轴子的质量极小(典型预测为 $10^{-6} \sim 10^{-2}~ \mathrm{eV} $),且耦合极弱,则可在早期宇宙中通过“冷凝机制”生成,成为冷暗物质候选者。这种非热起源模型避免了 WIMP 所需的高能对撞过程。

轴子的探测策略包括:

腔体实验(如 ADMX):探测轴子转化为微波光子;

太阳轴子实验(如 CAST):检测来自太阳的轴子信号;

光轴转换实验:例如“光通过墙”实验。

与 WIMP 不同,轴子模型允许暗物质质量极低,但仍可提供足够的宇宙质量密度,是近年来研究热点之一。

\subsection{MACHO:大质量致密天体(Massive Astrophysical Compact Halo Objects)}

MACHO 模型认为暗物质可能由普通但不可见的天体组成,如褐矮星、白矮星、中子星或小质量黑洞。这些天体因质量较大、亮度极低而难以被直接观测。

上世纪 1990 年代,MACHO 与 EROS 等合作通过微引力透镜实验对银河系晕中的暗物质成分进行调查,结果表明:这类天体虽然存在,但只能解释银河系晕质量的一小部分(< 20%),不足以构成主要的暗物质成分。

此外,宇宙核合成与微波背景辐射对重子成分的精确限制也排除了 MACHO 模型作为主要解释的可能性。因此,MACHO 目前被认为只能贡献少量局部暗物质。

\subsection{修改引力理论(Modified Gravity Theories)}

另一类理论试图不引入暗物质,而是通过修改引力定律来解释观测现象。最典型的是 MOND(Modified Newtonian Dynamics),由 Milgrom 在1983年提出,其基本思想是:在低加速度极限(远离星系中心),引力定律从牛顿形式偏离。

MOND 成功解释了一些星系旋转曲线,但存在以下问题:

无法统一解释所有星系;

难以在星系团与宇宙大尺度结构上应用;

与广义相对论不兼容,难以纳入标准宇宙模型。

为克服这些缺点,Bekenstein 提出了TeVeS(Tensor-Vector-Scalar Gravity),作为 MOND 的相对论推广版本。但这些理论在引力透镜、CMB 与结构形成等问题上仍难与 ΛCDM 模型媲美。

尽管如此,修改引力理论仍是值得探索的替代路径,尤其在当前暗物质粒子未被直接探测到的背景下,具有一定理论吸引力。

\newpage