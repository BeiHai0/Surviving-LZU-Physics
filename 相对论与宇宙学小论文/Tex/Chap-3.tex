\section*{第三章~~暗物质的候选者}
\setcounter{section}{3} \setcounter{subsection}{0} 
\setcounter{table}{0} \setcounter{figure}{0} \setcounter{equation}{0}
\addcontentsline{toc}{section}{第三章~~暗物质的候选者}

要成为暗物质的候选粒子,粒子必须满足一些一般性的约束限制:

\begin{itemize}
    \item 它必须在宇宙尺度上稳定,这样它才能到今天仍然存在;
    \item 它不参与强或电磁相互作用;
    \item 所有的候选者加在一起必须有合适的残留丰度;
    \item 重子物质不能在暗物质组分中占有太大比例。
\end{itemize}

\subsection{WIMP:弱相互作用大质量粒子(Weakly Interacting Massive Particles)}

根据目前流行的宇宙学模型,早期宇宙处于高温状态,各种粒子通过快速交换能量处于热平衡态粒子数密度满足玻尔兹曼分布,宇宙的温度随着其膨胀而逐渐下降,直到今天的温度为3K 左右。在温度下降的过程中,与热力学平衡体系相互作用较弱的稳定粒子会在某个时期脱离热平衡,其粒子数密度不再随温度而快速下降,成为剩余丰度。这些剩余粒子最终形成今天宇宙中存在的各类基本元素。这一热退耦机制很好地解释了宇宙中的轻元素比如氦的丰度。暗物质作为一种弱相互作用的粒子,其丰度也有可能是来源于热力学退耦。研究表明,如果暗物质的相互作用强度和质量在电弱能标附近,可以自然解释宇宙中的暗物质丰度。这一类候选者通常被称为弱相互作用大质量粒子(WIMP)。\cite{周宇峰2011暗物质问题简介}

在WIMP类暗物质丰度的计算中,由于存在各种数量级上差别巨大的物理量如 CMB温度、哈勃常数、普朗克常数等的偶然相消,导致了最后要求的暗物质质量在电弱质量标度附近,而很多新物理理论模型中预期的暗物质质量也正好在电弱质量标度附近。这种巧合被称为 WIMP “奇迹”(WIMP miracle)。

\subsection{轴子(Axion)}

轴子最初并非为了解释暗物质而提出,而是为解决强相互作用中的 强 CP 问题(为何量子色动力学不违反 CP 守恒)。Peccei–Quinn 理论引入一种自发对称性破缺,伴随而生的 Goldstone 粒子即为轴子。

若轴子的质量极小(典型预测为 $10^{-6} \sim 10^{-2}~ \mathrm{eV} $),且耦合极弱,则可在早期宇宙中通过“冷凝机制”生成,成为冷暗物质候选者。这种非热起源模型避免了 WIMP 所需的高能对撞过程。

轴子的探测策略包括:

腔体实验(如 ADMX):探测轴子转化为微波光子;

太阳轴子实验(如 CAST):检测来自太阳的轴子信号;

光轴转换实验:例如“光通过墙”实验。

与 WIMP 不同,轴子模型允许暗物质质量极低,但仍可提供足够的宇宙质量密度,是近年来研究热点之一。

\subsection{MACHO:大质量致密天体(Massive Astrophysical Compact Halo Objects)}

MACHO 模型认为暗物质可能由普通但不可见的天体组成,如褐矮星、白矮星、中子星或小质量黑洞。这些天体因质量较大、亮度极低而难以被直接观测。

上世纪 1990 年代,MACHO 与 EROS 等合作通过微引力透镜实验对银河系晕中的暗物质成分进行调查,结果表明:这类天体虽然存在,但只能解释银河系晕质量的一小部分(< 20%),不足以构成主要的暗物质成分。

此外,宇宙核合成与微波背景辐射对重子成分的精确限制也排除了 MACHO 模型作为主要解释的可能性。因此,MACHO 目前被认为只能贡献少量局部暗物质。

\subsection{修改引力理论(Modified Gravity Theories)}

另一类理论试图不引入暗物质,而是通过修改引力定律来解释观测现象。最典型的是 MOND(Modified Newtonian Dynamics),由 Milgrom 在1983年提出,其基本思想是:在低加速度极限(远离星系中心),引力定律从牛顿形式偏离。

MOND 成功解释了一些星系旋转曲线,但存在以下问题:

无法统一解释所有星系;

难以在星系团与宇宙大尺度结构上应用;

与广义相对论不兼容,难以纳入标准宇宙模型。

为克服这些缺点,Bekenstein 提出了TeVeS(Tensor-Vector-Scalar Gravity),作为 MOND 的相对论推广版本。但这些理论在引力透镜、CMB 与结构形成等问题上仍难与 ΛCDM 模型媲美。

尽管如此,修改引力理论仍是值得探索的替代路径,尤其在当前暗物质粒子未被直接探测到的背景下,具有一定理论吸引力。

\newpage