\section*{第二章~~暗物质存在的观测证据}
\setcounter{section}{2} \setcounter{subsection}{0}
\setcounter{table}{0} \setcounter{figure}{0} \setcounter{equation}{0}
\addcontentsline{toc}{section}{第二章~~暗物质存在的观测证据}

尽管暗物质无法通过电磁波直接观测,但多种独立的天文观测一致地表明,在宇宙的各个尺度上都存在着某种“不可见”的质量成分。以下三类观测证据构成了暗物质存在的最直接、最重要的实证基础。

\subsection{星系旋转曲线(Galaxy Rotation Curves)}

在螺旋星系中,恒星围绕星系中心旋转,其角速度应主要由中心区域的质量决定。根据牛顿引力定律,若质量主要集中于星系中心,则恒星的轨道速度 $v(r) $ 应随距离 $r$ 的增大而下降,满足

$$
v(r)
=\sqrt{\frac{G M(r) }{r } }
\propto r^{-1/2},
$$

其中 $M(r) $ 为半径 $r$ 内的质量总和。

然而,自1970年代起,Vera Rubin 和 Kent Ford 等人对多个螺旋星系的旋转曲线进行观测,发现实际数据与经典预言严重不符:在远离星系中心的区域,恒星的轨道速度并未下降,而是趋于平坦。这意味着在星系外围区域仍存在大量未被观测到的质量,其引力足以维持恒星的高速旋转。

这一“平坦旋转曲线”现象是暗物质存在的首个强烈证据。为了解释该现象,天文学家提出了“暗晕”模型,认为星系被包裹在一个大质量、不可见的暗物质晕中,质量分布随半径增长而增加,从而维持旋转速度恒定。后续大量星系的观测(包括低表面亮度星系和矮星系)也支持了这一结论,暗物质似乎普遍存在于星系结构中。

\subsection{引力透镜效应(Gravitational Lensing)}

根据广义相对论,质量分布可以导致时空弯曲,进而使背景天体发出的光线发生偏折。这一现象称为引力透镜效应,可分为强透镜和弱透镜两类。在天文学中,引力透镜被用作测量系统总质量的重要手段,尤其是在星系团尺度。

通过对多个星系团的引力透镜效应进行分析,科学家发现引力作用远远超过可见物质所能提供的水平。例如,在著名的子弹星系团(Bullet Cluster)中,两个星系团发生高速碰撞:X射线观测显示热气体集中在中心,而引力透镜揭示的质量分布却偏移至两个星系团的外围部分。换言之,引力“重心”不在可见质量所在的位置,而更接近星系的恒星成分。

这一现象被广泛认为是“可见物质无法解释全部引力作用”的直接证据。与修改引力理论相比,引入“不可见但有质量”的暗物质成分能够更自然地解释质量分布与光偏折之间的关系。此外,弱引力透镜在大范围宇宙尺度上也支持类似结论:观测得到的总质量普遍高于可见质量之和。

\subsection{宇宙微波背景与大尺度结构(CMB and Large-Scale Structure)}

宇宙微波背景辐射(Cosmic Microwave Background, CMB)是大爆炸残留下的热辐射,携带着早期宇宙密度扰动的信息。通过对CMB各个多极谱峰的位置与高度进行精密测量(如 WMAP、Planck 计划),研究者可以推断宇宙的基本成分比例。

CMB各阶峰结构强烈依赖暗物质的存在。例如,第二声波峰的相对高度要求在早期宇宙中必须存在不与光子耦合的质量成分,以推动重子涨落的“回弹”。无暗物质模型无法同时再现第一与第二声波峰的比值。此外,CMB测得的物质总密度约为 $\Omega_m\approx 0.3 $, 其中仅有 $\sim 0.05 $ 属于重子,剩余部分必须由非重子成分(即冷暗物质)构成。

进一步地,宇宙中的大尺度结构(如星系团、丝状结构)形成所需的“种子扰动”必须在重子退耦前就已存在。由于重子在早期宇宙中被光子强烈耦合,其扰动增长受到抑制,因此唯有冷暗物质先于重子形成涨落,才能解释今日宇宙中复杂的结构格局。

\newpage