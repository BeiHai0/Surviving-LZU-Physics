\section*{第四章~~暗物质探测实验现状}

\setcounter{section}{4} \setcounter{subsection}{0}
\setcounter{table}{0} \setcounter{figure}{0} \setcounter{equation}{0}

\addcontentsline{toc}{section}{第四章~~暗物质探测实验现状}

\subsection{直接探测实验(Direct Detection)}

直接探测的基本思想是:若暗物质粒子(如 WIMP)穿过地球时与原子核发生弹性散射,可在地下探测器中留下微弱信号。由于背景噪声(如宇宙射线、天然放射性)极高,因此这些实验通常建在深地下实验室中,以屏蔽干扰。

目前最前沿的直接探测实验包括:

XENONnT(意大利 Gran Sasso 实验室):使用7吨超纯液氙,探测能量低至 keV 级的核反冲事件;

LUX-ZEPLIN (LZ)(美国 Sanford 地下实验室):采用10吨液氙,具有极高的本底抑制能力;

PandaX-4T(中国锦屏地下实验室):亚洲最大的液氙暗物质实验,灵敏度逐年上升。

这些实验目前尚未探测到明确的 WIMP 信号,但已对暗物质质量–截面参数空间施加了极强限制。例如,LZ 实验对 $m_\chi\sim 40~\mathrm{GeV} $ 的 WIMP 排除了截面大于 $6\times 10^{-48}~\mathrm{cm}^2$ 的情形。

新一代实验(如 DARWIN)计划在未来十年内进一步扩大探测质量和背景抑制能力,若 WIMP 存在于当前理论预测区间内,有望被发现或排除。

\subsection{间接探测实验(Indirect Detection)}

间接探测的目标是寻找暗物质湮灭或衰变产生的产物,如高能γ射线、反质子、正电子或中微子。这类信号通常来源于暗物质密度较高区域,如银河中心、矮星系或星系团。

主要实验与观测包括:

Fermi-LAT:空间 $\gamma$ 射线望远镜,观测到来自矮星系的 $\gamma$ 射线谱;

AMS-02:国际空间站上的宇宙射线谱探测器,曾发现高能正电子过量,可能是暗物质湮灭或天体源;

H.E.S.S.、CTA(Cherenkov Telescope Array):地面切伦科夫望远镜,观测高能光子;

IceCube:南极中微子望远镜,搜索来自太阳或地心的中微子湮灭信号。

尽管个别观测结果(如 Fermi-LAT 的银河中心γ射线过量)引发了对暗物质解释的兴趣,但尚未获得无歧义的信号,多数结果可由天体源(如脉冲星)解释。因此,间接探测仍处于待定阶段。

\subsection{对撞机实验(Collider Searches)}

若暗物质粒子可以在高能对撞中被生成,则其存在可能在大型强子对撞机(LHC)中被揭示。由于暗物质粒子不会与探测器直接相互作用,故信号通常表现为“动量缺失事件”。

典型信号模式包括:

喷注 + Missing Energy;

重子、轻子 + Missing Energy;

假设模型下的新共振粒子的衰变路径。

ATLAS 和 CMS 实验对暗物质粒子的产生截面施加了限制,并对多种 WIMP 相关模型设定了排除范围。例如,假设存在新 Z′ 玻色子或超对称轻子,已有部分参数空间被排除。

尽管未见直接信号,但对撞机实验在限制模型参数、探索暗物质伴随粒子方面提供了独特信息,与其他两类方法互为补充。

\newpage