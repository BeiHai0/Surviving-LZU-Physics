\chapter{数学物理方程与分离变量法}

\section{波动方程、输运方程、泊松方程的标准形式}

\subsection{波动方程的标准形式}

\begin{equation}
u_{tt}-a^2\nabla^2 u(\vec{r},t)
=f(\vec{r},t).
\end{equation}

\subsection{输运方程(抛物方程)的标准形式}

\begin{equation}
u_{t}-a^2\nabla^2 u(\vec{r},t)
=f(\vec{r},t).
\end{equation}

\subsection{泊松方程(椭圆方程)的标准形式}

\begin{equation}
\nabla^2 u(\vec{r})
=f(\vec{r}).
\end{equation}

\subsection{拉普拉斯方程的标准形式}

\begin{equation}
\nabla^2 u(\vec{r})
=0.
\end{equation}

\section{定解条件}

定解条件是一个方程有确定解所要满足的条件。

定解条件包括初始条件和边界条件。

\subsection{初始条件}

初始条件指场量 $u(\vec{r},t) $ 和其关于时间的一阶导数 $u_t(\vec{r},t) $ 在初始时刻,也即 $t=0 $ 时所要满足的条件。

\subsubsection{波动方程的初始条件}

由于波动方程 $u_{tt}-a^2\nabla^2 u(\vec{r},t)=f(\vec{r},t) $ 中有场量对时间的二阶导 $u_{tt} $,因此要想让波动方程有确定解,既需要场量 $u(\vec{r},t) $ 在初始时刻的空间分布,也需要场量对时间的一阶导在初始时刻的空间分布。

因此,波动方程的初始条件为:

\begin{equation}
\left\{
\begin{aligned}
\left. u(\vec{r},t)\right|_{t=0}=\varphi(\vec{r}) \\
\left. u_t(\vec{r},t)\right|_{t=0}=\nu(\vec{r})
\end{aligned}
\right..
\end{equation}

\subsubsection{输运方程的初始条件}

由于输运方程 $u_{t}-a^2\nabla^2 u(\vec{r},t)=f(\vec{r},t) $ 中有场量对时间的一阶导 $u_{tt} $,因此要想让输运方程有确定解,只需要场量在初始时刻的空间分布,\textbf{或者}场量对时间的一阶导在初始时刻的空间分布。

因此,输运方程的初始条件为:

\begin{equation}
\left. u(\vec{r},t) \right|_{t=0} = \varphi(\vec{r}),\quad \text{or} \quad
\left. u_t(\vec{r},t) \right|_{t=0} = \nu(\vec{r}).  
\end{equation}

\subsubsection{泊松方程的初始条件}

由于泊松方程 $\nabla^2 u(\vec{r})=f(\vec{r}) $ 不含时间,因此泊松方程不需要初始条件。

\subsection{边界条件}

边界条件指场量在边界上所要满足的条件。

\subsubsection{第一类边界条件}

第一类边界条件是场量 $u(\vec{r},t)$ 在边界 $\partial \Omega$ 处的取值所要满足的条件:

\begin{equation}
\left. u(\vec{r},t) \right|_{\vec{r}\in \partial\Omega} 
=g(\vec{r},t).
\end{equation}

若 $g(\vec{r},t)=0$,则得到第一类齐次边界条件:

\begin{equation}
\left. u(\vec{r},t) \right|_{\vec{r}\in \partial\Omega} 
=0.
\end{equation}

\subsubsection{第二类边界条件}

第二类边界条件是边界上场量沿边界的外法线的方向导数所要满足的关系:

\begin{equation}
\left. \frac{\partial u(\vec{r},t)}{\partial n} \right|_{\vec{r}\in \partial \Omega}
=g(\vec{r},t).
\end{equation}

若 $g(\vec{r},t)=0$,则得到第二类齐次边界条件:

\begin{equation}
\left. \frac{\partial u(\vec{r},t)}{\partial n} \right|_{\vec{r}\in \partial \Omega}
=0.
\end{equation}

\subsubsection{第三类边界条件}

把第一类边界条件和第二类边界条件进行线性组合就得到第三类边界条件:

\begin{equation}
\left. \left(\alpha u(\vec{r},t)+\beta\frac{\partial u(\vec{r},t)}{\partial n} \right) \right|_{\vec{r}\in \partial \Omega} 
=g(\vec{r},t).
\end{equation}

若 $g(\vec{r},t)=0 $,则得到第三类齐次边界条件:

\begin{equation}
\left. \left(\alpha u(\vec{r},t)+\beta\frac{\partial u(\vec{r},t)}{\partial n} \right) \right|_{\vec{r}\in \partial \Omega} 
=0.
\end{equation}

\subsubsection{自然边界条件}

所要求解的场量 $u$ 在考虑的区域 $\Omega$ 及其边界 $\partial \Omega$ 上,都是有界的,不发散的,即:

\begin{equation}
|u|<+\infty.
\end{equation}

\subsubsection{周期性边界条件}

场量 $u(\vec{r},t) $ 具有空间周期性。

\subsubsection{衔接条件}

若研究的区域 $\Omega $ 可分成几个性质不同的子区域,则在相邻子区域的边界上要求用特殊的衔接条件。

\subsection{定解条件}

\subsubsection{波动方程定解条件}

\begin{equation}
\left\{
\begin{aligned}
&u_{tt}(\vec{r},t)-a^2\nabla^2 u = f(\vec{r},t) \\
&\left. u(\vec{r},t) \right|_{t=0} = \varphi(\vec{r}) \\
&\left. u_t(\vec{r},t) \right|_{t=0} = \nu(\vec{r}) \\
&\left[\alpha u(\vec{r},t)+\beta\frac{\partial u(\vec{r},t) }{\partial n } \right]_{\vec{r}\in\partial \Omega} = \left. g(\vec{r},t) \right|_{\vec{r}\in \partial \Omega}
\end{aligned}
\right.
\end{equation}

\subsubsection{输运方程定解条件}

\begin{equation}
\left\{
\begin{aligned}
&u_{t}(\vec{r},t)-a^2\nabla^2 u = f(\vec{r},t) \\
&\left. u(\vec{r},t) \right|_{t=0} = \varphi(\vec{r}), \quad \mathrm{or} \quad \left. u_t(\vec{r},t) \right|_{t=0} = \nu(\vec{r}) \\
&\left[\alpha u(\vec{r},t)+\beta\frac{\partial u(\vec{r},t) }{\partial n } \right]_{\vec{r}\in\partial \Omega} = \left. g(\vec{r},t) \right|_{\vec{r}\in \partial \Omega}
\end{aligned}
\right.
\end{equation}

\subsubsection{泊松方程定解条件}

\begin{equation}
\left\{
\begin{aligned}
&\nabla^2 u(\vec{r}) = f(\vec{r}) \\
&\left[\alpha u(\vec{r})+\beta\frac{\partial u(\vec{r}) }{\partial n } \right]_{\vec{r}\in\partial \Omega} = \left. g(\vec{r}) \right|_{\vec{r}\in \partial \Omega}
\end{aligned}
\right.
\end{equation}

\section{分离变量法}

\subsection{Sturm-Liouville本征值问题}

\subsubsection{Sturm-Liouville方程}

如下带有参数 $\lambda $ 的二阶常微分方程称为Sturm-Liouville方程(S-L)方程:

\begin{equation}
\frac{\mathrm{d} }{\mathrm{d}x }\left[k(x) y'(x) \right] - q(x) y(x) + \lambda \rho(x) y(x)
=0. 
\end{equation}

其中 $\lambda $ 称为本征参数(或本征值),$\rho(x) $ 称为权重函数。

若定义线性算子

\begin{equation}
L
\equiv -\frac{\mathrm{d} }{\mathrm{d}x } \left[k(x) \frac{\mathrm{d} }{\mathrm{d}x } \right] + q(x),
\end{equation}

则 S-L 方程可写为

\begin{equation}
L y(x) = \lambda \rho(x) y(x).
\end{equation}

\subsubsection{S-L方程本征函数的一般性质}

可以证明,在三类齐次边界条件或自然边界条件或周期性边界条件下,且约定 $k(x) \geqslant 0,q(x)\geqslant 0,\rho(x)>0 $ 的情况下,S-L方程存在无穷多组本征解,即存在一系列本征值 $\lambda_n $ 和本征函数 $y_n(x) $ 满足S-L方程:

\begin{equation}
\frac{\mathrm{d} }{\mathrm{d}x }\left[k(x) y'_n(x) \right] - q(x) y_n(x) + \lambda_n \rho(x) y_n(x)
=0,\quad,x\in\left[a,b \right],\quad n=1,2,\cdots,\infty
\end{equation}

这些本征解有一些一般的性质。

\subsubsection{S-L方程本征解的带权正交性}

设 $\lambda_n\ne \lambda_m $,二者分别对应本征函数 $y_n(x),y_m(x) $,即:

\begin{equation}
\left\{
\begin{aligned}
&\frac{\mathrm{d} }{\mathrm{d}x }\left[k(x) y'_n(x) \right] - q(x) y_n(x) + \lambda_n \rho(x) y_n(x)
=0, \\
&\frac{\mathrm{d} }{\mathrm{d}x }\left[k(x) y'_m(x) \right] - q(x) y_m(x) + \lambda_m \rho(x) y_m(x)
=0, \\
&y_n \ne y_m.
\end{aligned}
\right.
\end{equation}

可以证明,$y_n(x),y_m(x) $ 以权重 $\rho(x) $ 正交,即:

\begin{equation}
\int_a^b \rho(x) y_n(x) y_m^*(x) \mathrm{d}x
=0. 
\end{equation}

\subsubsection{S-L方程本征值的性质}

可以证明,S-L问题有无穷多非负本征值,所有本征值组成一个单调递增以无穷远点为凝聚点的序列。

\subsubsection{S-L方程函数的完备性与广义Fourier展开}

可以证明,所有的本征函数 $\left\{y_n(x) \right\} $ 构成一个完备的带权正交函数系,任何定义在 $x\in\left[a,b \right] $ 上的满足Direchlet条件的函数 $f(x) $ 可以其上进行广义Fourier展开:

\begin{equation}
f(x)
=\sum_{n=1}^{\infty} C_n y_n(x),
\end{equation}

为了求出系数 $C_n $,上式两边同乘 $\rho(x) y_m^*(x) $ 并积分,利用本征函数带权正交性就有:

\begin{equation}
\begin{split}
\int_a^b f(x) \rho(x) y_m^*(x) \mathrm{d}x
&=\sum_{n=1}^{\infty} C_n \int_a^b \rho(x) y_n(x) y_m^*(x) \mathrm{d}x \\
&=C_m \int_a^b \rho(x) y_m(x) y_m^*(x) \mathrm{d}x \\
&=C_m \int_a^b \rho(x) \left|y_m(x) \right|^2 \mathrm{d}x
\end{split}
\end{equation}

于是有:

\begin{equation}
C_m
=\dfrac{1 }{\int_a^b \rho(x) \left|y_m(x) \right|^2 \mathrm{d}x } \int_a^b f(x) \rho(x) y_m^*(x) \mathrm{d}x.
\end{equation}

\subsection{例题}

\subsubsection{例1}

\begin{example}
    
求解四边固定,$x,y $ 方向上边长分别为 $l,d $ 的矩形薄膜的本征振动(即求本征振动频率和本征振动函数)。

\end{example}

\begin{solution}
    
矩形薄膜的振动满足二维波动方程。这里采用直角坐标,结合“四周固定”这一边界条件,可得定解问题:

\begin{equation}
\left\{
\begin{aligned}
&\frac{\partial^2 u(x,y,t) }{\partial t^2 } - a^2\left[\frac{\partial^2 u(x,y,t) }{\partial x^2 } + \frac{\partial^2 u(x,y,t) }{\partial y^2 }  \right] = 0, \\
&\left. u \right|_{x=0} = \left. u \right|_{x=l} = 0, \\
&\left. u \right|_{y=0} = \left. u \right|_{y=d} = 0. \\
\end{aligned}
\right.
\end{equation}

设 $u(x,y,t) $ 可分离变量:

\begin{equation}
u(x,y,t) = U(x,y)T(t) = X(x)Y(y)T(t),
\end{equation}

代入波动方程可得:

\begin{equation}
X(x)Y(y)T''(t) - a^2\left[Y(y)T(t)X''(x) + X(x)T(t)Y''(y) \right]
=0.
\end{equation}

上式两边同时除以 $X(x)Y(y)T(t) $ 得:

\begin{equation}
\frac{T''(t) }{T(t) } - a^2\left[\frac{X''(x) }{X(x) } + \frac{Y''(y) }{Y(y) }  \right]
=0. \label{eq:rect-sep}
\end{equation}

观察可知,必定有:

\begin{equation}
\frac{T''(t) }{T(t) } = -\omega^2,\quad
\frac{X''(x) }{X(x) } = -k_x^2,\quad
\frac{Y''(y) }{Y(y) } = -k_y^2.
\end{equation}

将上式代入\eqref{eq:rect-sep},可知 $\omega,k_x,k_y $ 满足:

\begin{equation}
\omega^2 = a^2\left(k_x^2+k_y^2 \right).
\end{equation}

下面解 $X(x),Y(y),T(t) $ 所要满足的方程。

\begin{equation}
\frac{X''(x) }{X(x) } = -k_x^2
\Longrightarrow X(x) = A\cos(k_x x) + B\sin(k_x x).
\end{equation}

结合边界条件 $\displaystyle{u\big|_{x=0} = u\big|_{x=l} = 0 }$  可得:

\begin{equation}
A = 0,\quad
k_x^{(n)} = \frac{n\pi }{l },\quad n=1,2,\cdots.
\end{equation}

因此,$X(x) $ 的本征函数为:

\begin{equation}
X^{(n)}
=B^{(n)} \sin\left(\frac{n\pi }{l } x \right).
\end{equation}

\begin{equation}
\frac{Y''(y) }{Y(y) } = -k_y^2
\Longrightarrow Y(y) = C\cos(k_y y)+D\sin(k_y y).
\end{equation}

结合边界条件 $\displaystyle{u\big|_{y=0} = u\big|_{y=d} = 0 }$  可得:

\begin{equation}
C = 0,\quad
k_y^{(m)} = \frac{m\pi }{d },\quad m=1,2,\cdots.
\end{equation}

因此,$Y(y) $ 的本征函数为:

\begin{equation}
Y^{(m)}
=D^{(m)} \sin\left(\frac{m\pi }{d } y \right).
\end{equation}

由 $U(x,y)=X(x)Y(y) $ 可知,本征振动函数为:

\begin{equation}
\begin{split}
U^{(nm)}(x,y)
&=X^{(n)}(x)Y^{(m)}(y) \\
&=B^{(n)}D^{(m)} \sin\left(\frac{n\pi }{l } x \right) \sin\left(\frac{m\pi }{d } y \right) \\
&\equiv E^{(nm)} \sin\left(\frac{n\pi }{l } x \right) \sin\left(\frac{m\pi }{d } y \right),\quad E^{(nm)}\equiv B^{(n)}D^{(m)},\quad n,m=1,2,\cdots.
\end{split}
\end{equation}

由 $\omega^2=a^2\left(k_x^2+k_y^2 \right) $ 可知,本征振动频率为:

\begin{equation}
\begin{split}
\omega^{(nm)}
&=a\sqrt{\left(k_x^{(n)} \right)^2 + \left(k_y^{(m)} \right)^2} \\
&=a\sqrt{\left(\frac{n\pi }{l }  \right)^2 + \left(\frac{m\pi }{d }  \right)^2},\quad n,m=1,2,\cdots
\end{split}
\end{equation}

\end{solution}

\subsubsection{例2}

\begin{example}

求定解问题:
    
\begin{equation}
\left\{
\begin{aligned}
&u_{tt}-a^2u_{xx}=0 \\
&\left. u_{x} \right|_{x=0}=0 \\
&\left. u_{x} \right|_{x=l}=0 \\
&\left. u \right|_{t=0}=\cos\left(\frac{\pi x}{l}\right)+0.3\cos\left(\frac{3\pi x }{l }  \right) \\
&\left. u_{x} \right|_{t=0}=0
\end{aligned}
\right.
\end{equation}

\end{example}

\begin{solution}
    
设 $u(x,t) $ 可分离变量:

\begin{equation}
u(x,t)=U(x)T(t),
\end{equation}

代入一维波动方程 $\displaystyle{u_{tt}-a^2u_{xx}=0 }$ 可得:

\begin{equation}
U(x) T''(t) - a^2 U''(x) T(t) = 0.
\end{equation}

两边同时除以 $U(x)T(t) $ 得到:

\begin{equation}
\frac{T''(t) }{T(t) } - a^2 \frac{U''(x) }{U(x) } = 0.
\end{equation}

观察可知:

\begin{equation}
\frac{T''(t) }{T(t) } = -\omega^2,\quad
\frac{U''(x) }{U(x) } = -k^2,\quad 
\omega^2 = a^2 k^2.
\end{equation}

先看场量 $u(x,t)=T(t) U(x) $ 的时间部分 $T(t) $ 满足的方程:

\begin{equation}
T''(t)+\omega^2 T(t) = 0
\Longrightarrow T(t) = A\cos\omega t+B\sin\omega t,
\end{equation}

\begin{equation}
T'(t)
=-\omega A\sin\omega t + \omega B\cos\omega t,
\end{equation}

\begin{equation}
\left. u_t \right|_{t=0}=0
\Longrightarrow \left. T'(t)\right|_{t=0} = 0
\Longrightarrow B = 0,
\end{equation}

因此:

\begin{equation}
T(t)=A\cos\omega t
\end{equation}

再看场量 $u(x,t)=T(t) U(x) $ 空间部分 $U(x) $ 满足的方程:

\begin{equation}
U''(x) + k^2 U(x) = 0
\Longrightarrow U(x) = C\cos k x + D\sin k x,
\end{equation}

\begin{equation}
U'(x)=-k C\sin k x + k D\cos k x,
\end{equation}

\begin{equation}
\left. u_x \right|_{x=0} = 0,
\Longrightarrow \left. U'(x)\right|_{x=0} = 0,
\Longrightarrow D = 0.
\end{equation}

因此:

\begin{equation}
U(x) = C\cos k x,\quad U'(x) = -kC\sin k x,
\end{equation}

\begin{equation}
\left. u_{x}\right|_{x=l}=0,
\Longrightarrow \left. U'(x)\right|_{x=l} = 0,
\Longrightarrow -kC\sin k l = 0.
\end{equation}

因此,$k $ 的本征值 $k_n $ 为:

\begin{equation}
k_n = \frac{n\pi }{l },\quad n=1,2,\cdots
\end{equation}

相应的本征函数 $U_n(x) $ 为:

\begin{equation}
U_n(x) = \cos k_n x = \cos\left(\frac{n\pi }{l } x \right),\quad n=1,2,\cdots
\end{equation}

由 $\omega = a k $,得 $\omega $ 的本征值 $\omega_n $ 为:

\begin{equation}
\omega_n = a k_n
=\frac{n\pi a}{l },\quad n=1,2,\cdots
\end{equation}

相应的本征函数 $T_n(x) $ 为:

\begin{equation}
T_n(t) = \cos\omega_n t = \cos\left(\frac{n\pi a}{l } t \right),~~n=,1,2,\cdots
\end{equation}

本征解 $u_n(x,t) $ 为:

\begin{equation}
u_n(x,t)
=U_n(t) T_n(t)
=\cos\left(\frac{n\pi }{l } x \right)\cos\left(\frac{n\pi a}{l } t \right),\quad n=,1,2,\cdots
\end{equation}

定解问题的通解 $u(x,t) $ 为:

\begin{equation}
u(x,t)
=\sum_{n=1}^{\infty} E_n u_n(x,t)
=\sum_{n=1}^{\infty} E_n \cos\left(\frac{n\pi }{l } x \right)\cos\left(\frac{n\pi a}{l } t \right) 
\end{equation}

最后结合初始条件

\begin{equation}
\left. u\right|_{t=0} = \cos\left(\frac{\pi x}{l}\right) +0.3\cos\left(\frac{3\pi x }{l }  \right)
\end{equation}

得到:

\begin{equation}
\sum_{n=1}^{\infty} E_n \cos\left(\frac{n\pi }{l } x \right)
=\cos\left(\frac{\pi x}{l}\right) +0.3\cos\left(\frac{3\pi x }{l }  \right)
\end{equation}

两边对比得到系数:

\begin{equation}
E_1 = 1,E_2 = 0,E_3 = 0.3,E_4=E_5=\cdots=0,
\end{equation}

最终得到定解问题的解为:

\begin{equation}
u(x,t)
=\cos\left(\frac{\pi }{l}x \right)\cos\left(\frac{\pi a}{l}t \right) + 0.3\cos\left(\frac{3\pi }{l}x \right)\cos\left(\frac{3\pi a}{l}t \right).
\end{equation}

\end{solution}

\subsubsection{例3}

\begin{example}
    
求定解问题:

\begin{equation}
\left\{
\begin{aligned}
&u_{t}-a^2\nabla^2 u=0 \\
&\left. u\right|_{x=0}=0,\left. u\right|_{x=l_1}=0 \\
&\left. u\right|_{y=0}=0,\left. u\right|_{y=l_2}=0 \\
&\left. u\right|_{z=0}=0,\left. u\right|_{z=l_3}=0 \\
&\left. u\right|_{t=0}=T_0 \\
\end{aligned}
\right.
\end{equation}

\end{example}

\begin{solution}
    
设 $u(x,y,z,t) $ 可分离变量:

\begin{equation}
u(x,y,z,t)=U(x,y,z)T(t),
\end{equation}

代入输运方程可得:

\begin{equation}
U(x,y,z)T'(t) - a^2 T(t) \nabla^2 U(x,y,z)
=0.
\end{equation}

上式两边同时除以 $U(x,y,z)T(t) $ 得:

\begin{equation}
\frac{T'(t) }{ T(t) } = a^2 \frac{\nabla^2  U(x,y,z) }{U(x,y,z) },
\end{equation}

观察可知:

\begin{equation}
\frac{T'(t) }{T(t) } = -\omega^2,\quad
a^2 \frac{\nabla^2  U(x,y,z) }{U(x,y,z) } = -k^2,\quad
\omega^2 = a^2 k^2,\quad 
\end{equation}

时间部分 $T(t) $ 满足方程:

\begin{equation}
T'(t) + \omega^2 T(t) = 0,
\end{equation}

空间部分 $U(x,y,z) $ 满足方程:

\begin{equation}
\nabla^2 U(x,y,z) + k^2 U(x,y,z) = 0
\end{equation}

设 $U(x,y,z) $ 可分离变量:

\begin{equation}
U(x,y,z) = X(x)Y(y)Z(z)
\end{equation}

代入 $U(x,y,z) $ 满足的方程,得:

\begin{equation}
Y(y)Z(z)X''(x) + X(x)Z(z)Y''(y) + X(x)Y(y)Z''(z) + k^2 X(x)Y(y)Z(z)
=0
\end{equation}

上式两边同时除以 $X(x)Y(y)Z(z) $ 得:

\begin{equation}
\frac{X''(x) }{X(x) } + \frac{Y''(y) }{Y(y) } + \frac{Z''(z) }{Z(z) } + k^2 
=0
\end{equation}

注意到,$\displaystyle{\frac{X''(x) }{X(x) } , \frac{Y''(y) }{Y(y) } , \frac{Z''(z) }{Z(z) } }$ 分别只与 $x,y,z $ 有关。上式成立,则有:

\begin{equation}
\frac{X''(x) }{X(x) } = -k_x^2,\quad k_x>0
\end{equation}

\begin{equation}
\frac{Y''(y) }{Y(y) } = -k_y^2,\quad k_y>0
\end{equation}

\begin{equation}
\frac{Z''(z) }{Z(z) } = -k_z^2,\quad k_z>0
\end{equation}

\begin{equation}
k_x^2 + k_y^2+k_z^2 = k^2
\end{equation}

即:

\begin{equation}
X''(x) + k_x^2 X(x) = 0
\end{equation}

\begin{equation}
Y''(y) + k_y^2 Y(y) = 0
\end{equation}

\begin{equation}
Z''(z) + k_z^2 Z(z) = 0
\end{equation}

\begin{equation}
k_x^2 + k_y^2+k_z^2 = k^2
\end{equation}

方程 $\displaystyle{X''(x) + k_x^2 X(x) = 0 }$ 的通解为:

\begin{equation}
X(x) = A_x \cos\left(k_x x \right) + B_x \sin\left(k_x x \right)
\end{equation}

边界条件:

\begin{equation}
\left. u\right|_{x=0}=0,\left. u\right|_{x=l_1}=0
\Longrightarrow \left. X(x)\right|_{x=0}=0,\left. X(x)\right|_{x=l_1}=0
\end{equation}

由 $\left. X(x)\right|_{x=0}=0 $ 可知:

\begin{equation}
A_x = 0,
\end{equation}

因此:

\begin{equation}
X(x) = B_x \sin\left(k_x x \right).
\end{equation}

再由 $X(x)\bigg|_{x=l_1}=0 $ 可得 $k_x $ 的本征值 $k_x^{(n_x)} $ 为:

\begin{equation}
k_x^{(n_x)} = \frac{n_x\pi }{l_1 } ,\quad n_x=1,2,\cdots
\end{equation}

相应的本征函数 $X_{n_x}(x) $ 为:

\begin{equation}
X_{n_x}(x) = \sin\left(\frac{n_x\pi x}{l_1 } \right),\quad n_x=1,2,\cdots
\end{equation}

类似的,$k_y $ 的本征值 $k_y^{(n_y)} $ 为:

\begin{equation}
k_y^{(n_y)} = \frac{n_y \pi }{l_2 } ,\quad n_y=1,2,\cdots
\end{equation}

相应的本征函数 $Y_{n_y}(y) $ 为:

\begin{equation}
Y_{n_y}(y) = \sin\left(\frac{n_y\pi y}{l_2 } \right),\quad n_y=1,2,\cdots
\end{equation}

$k_z $ 的本征值 $k_z^{(n_z)} $ 为:

\begin{equation}
k_z^{(n_z)} = \frac{n_z \pi }{l_3 } ,\quad n_z=1,2,\cdots
\end{equation}

相应的本征函数 $Z_{n_z}(z) $ 为:

\begin{equation}
Z_{n_z}(z) = \sin\left(\frac{n_z\pi z}{l_3 } \right),\quad n_z=1,2,\cdots
\end{equation}

由 $k^2=k_x^2+k_y^2+k_z^2 $ 可知,$k $ 的本征值 $k_{n_xn_yn_z} $ 为:

\begin{equation}
\begin{aligned}
k_{n_xn_yn_z}
&=\sqrt{\left(k_x^{(n_x)} \right)^2 + \left(k_y^{(n_y)} \right)^2 + \left(k_z^{(n_z)} \right)^2} \\
&=\sqrt{\left(\frac{n_x\pi }{l_1 } \right)^2 + \left(\frac{n_y\pi }{l_2 } \right)^2 + \left(\frac{n_z\pi }{l_3 } \right)^2} \\
&=\pi\sqrt{\left(\frac{n_x }{l_1 } \right)^2 + \left(\frac{n_y }{l_2 } \right)^2 + \left(\frac{n_z }{l_3 } \right)^2}
\end{aligned}
\end{equation}

由 $U(x,y,z)=X(x)Y(y)Z(z) $ 可知,相应于本征值 $k_{n_xn_yn_z} $ 的本征函数 $U_{n_xn_yn_z} $ 为:

\begin{equation}
U_{n_xn_yn_z}(x,y,z)
=X_{n_x}(x)Y_{n_y}(y)Z_{n_z}(z)
=\sin\left(\frac{n_x\pi x}{l_1 } \right)\sin\left(\frac{n_y\pi y}{l_2 } \right) \sin\left(\frac{n_z\pi z}{l_3 } \right).
\end{equation}

再看时间部分。

由 $\omega = a k $ 可知,$\omega $ 的本征值 $\omega_{n_xn_yn_z} $ 为:

\begin{equation}
\omega_{n_xn_yn_z}
=a k_{n_xn_yn_z}
=\pi a \sqrt{\left(\frac{n_x }{l_1 } \right)^2 + \left(\frac{n_y }{l_2 } \right)^2 + \left(\frac{n_z }{l_3 } \right)^2}.
\end{equation}

方程 $T'(t) + \omega^2 T(t) = 0 $ 的特解为:

\begin{equation}
T(t) = \exp\left(-\omega^2 t \right),
\end{equation}

$T(t) $ 对应本征值 $\omega_{n_xn_yn_z} $ 的本征函数 $T_{n_xn_yn_z}(t) $ 为:

\begin{equation}
T_{n_xn_yn_z}(t)
=\mathrm{e}^{-\omega_{n_xn_yn_z}^2 t}
\end{equation}

由 $u(x,y,z,t)=U(x,y,z)T(t) $ 可知,$u(x,y,z,t) $ 的本征函数 $u_{n_xn_yn_z}(x,y,z,t) $ 为:

\begin{equation}
\begin{aligned}
u_{n_xn_yn_z}(x,y,z,t)
&=U_{n_xn_yn_z}(x,y,z) T_{n_xn_yn_z}(t) \\
&=\mathrm{e}^{-\omega_{n_xn_yn_z}^2 t} \sin\left(\frac{n_x\pi x}{l_1 } \right)\sin\left(\frac{n_y\pi y}{l_2 } \right) \sin\left(\frac{n_z\pi z}{l_3 } \right) \\
\end{aligned}
\end{equation}

因此,定解问题的形式解 $u(x,y,z,t) $ 为:

\begin{equation}
\begin{aligned}
u(x,y,z,t)
&=\sum_{n_x=1}^{\infty} \sum_{n_y=1}^{\infty} \sum_{n_z=1}^{\infty} C_{n_xn_yn_z}u_{n_xn_yn_z}(x,y,z,t) \\
&=\sum_{n_x=1}^{\infty} \sum_{n_y=1}^{\infty} \sum_{n_z=1}^{\infty} C_{n_xn_yn_z}\mathrm{e}^{-\omega_{n_xn_yn_z}^2 t} \sin\left(\frac{n_x\pi x}{l_1 } \right)\sin\left(\frac{n_y\pi y}{l_2 } \right) \sin\left(\frac{n_z\pi z}{l_3 } \right) \\
&=\sum_{n_x=1}^{\infty} \sum_{n_y=1}^{\infty} \sum_{n_z=1}^{\infty} C_{n_xn_yn_z}\exp\left\{-\pi^2 a^2 \left[\left(\frac{n_x }{l_1 } \right)^2 + \left(\frac{n_y }{l_2 } \right)^2 + \left(\frac{n_z }{l_3 } \right)^2 \right]t  \right\} \times \\
&\sin\left(\frac{n_x\pi x}{l_1 } \right)\sin\left(\frac{n_y\pi y}{l_2 } \right) \sin\left(\frac{n_z\pi z}{l_3 } \right) \\
\end{aligned}
\end{equation}

结合初始条件 $u(x,y,z,t)\bigg|_{t=0} = T_0 $,得:

\begin{equation}
\sum_{n_x=1}^{\infty} \sum_{n_y=1}^{\infty} \sum_{n_z=1}^{\infty} C_{n_xn_yn_z} \sin\left(\frac{n_x\pi x}{l_1 } \right)\sin\left(\frac{n_y\pi y}{l_2 } \right) \sin\left(\frac{n_z\pi z}{l_3 } \right) 
=T_0
\end{equation}

等号两边同乘 $\displaystyle{\sin\left(\frac{n_x'\pi x}{l_1 } \right)\sin\left(\frac{n_y'\pi y}{l_2 } \right) \sin\left(\frac{n_z'\pi z}{l_3 } \right)  }$ 并积分($n_x',n_y',n_z'\in \left\{1,2,\cdots \right\} $):

\begin{equation}
\begin{aligned}
&\sum_{n_x=1}^{\infty} \sum_{n_y=1}^{\infty} \sum_{n_z=1}^{\infty} C_{n_xn_yn_z} \int_{x=0}^{x=l_1} \sin\left(\frac{n_x\pi x}{l_1 } \right)\sin\left(\frac{n_x'\pi x}{l_1 } \right)\mathrm{d}x \times \\
&\int_{y=0}^{y=l_2} \sin\left(\frac{n_y\pi y}{l_2 } \right)\sin\left(\frac{n_y'\pi y}{l_2 } \right)\mathrm{d}y \int_{z=0}^{z=l_3} \sin\left(\frac{n_z\pi z}{l_3 } \right) \sin\left(\frac{n_z'\pi z}{l_3 } \right) \mathrm{d}z \\
=&T_0 \int_{x=0}^{x=l_1}\sin\left(\frac{n_x'\pi x}{l_1 } \right)\mathrm{d}x \int_{y=0}^{y=l_2}\sin\left(\frac{n_y'\pi y}{l_2 } \right)\mathrm{d}y\int_{z=0}^{z=l_3}\sin\left(\frac{n_z'\pi z}{l_3 } \right) \mathrm{d}z
\end{aligned}
\end{equation}

注意到:

\begin{equation}
\begin{aligned}
\int_{x=0}^{x=l_1}\sin\left(\frac{n_x'\pi x}{l_1 } \right)\mathrm{d}x
&=\frac{l_1 }{n_x'\pi } \int_{x=0}^{x=l_1}\sin\left(\frac{n_x'\pi x}{l_1 } \right)\mathrm{d}\left(\frac{n_x'\pi x }{l_1 }  \right) \\
&=\frac{-l_1 }{n_x'\pi }\int_{x=0}^{x=l_1}\mathrm{d}\left[\cos\left(\frac{n_x'\pi x }{l_1 }  \right) \right] \\
&=\frac{-l_1 }{n_x'\pi }\cdot \cos\left(\frac{n_x'\pi x }{l_1 }  \right)\bigg|_{x=0}^{x=l_1} \\
&=\frac{-l_1 }{n_x'\pi }\cdot\left[\cos\left(n_x'\pi \right) - 1 \right] \\
&=\frac{l_1 }{n_x'\pi }\left[1-\left(-1 \right)^{n_x'} \right]
\end{aligned}
\end{equation}

利用积化和差公式 $\displaystyle{\sin\alpha\sin\beta=\frac{1 }{2 }\left[\cos(\alpha-\beta) - \cos(\alpha+\beta) \right] }$,有:

\begin{equation}
\begin{aligned}
&\int_{x=0}^{x=l_1} \sin\left(\frac{n_x\pi x}{l_1 } \right)\sin\left(\frac{n_x'\pi x}{l_1 } \right)\mathrm{d}x \\
=&\frac{1 }{2 } \int_{x=0}^{x=l_1} \cos\left(\frac{\left(n_x-n_x' \right)\pi x }{l_1 } \right)\mathrm{d}x - \frac{1 }{2 } \int_{x=0}^{x=l_1} \cos\left(\frac{\left(n_x+n_x' \right)\pi x }{l_1 } \right)\mathrm{d}x
\end{aligned}
\end{equation}

注意到:

\begin{equation}
\begin{aligned}
\int_{x=0}^{x=l_1} \cos\left(\frac{\left(n_x+n_x' \right)\pi x }{l_1 } \right)\mathrm{d}x
&=\frac{l_1 }{\left(n_x+n_x' \right)\pi } \int_{x=0}^{x=l_1} \mathrm{d}\left[\sin\left(\frac{\left(n_x+n_x' \right)\pi x }{l_1 } \right) \right] \\
&=\frac{l_1 }{\left(n_x+n_x' \right)\pi } \cdot \sin\left(\frac{\left(n_x+n_x' \right)\pi x }{l_1 } \right)\bigg|_{x=0}^{x=l_1} \\
&=0
\end{aligned}
\end{equation}

再注意到,当 $n_x=n_x' $ 时,

\begin{equation}
\begin{aligned}
\int_{x=0}^{x=l_1} \cos\left(\frac{\left(n_x-n_x' \right)\pi x }{l_1 } \right)\mathrm{d}x
&=l_1
\end{aligned}
\end{equation}

当 $n_x\ne n_x' $ 时,

\begin{equation}
\begin{aligned}
\int_{x=0}^{x=l_1} \cos\left(\frac{\left(n_x-n_x' \right)\pi x }{l_1 } \right)\mathrm{d}x
&=\frac{l_1 }{\left(n_x-n_x' \right)\pi } \int_{x=0}^{x=l_1} \mathrm{d}\left[\sin\left(\frac{\left(n_x-n_x' \right)\pi x }{l_1 } \right) \right] \\
&=\frac{l_1 }{\left(n_x-n_x' \right)\pi } \cdot \sin\left(\frac{\left(n_x-n_x' \right)\pi x }{l_1 } \right)\bigg|_{x=0}^{x=l_1} \\
&=0
\end{aligned}
\end{equation}

因此:

\begin{equation}
\int_{x=0}^{x=l_1} \cos\left(\frac{\left(n_x-n_x' \right)\pi x }{l_1 } \right)\mathrm{d}x
=l_1\delta_{n_x,n_x'}
\end{equation}

终于,我们可以计算:

\begin{equation}
\begin{aligned}
&\int_{x=0}^{x=l_1} \sin\left(\frac{n_x\pi x}{l_1 } \right)\sin\left(\frac{n_x'\pi x}{l_1 } \right)\mathrm{d}x \\
=&\frac{1 }{2 } \int_{x=0}^{x=l_1} \cos\left(\frac{\left(n_x-n_x' \right)\pi x }{l_1 } \right)\mathrm{d}x - \frac{1 }{2 } \int_{x=0}^{x=l_1} \cos\left(\frac{\left(n_x+n_x' \right)\pi x }{l_1 } \right)\mathrm{d}x \\
=&\frac{1 }{2 } l_1\delta_{n_x,n_x'} - \frac{1 }{2 } \cdot 0 \\
=&\frac{l_1 }{2 } \delta_{n_x,n_x'}
\end{aligned}
\end{equation}

于是,下面这个复杂的方程

\begin{equation}
\begin{aligned}
&\sum_{n_x=1}^{\infty} \sum_{n_y=1}^{\infty} \sum_{n_z=1}^{\infty} C_{n_xn_yn_z} \int_{x=0}^{x=l_1} \sin\left(\frac{n_x\pi x}{l_1 } \right)\sin\left(\frac{n_x'\pi x}{l_1 } \right)\mathrm{d}x \times \\
&\int_{y=0}^{y=l_2} \sin\left(\frac{n_y\pi y}{l_2 } \right)\sin\left(\frac{n_y'\pi y}{l_2 } \right)\mathrm{d}y \int_{z=0}^{z=l_3} \sin\left(\frac{n_z\pi z}{l_3 } \right) \sin\left(\frac{n_z'\pi z}{l_3 } \right) \mathrm{d}z \\
=&T_0 \int_{x=0}^{x=l_1}\sin\left(\frac{n_x'\pi x}{l_1 } \right)\mathrm{d}x \int_{y=0}^{y=l_2}\sin\left(\frac{n_y'\pi y}{l_2 } \right)\mathrm{d}y\int_{z=0}^{z=l_3}\sin\left(\frac{n_z'\pi z}{l_3 } \right) \mathrm{d}z
\end{aligned}
\end{equation}

可简化为:

\begin{equation}
\begin{split}
&\sum_{n_x=1}^{\infty} \sum_{n_y=1}^{\infty} \sum_{n_z=1}^{\infty} C_{n_xn_yn_z} \frac{l_1l_2l_3 }{8 } \delta_{n_x,n_x'}\delta_{n_y,n_y'}\delta_{n_z,n_z'} \\
=&T_0\frac{l_1l_2l_3 }{n_x'n_y'n_z'\pi^3 }\left[1-\left(-1 \right)^{n_x'} \right]\left[1-\left(-1 \right)^{n_y'} \right]\left[1-\left(-1 \right)^{n_z'} \right]
\end{split}
\end{equation}

继续化简:

\begin{equation}
C_{n_x'n_y'n_z'}\cdot\frac{1 }{8 } 
=T_0\frac{1 }{n_x'n_y'n_z'\pi^3 } \left[1-\left(-1 \right)^{n_x'} \right]\left[1-\left(-1 \right)^{n_y'} \right]\left[1-\left(-1 \right)^{n_z'} \right]
\end{equation}

解得系数:

\begin{equation}
C_{n_x'n_y'n_z'}
=\frac{8 }{n_x'n_y'n_z'\pi^3 } T_0\left[1-\left(-1 \right)^{n_x'} \right]\left[1-\left(-1 \right)^{n_y'} \right]\left[1-\left(-1 \right)^{n_z'} \right]
\end{equation}

把下标 $n_x',n_y,n_z' $ 替换为 $n_x,n_y,n_z $:

\begin{equation}
C_{n_xn_yn_z}
=\frac{8 }{n_xn_yn_z\pi^3 } T_0\left[1-\left(-1 \right)^{n_x} \right]\left[1-\left(-1 \right)^{n_y} \right]\left[1-\left(-1 \right)^{n_z} \right]
\end{equation}

综上,定解问题的解 $u(x,y,z,t) $ 为:

\begin{equation}
\begin{aligned}
u(x,y,z,t)
&=\sum_{n_x=1}^{\infty} \sum_{n_y=1}^{\infty} \sum_{n_z=1}^{\infty} C_{n_xn_yn_z}\exp\left\{-\pi^2 a^2 \left[\left(\frac{n_x }{l_1 } \right)^2 + \left(\frac{n_y }{l_2 } \right)^2 + \left(\frac{n_z }{l_3 } \right)^2 \right]t  \right\} \times \\
&\sin\left(\frac{n_x\pi x}{l_1 } \right)\sin\left(\frac{n_y\pi y}{l_2 } \right) \sin\left(\frac{n_z\pi z}{l_3 } \right) \\
&=\sum_{n_x=1}^{\infty} \sum_{n_y=1}^{\infty} \sum_{n_z=1}^{\infty} \frac{8 }{n_xn_yn_z\pi^3 } T_0\left[1-\left(-1 \right)^{n_x} \right]\left[1-\left(-1 \right)^{n_y} \right]\left[1-\left(-1 \right)^{n_z} \right] \times \\
&\exp\left\{-\pi^2   a^2 \left[\left(\frac{n_x }{l_1 } \right)^2 + \left(\frac{n_y }{l_2 } \right)^2 + \left(\frac{n_z }{l_3 } \right)^2 \right]t  \right\}\times \\
&\sin\left(\frac{n_x\pi x}{l_1 } \right)\sin\left(\frac{n_y\pi y}{l_2 } \right) \sin\left(\frac{n_z\pi z}{l_3 } \right) \\
\end{aligned}
\end{equation}


\end{solution}

\section{曲线坐标系下的分离变量}

\subsection{亥姆霍兹方程在球坐标系下的分离变量}

在球坐标系下,亥姆霍兹方程为:

\begin{equation}
\nabla^2 u(r,\theta,\varphi) + k^2 u(r,\theta,\varphi) = 0,
\end{equation}

其中,拉普拉斯算子 $\nabla^2 $ 在球坐标系下的表达式为:

\begin{equation}
\nabla^2
=\frac{1}{r^2}\frac{\partial}{\partial r}\left(r^2\frac{\partial}{\partial r}\right) + \frac{1}{r^2\sin\theta}\frac{\partial}{\partial \theta}\left(\sin\theta\frac{\partial}{\partial\theta}\right) + \frac{1}{r^2\sin^2\theta}\frac{\partial^2}{\partial \varphi^2}.
\end{equation}

设 $u(r,\theta,\varphi) $ 可分离变量:

\begin{equation}
u(r,\theta,\varphi) = R(r)Y(\theta,\varphi) = R(r) \Theta(\theta)\Phi(\varphi),
\end{equation}

代入亥姆霍兹方程可得:

\begin{equation}
\begin{split}
&k^2 R(r) Y(\theta,\varphi) + Y(\theta,\varphi) \frac{1}{r^2}\frac{\partial}{\partial r}\left(r^2\frac{\partial}{\partial r} R(r)\right) \\
+&R(r) \frac{1}{r^2\sin\theta}\frac{\partial}{\partial \theta}\left(\sin\theta\frac{\partial}{\partial\theta}Y(\theta,\varphi) \right) + R(r) \frac{1}{r^2\sin^2\theta}\frac{\partial^2}{\partial \varphi^2} Y(\theta,\varphi)
=0.
\end{split}
\end{equation}

上式两边同除 $R(r) Y(\theta,\varphi) / r^2 $ 得:

\begin{equation}
\begin{split}
&k^2 r^2 + \frac{1 }{R(r) } \frac{\partial}{\partial r}\left(r^2\frac{\partial}{\partial r} R(r)\right) \\
+ &\frac{1}{Y(\theta,\varphi)\sin\theta}\frac{\partial}{\partial \theta}\left(\sin\theta\frac{\partial}{\partial\theta}Y(\theta,\varphi) \right) + \frac{1}{Y(\theta,\varphi)\sin^2\theta}\frac{\partial^2}{\partial \varphi^2} Y(\theta,\varphi)
=0.
\end{split}
\end{equation}

由上式可知,径向部分 $R(r) $ 满足\textbf{球贝塞尔方程}:

\begin{equation}
k^2 r^2 + \frac{1 }{R(r) } \frac{\partial}{\partial r}\left(r^2\frac{\partial}{\partial r} R(r)\right)
=l(l+1),
\end{equation}

也即

\begin{equation}
\frac{\mathrm{d}^2 R(r) }{\mathrm{d}r^2 } + \frac{2 }{r } \frac{\mathrm{d}R(r) }{\mathrm{d}r } + \left[k^2 - \frac{l(l+1) }{r^2 }  \right]R(r)
=0.
\end{equation}

角度部分 $Y(\theta,\varphi) $ 满足

\begin{equation}
\frac{1}{Y(\theta,\varphi)\sin\theta}\frac{\partial}{\partial \theta}\left(\sin\theta\frac{\partial}{\partial\theta}Y(\theta,\varphi) \right) + \frac{1}{Y(\theta,\varphi)\sin^2\theta}\frac{\partial^2}{\partial \varphi^2} Y(\theta,\varphi)
=-l(l+1),
\end{equation}

也即

\begin{equation}
\frac{1 }{\sin\theta } \frac{\partial }{\partial \theta } \left(\sin\theta\frac{\partial Y(\theta,\varphi) }{\partial \theta }  \right) + \frac{1 }{\sin^2\theta } \frac{\partial^2 Y(\theta,\varphi) }{\partial \varphi^2 } + l(l+1)Y(\theta,\varphi)
=0.
\end{equation}

设角度部分 $Y(\theta,\varphi)=\Theta(\theta) \Phi(\varphi) $,代入上式可得:

\begin{equation}
\Phi(\varphi) \frac{1 }{\sin\theta } \frac{\partial }{\partial\theta } \left(\sin\theta \frac{\partial \Theta(\theta) }{\partial \theta }  \right) + \frac{1 }{\sin^2\theta } \Theta(\theta) \frac{\partial^2 \Phi(\varphi) }{\partial \varphi^2 } + l(l+1) \Theta(\theta) \Phi(\varphi)
=0.
\end{equation}

两边同除 $\Theta(\theta) \Phi(\varphi) / \sin^2\theta $ 得:

\begin{equation}
\frac{\sin\theta }{\Theta(\theta) } \frac{\mathrm{d} }{\mathrm{d}\theta } \left(\sin\theta \frac{\mathrm{d}\Theta(\theta) }{\mathrm{d}\theta }  \right) + l(l+1)\sin^2\theta + \frac{1 }{\Phi(\varphi) } \frac{\mathrm{d}^2\Phi(\varphi ) }{\mathrm{d}\varphi^2 } 
=0.
\end{equation}

由上式可知,方位角部分 $\Phi(\varphi) $ 满足

\begin{equation}
\frac{1 }{\Phi(\varphi) } \frac{\mathrm{d}^2\Phi(\varphi ) }{\mathrm{d}\varphi^2 } 
=-m^2,
\end{equation}

也即

\begin{equation}
\Phi''(\varphi) + m^2 \Phi(\varphi)
=0.
\end{equation}

极角部分 $\Theta(\theta) $ 满足

\begin{equation}
\frac{1 }{\sin\theta }\frac{\mathrm{d} }{\mathrm{d}\theta } \left(\sin\theta\frac{\mathrm{d}\Theta(\theta) }{\mathrm{d}\theta }  \right) + \left[l(l+1)-\frac{m^2 }{\sin^2\theta }  \right] \Theta(\theta)
=0.
\end{equation}

令 $x=\cos\theta,\Theta(\theta)=y(x),\theta\in(0,\pi),\left|x \right|<1,\sin\theta=\sqrt{1-x^2} $,

\begin{equation}
\mathrm{d}x
=-\sin\theta\mathrm{d}\theta,
\end{equation}

\begin{equation}
\frac{\mathrm{d} }{\mathrm{d}\theta } 
=\frac{\mathrm{d}x }{\mathrm{d}\theta } \frac{\mathrm{d} }{\mathrm{d}x } 
=-\sin\theta\frac{\mathrm{d} }{\mathrm{d}x } ,
\end{equation}

\begin{equation}
\begin{split}
\frac{1 }{\sin\theta }\frac{\mathrm{d} }{\mathrm{d}\theta } \left(\sin\theta\frac{\mathrm{d}\Theta(\theta) }{\mathrm{d}\theta }  \right)
&=\frac{1 }{\sin\theta }\frac{\mathrm{d} }{\mathrm{d}\theta } \left(-\sin^2\theta\frac{\mathrm{d}y(x) }{\mathrm{d}x } \right) \\
&=\frac{1 }{\sin\theta }\frac{\mathrm{d} }{\mathrm{d}\theta } \left(-\left(1-\cos^2 \theta \right) \frac{\mathrm{d}y(x) }{\mathrm{d}x } \right) \\
&=\frac{1 }{\sin\theta }\frac{\mathrm{d} }{\mathrm{d}\theta } \left(-\left(1-x^2 \right)\frac{\mathrm{d}y(x) }{\mathrm{d}x } \right) \\
&=\frac{1 }{\sin\theta } \cdot \left(-\sin\theta \right) \frac{\mathrm{d} }{\mathrm{d}x } \left(-\left(1-x^2 \right)\frac{\mathrm{d}y(x) }{\mathrm{d}x } \right) \\
&=\frac{\mathrm{d} }{\mathrm{d}x } \left(\left(1-x^2 \right)\frac{\mathrm{d}y(x) }{\mathrm{d}x } \right) \\
&=\left(1-x^2 \right)\frac{\mathrm{d}^2y(x) }{\mathrm{d}x^2 } - 2x\frac{\mathrm{d}y(x) }{\mathrm{d}x },
\end{split}
\end{equation}

则方程可化为\textbf{连带勒让德方程}:

\begin{equation}
(1-x^2)\frac{\mathrm{d}^2 y }{\mathrm{d}x^2 } - 2x\frac{\mathrm{d}y }{\mathrm{d}x } + \left[l(l+1)-\frac{m^2 }{1-x^2 }  \right]y
=0.
\end{equation}

\subsection{亥姆霍兹方程在柱坐标系下的分离变量}

在柱坐标系下,亥姆霍兹方程为:

\begin{equation}
\nabla^2 u(\rho,\varphi,z) + k^2 u(\rho,\varphi,z) = 0,
\end{equation}

其中,拉普拉斯算子在柱坐标系下的表达式为:

\begin{equation}
\nabla^2
=\frac{1}{\rho}\frac{\partial}{\partial \rho}\left(\rho\frac{\partial }{\partial \rho}\right)+\frac{1}{\rho^2}\frac{\partial^2}{\partial \varphi^2}+\frac{\partial^2}{\partial z^2}
\end{equation}

设 $u(\rho,\varphi,z) $ 可分离变量:

\begin{equation}
u(\rho,\varphi,z) = R(\rho)\Phi(\varphi)Z(z),
\end{equation}

经过运算,可以得到,$\Phi(\varphi) $ 满足

\begin{equation}
\Phi''(\varphi) + \nu^2\Phi(\varphi)
=0,\quad \nu\geqslant 0,
\end{equation}

\begin{equation}
Z''(z) - \lambda Z(z)
=0,
\end{equation}

\begin{equation}
\frac{1 }{\rho } \frac{\mathrm{d} }{\mathrm{d}\rho }\left(\rho\frac{\mathrm{d}R(\rho) }{\mathrm{d}\rho }  \right) + \left(k^2+\lambda-\frac{\nu^2 }{\rho^2 }  \right)R(\rho)
=0.
\end{equation}

令 $x=\sqrt{k^2+\lambda} \rho,\rho=x/\sqrt{k^2+\lambda},(k^2+\lambda\ne 0),R(\rho) \equiv y(x),y(x)=R(\rho) $,则上面方程可化为\textbf{贝塞尔方程}:

\begin{equation}
\frac{\mathrm{d}^2 y(x) }{\mathrm{d}x^2 } + \frac{1 }{x } \frac{\mathrm{d}y(x) }{\mathrm{d}x } + \left(1-\frac{\nu^2 }{x^2 }  \right) y(x)
=0,\quad \nu\geqslant 0.
\end{equation}

或称为 $\nu $ 阶\textbf{贝塞尔方程}。


