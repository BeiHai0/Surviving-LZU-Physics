\chapter{\texorpdfstring{$\delta$函数}{delta函数}}

\section{\texorpdfstring{$\delta$函数的定义}{delta函数的定义}}

$\delta$ 函数是一个定义在 $\mathbb{R}$ 上的广义函数,其满足:

\begin{equation}
\delta(x-x_0)
=\begin{cases}
0&,x\ne x_0 \\
+\infty&, x=x_0 
\end{cases},\quad
\text{且}
\int_a^b\delta (x-x_0)\mathrm{d}x
=\begin{cases}
1&,x_0\in(a,b) \\
0&,x_0\notin (a,b)
\end{cases}
\end{equation}

\section{\texorpdfstring{$\delta$函数的性质}{delta函数的性质}}

\begin{enumerate}
\item 设 $f(x)$ 为连续函数,则:

\begin{equation}
\int_{-\infty}^{+\infty} f(x)\delta(x-x_0) \mathrm{d}x
=f(x_0).
\end{equation}

\item $\delta(x)$ 是偶函数:

\begin{equation}
\delta(-x)
=\delta(x).
\end{equation}

\item 

\begin{equation}
f(x)\delta(x-x_0)
=f(x_0)\delta(x-x_0).
\end{equation}

\item 

\begin{equation}
x\delta(x)
=0.
\end{equation}

\item 

\begin{equation}
\int_{-\infty}^{+\infty} \delta(x-x_2)\delta(x-x_1)\mathrm{d}x
=\delta(x_1-x_2).
\end{equation}

\item 设 $\{x_i\}$ 为 $\varphi(x)$ 的单根,即 $\varphi(x_i)=0 $ 且 $\varphi'(x_i)\ne 0 $,则:

\begin{equation}
\delta(\varphi(x))
=\sum_{i}\frac{1}{|\varphi'(x_i)|}\delta(x-x_i).
\end{equation}

\end{enumerate}

\section{\texorpdfstring{三维$\delta$函数}{三维delta函数}}

\section{\texorpdfstring{三维$\delta$函数的定义}{三维delta函数的定义}}

\begin{equation}
\delta\left(\vec{r}-\vec{r}_0\right)
=\begin{cases}
0&,\vec{r}\ne\vec{r}_0 \\
+\infty&,\vec{r}=\vec{r}_0
\end{cases},\quad \text{且}
\int\limits_{V}\delta(\vec{r}-\vec{r}_0)\mathrm{d}^3\vec{r}
=1,\quad
\vec{r}_0\in V.
\end{equation}

\subsection{三维直角坐标系}

\begin{equation}
\mathrm{d}^3\vec{r}
=\mathrm{d}x\mathrm{d}y\mathrm{d}z,
\end{equation}

\begin{equation}
\delta(\vec{r}-\vec{r}_0)
=\delta(x-x_0)\delta(y-y_0)\delta(z-z_0).
\end{equation}

\subsection{三维球坐标系}

\begin{equation}
\mathrm{d}^3\vec{r}
=r^2\sin\theta\mathrm{d}r\mathrm{d}\theta\mathrm{d}\varphi,
\end{equation}

\begin{equation}
\delta(\vec{r}-\vec{r}_0)
=\frac{1}{r^2\sin\theta}\delta(r-r_0)\delta(\theta-\theta_0)\delta(\varphi-\varphi_0).
\end{equation}

\subsection{三维柱坐标系}

\begin{equation}
\mathrm{d}^3\vec{r}
=\rho\mathrm{d}\rho\mathrm{d}\varphi\mathrm{d}z,
\end{equation}

\begin{equation}
\delta(\vec{r}-\vec{r}_0)
=\frac{1}{\rho}\delta(\rho-\rho_0)\delta(\varphi-\varphi_0)\delta(z-z_0).
\end{equation}

\section{\texorpdfstring{$\delta$函数的傅里叶积分展开}{delta函数傅里叶积分展开}}

\subsection{一维情况}

设 $\delta(x-x_0) $ 的傅里叶积分展开为:

\begin{equation}
\delta(x-x_0)
=\frac{1 }{2\pi } \int_{k=-\infty}^{k=+\infty} C(k)\mathrm{e}^{\mathrm{i}kx}\mathrm{d}k,
\end{equation}

其中 $C(k) $ 就是 $\delta(x-x_0) $ 的傅里叶变换。$C(k) $ 由下式给出:

\begin{equation}
C(k)
=\int_{k=-\infty}^{k=+\infty}\delta(x-x_0)\mathrm{e}^{-\mathrm{i}kx}\mathrm{d}x
=\mathrm{e}^{-\mathrm{i}kx_0}.
\end{equation}

代回 $\delta(x-x_0) $ 的傅里叶积分展式,得到:

\begin{equation}
\begin{split}
\delta(x-x_0)
&=\frac{1 }{2\pi } \int_{k=-\infty}^{k=+\infty} C(k)\mathrm{e}^{\mathrm{i}kx}\mathrm{d}k \\
&=\frac{1 }{2\pi } \int_{k=-\infty}^{k=+\infty} \mathrm{e}^{-\mathrm{i}kx_0} \mathrm{e}^{\mathrm{i}kx}\mathrm{d}k \\
&=\frac{1 }{2\pi } \int_{k=-\infty}^{k=+\infty} \mathrm{e}^{\mathrm{i}k(x-x_0)}\mathrm{d}k.
\end{split}
\end{equation}

\subsection{三维情况}

\begin{equation}
\begin{split}
&\delta(\vec{r}-\vec{r}_0) \\
=&\delta(x-x_0)\delta(y-y_0)\delta(z-z_0) \\
=&\left(\frac{1}{2\pi}\int_{k_x=-\infty}^{k_x=+\infty} \mathrm{e}^{\mathrm{i}k_x(x-x_0)}\mathrm{d}k_x \right)\left(\frac{1}{2\pi}\int_{k_y=-\infty}^{k_y=+\infty} \mathrm{e}^{\mathrm{i}k_y(y-y_0)}\mathrm{d}k_y \right)\left(\frac{1}{2\pi}\int_{k_z=-\infty}^{k_z=+\infty} \mathrm{e}^{\mathrm{i}k_z(z-z_0)}\mathrm{d}k_z \right) \\
=&\frac{1 }{\left(2\pi \right)^3 } \int_{k_x=-\infty}^{k_x=+\infty} \int_{k_y=-\infty}^{k_y=+\infty} \int_{k_z=-\infty}^{k_z=+\infty} \mathrm{e}^{\mathrm{i}k_x(x-x_0)} \mathrm{e}^{\mathrm{i}k_y(y-y_0)} \mathrm{e}^{\mathrm{i}k_z(z-z_0)} \mathrm{d}k_x \mathrm{d}k_y \mathrm{d}k_z \\
=&\frac{1 }{\left(2\pi \right)^3 } \int\limits_{\vec{k}\in \mathbb{R}^3} \mathrm{e}^{\mathrm{i}\vec{k}\cdot(\vec{r}-\vec{r}_0)}\mathrm{d}^3\vec{k}.
\end{split}
\end{equation}

\section{例题}

\begin{example}
    
证明:$\displaystyle{\delta(\vec{r})=-\frac{1}{4\pi}\nabla^2\frac{1}{r} }$. 

\end{example}

\begin{solution}
    
对于标量函数 $f(r) $,有:

\begin{equation}
\begin{split}
\nabla f(r)
&=\vec{\mathrm{e}}_i \partial_i f(r)
=\vec{\mathrm{e}}_i \frac{\mathrm{d} f(r) }{\mathrm{d} r } \frac{\partial r }{\partial x_i } 
=f'(r) \vec{\mathrm{e}}_i \partial_i r
=f'(r) \nabla r \\
&=f'(r) \vec{\mathrm{e}}_i \partial_i \sqrt{x_j x_j} \\
&=f'(r) \vec{\mathrm{e}}_i \frac{1 }{2 } (x_j x_j)^{-1/2} \partial_i (x_j x_j) \\
&=f'(r) \vec{\mathrm{e}}_i \frac{1 }{2 } (x_j x_j)^{-1/2} 2 x_j \partial_i x_j \\
&=f'(r) \vec{\mathrm{e}}_i (x_j x_j)^{-1/2} x_j \delta_{ij} \\
&=f'(r) \vec{\mathrm{e}}_i (x_j x_j)^{-1/2} x_i \\
&=f'(r) \vec{\mathrm{e}}_i x_i (x_j x_j)^{-1/2} \\
&=f'(r) \vec{r} / r.
\end{split}
\end{equation}

对于矢量场 $\vec{A} $ 和标量场 $\phi $,有:

\begin{equation}
\begin{split}
\nabla \cdot \left(\phi \vec{A} \right)
&=\partial_i \left(\phi A_i \right)
=A_i \partial_i \phi + \phi \partial_i A_i
=A_i \left(\nabla \phi \right)_i + \phi \partial_i A_i \\
&=\vec{A} \cdot \left(\nabla \phi \right) + \phi \nabla \cdot \vec{A}
\end{split}
\end{equation}

当 $\vec{r}\ne \vec{0} $,有:

\begin{equation}
\nabla \frac{1 }{r }
=-\frac{1 }{r^2 } \nabla r
=-\frac{1 }{r^2 } \frac{\vec{r} }{r } 
=-\frac{\vec{r} }{r^3 }.
\end{equation}

\begin{equation}
\begin{split}
\nabla^2\frac{1 }{r } 
&=\nabla\cdot\left(\nabla\frac{1 }{r }  \right) \\
&=\nabla\cdot\left(-\frac{\vec{r} }{r^3 }  \right) \\
&=-\left(\vec{r}\cdot\nabla\frac{1 }{r^3 } + \frac{1 }{r^3 } \nabla\cdot\vec{r} \right) \\
&=-\left(\vec{r}\cdot \left(-3r^{-4} \nabla r  \right) + \frac{1 }{r^3 } \cdot 3 \right) \\
&=-\left(\vec{r}\cdot \left(-3r^{-4} \frac{\vec{r} }{r }  \right) + \frac{1 }{r^3 } \cdot 3 \right) \\
&=0.
\end{split}
\end{equation}

$\displaystyle{\nabla^2\frac{1}{r} }$ 在 $\vec{r}=\vec{0} $ 处无定义,但可人为定义其在 $\vec{0} $ 处的函数值为 $+\infty .$

取以坐标原点为球心,半径为 $R $ 的一个球体 $V $,计算体积分:

\begin{equation}
\begin{split}
\int\limits_{\vec{r}\in V} -\frac{1}{4\pi}\nabla^2\frac{1}{r} \mathrm{d}^3\vec{r} 
&=-\frac{1 }{4\pi } \int\limits_{\vec{r}\in V} \nabla\cdot\left(\nabla\frac{1 }{r }  \right)\mathrm{d}^3\vec{r} \\
&=-\frac{1 }{4\pi } \oint\limits_{\partial V^+} \nabla\frac{1 }{r } \cdot\mathrm{d}\vec{S} \\
&=-\frac{1 }{4\pi } \oint\limits_{\partial V^+} -\frac{\vec{r} }{r^3 }  \cdot\mathrm{d}\vec{S} \\
&=\frac{1 }{4\pi }\cdot \frac{1 }{R^2 }  \oint\limits_{\partial V^+} \mathrm{d}S \\
&=\frac{1 }{4\pi R^2 } \cdot 4\pi R^2 \\
&=1.
\end{split}
\end{equation}

\end{solution}