
\chapter{\texorpdfstring{$\mathbb{R}^3$空间曲线坐标系中的向量分析}{R3空间曲线坐标系中的向量分析}}

\section{总结}

\subsection{\texorpdfstring{$\nabla $在三种坐标系下的表达式}{nabla在三种坐标系下的表达式}}

\subsubsection{直角坐标}

\begin{equation}
\nabla
=\vec{\mathrm{e}}_x\frac{\partial}{\partial x}+\vec{\mathrm{e}}_y\frac{\partial}{\partial y}+\vec{\mathrm{e}}_z\frac{\partial }{\partial z}
\end{equation}

\subsubsection{球坐标}

\begin{equation}
\nabla
=\vec{\mathrm{e}}_r \frac{\partial}{\partial r}+\vec{\mathrm{e}}_\theta\frac{1}{r}\frac{\partial}{\partial \theta}+\vec{\mathrm{e}}_\varphi\frac{1}{r\sin\theta}\frac{\partial }{\partial\varphi}
\end{equation}

\subsubsection{柱坐标}

\begin{equation}
\nabla
=\vec{\mathrm{e}}_\rho\frac{\partial}{\partial \rho}+\vec{\mathrm{e}}_\varphi \frac{1}{\rho} \frac{\partial}{\partial \varphi}+\vec{\mathrm{e}}_z\frac{\partial}{\partial z}
\end{equation}

\subsection{\texorpdfstring{$\nabla^2 $在三种坐标系下的表达式}{nabla2在三种坐标系下的表达式}}

\subsubsection{直角坐标}

\begin{equation}
\nabla^2
=\frac{\partial^2 }{\partial x^2 } + \frac{\partial^2 }{\partial y^2 } + \frac{\partial^2 }{\partial z^2 }
\end{equation}

\subsubsection{球坐标}

\begin{equation}
\nabla^2
=\frac{1 }{r^2 } \frac{\partial }{\partial r } \left(r^2 \frac{\partial }{\partial r }  \right) + \frac{1 }{r^2 \sin\theta } \frac{\partial }{\partial \theta } \left(\sin\theta \frac{\partial }{\partial \theta }  \right) + \frac{1 }{r^2 \sin^2\theta } \frac{\partial^2 }{\partial \varphi^2 }
\end{equation}

\subsubsection{柱坐标}

\begin{equation}
\nabla^2
=\frac{1 }{\rho } \frac{\partial }{\partial \rho } \left(\rho \frac{\partial }{\partial \rho }  \right) + \frac{1 }{\rho^2 } \frac{\partial^2 }{\partial \varphi^2 } + \frac{\partial^2 }{\partial z^2 } 
\end{equation}


