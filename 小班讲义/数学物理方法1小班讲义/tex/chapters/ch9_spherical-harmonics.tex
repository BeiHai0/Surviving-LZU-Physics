\chapter{球函数}

\section{勒让德多项式}

\subsection{极轴对称性下连带勒让德方程退化为勒让德方程}

我们已经知道,对球坐标下的亥姆霍兹方程

\begin{equation}
\nabla^2 u(r,\theta,\varphi) + k^2 u(r,\theta,\varphi)
=0
\end{equation}

进行分离变量,即设

\begin{equation}
u(r,\theta,\varphi) = R(r) \Theta(\theta) \Phi(\varphi),
\end{equation}

最后得到三条方程。

径向部分 $R(r) $ 满足\textbf{球贝塞尔方程}:

\begin{equation}
\frac{\mathrm{d}^2 R(r) }{\mathrm{d}r^2 } + \frac{2 }{r } \frac{\mathrm{d}R(r) }{\mathrm{d}r } + \left[k^2 - \frac{l(l+1) }{r^2 }  \right]R(r)
=0.
\end{equation}

极角部分 $\Theta(\theta) $ 满足

\begin{equation}
\frac{1 }{\sin\theta }\frac{\mathrm{d} }{\mathrm{d}\theta } \left(\sin\theta\frac{\mathrm{d}\Theta(\theta) }{\mathrm{d}\theta }  \right) + \left[l(l+1)-\frac{m^2 }{\sin^2\theta }  \right] \Theta(\theta)
=0.
\end{equation}

在进行变量替换 $\cos\theta=x,\Theta(\theta)=y(x) $ 后得到连带勒让德方程

\begin{equation}
(1-x^2)\frac{\mathrm{d}^2 y }{\mathrm{d}x^2 } - 2x\frac{\mathrm{d}y }{\mathrm{d}x } + \left[l(l+1)-\frac{m^2 }{1-x^2 }  \right]y
=0.
\end{equation}

方位角部分 $\Phi(\varphi) $ 满足

\begin{equation}
\Phi''(\varphi) + m^2 \Phi(\varphi)
=0.
\end{equation}

由上面方程,容易得到方位角部分的本征解为:

\begin{equation}
\Phi_m(\varphi)
=\mathrm{e}^{\mathrm{i}m\varphi}.
\end{equation}

由于 $\Phi(\varphi) $ 要满足周期性边界条件($\varphi $转一圈不改变场值)

\begin{equation}
\Phi(\varphi+2\pi) = \Phi(\varphi),
\end{equation}

因此本征值 $m $ 只能取整数,即:

\begin{equation}
m\in \mathbb{Z}.
\end{equation}

现在假设场量 $u(r,\theta,\varphi) $ 具有极轴对称性,也即其值与 $\varphi $ 无关,则本征值 $m $ 只能取 $0 $:

\begin{equation}
m = 0.
\end{equation}

把 $m=0 $ 代回连带勒让德方程就得到\textbf{勒让德方程}:

\begin{equation}
(1-x^2)\frac{\mathrm{d}^2 y }{\mathrm{d}x^2 } - 2x\frac{\mathrm{d}y }{\mathrm{d}x } + l(l+1) y
=0.
\end{equation}

总结一下,在极轴对称性下,亥姆霍兹方程 $\nabla^2 u(r,\theta) + k^2 u(r,\theta) = 0 $ 经分离变量后,径向部分 $R(r) $ 满足仍满足球贝塞尔方程:

\begin{equation}
\frac{\mathrm{d}^2 R(r) }{\mathrm{d}r^2 } + \frac{2 }{r } \frac{\mathrm{d}R(r) }{\mathrm{d}r } + \left[k^2 - \frac{l(l+1) }{r^2 }  \right]R(r)
=0,
\end{equation}

极角部分 $\Theta(\theta)=y(x),x=\cos\theta $ 满足勒让德方程:

\begin{equation}
(1-x^2)\frac{\mathrm{d}^2 y }{\mathrm{d}x^2 } - 2x\frac{\mathrm{d}y }{\mathrm{d}x } + l(l+1) y
=0.
\end{equation}


\subsection{勒让德多项式作为勒让德方程在自然边界条件下的本征函数}

物理上,我们认为场值有限。勒让德方程与自然边界条件就构成S-L本征值问题:

\begin{equation}
\left\{
\begin{aligned}
&x=\cos\theta,\quad\left|x \right|\leqslant 1 \\
&\left(1-x^2\right)\frac{\mathrm{d}^2 y(x) }{\mathrm{d}x^2 } - 2x\frac{\mathrm{d}y(x) }{\mathrm{d}x } + l(l+1) y(x)
=0, \\
&\left|y(x) \right|<+\infty
\end{aligned}
\right.
\end{equation}

利用级数法,可以得到这个S-L本征值问题的本征解:

本征值 $l $ 取自然数 $0,1,2,\cdots $;

相应于本征值 $l $ 的本征函数 $y_l(x) $ 是第一类勒让德多项式(简称勒让德多项式或勒让德函数) $\mathrm{P}_l(x) $:

\begin{equation}
y_l(x) = \mathrm{P}_l(x)
\equiv \sum_{n=0}^{N} \frac{(-1)^n(2l-2n)!}{2^ln!(l-n)!(l-2n)!}x^{l-2n},\quad
N
=\left\{
\begin{aligned}
&\frac{l}{2}&,l\text{为偶数} \\
&\frac{l-1}{2}&,l\text{为奇数}
\end{aligned}
\right.
\end{equation}

\subsection{前几个勒让德多项式}

\begin{equation}
\mathrm{P}_0(x)
=1
\end{equation}

\begin{equation}
\mathrm{P}_1(x)
=x
\end{equation}

\begin{equation}
\mathrm{P}_2(x)
=\frac{1}{2}\left(3x^2-1 \right)
\end{equation}

\begin{equation}
\mathrm{P}_3(x)
=\frac{1}{2}\left(5x^3-3x \right)
\end{equation}

\section{勒让德多项式的性质}

\subsection{罗德里格斯公式}

\begin{equation}
\mathrm{P}_l(x)
=\frac{1}{2^l l! }\frac{\mathrm{d}^l}{\mathrm{d} x^l}\left(x^2-1 \right)^l.
\end{equation}

\subsection{勒让德多项式的生成函数(母函数)}

定义勒让德多项式的生成函数 $f(r) $ 为:

\begin{equation}
f(r)
\equiv \frac{1 }{\sqrt{1+r^2-2r\cos\theta} } 
=\frac{1}{\sqrt{1+r^2-2rx}}
\end{equation}

其中,$x=\cos\theta,\left|x \right|\leqslant 1 $

当 $r<1 $,可将 $f(r) $ 在 $r=0 $ 处进行泰勒展开,可得:

\begin{equation}
f(r)
\equiv\frac{1}{\sqrt{1+r^2-2rx}}
=\sum_{l=0}^{\infty} \mathrm{P}_l(x) r^l,\quad r<1
\end{equation}

或者用 $\theta $ 表达:

\begin{equation}
\boxed{
f(r)
\equiv\frac{1}{\sqrt{1+r^2-2r\cos\theta}}
=\sum_{l=0}^{\infty} \mathrm{P}_l(\cos\theta) r^l,\quad r<1
}
\end{equation}

当 $r>1,\frac{1 }{r } <1 $,有:

\begin{equation}
\begin{aligned}
f(r)
&=\frac{1 }{\sqrt{1+r^2-2rx} } \\
&=\frac{1 }{r\sqrt{1+\left(1/r \right)^2 - 2\left(1/r \right)x} } \\
&=\frac{1 }{r }  \sum_{l=0}^{\infty} \mathrm{P}_l(x)\left(\frac{1 }{r }  \right)^l \\
&=\sum_{l=0}^{\infty}\mathrm{P}_l(x)r^{-(l+1)},\quad r>1
\end{aligned}
\end{equation}

或者用 $\theta $ 表达:

\begin{equation}
\boxed{
f(r)
=\frac{1 }{\sqrt{1+r^2-2r\cos\theta} }
=\sum_{l=0}^{\infty}\mathrm{P}_l(\cos\theta)r^{-(l+1)},\quad r>1
}
\end{equation}

\subsection{勒让德多项式的递推公式}

\begin{equation}
x(1+2l)\mathrm{P}_l(x)-(l+1)\mathrm{P}_{l+1}(x)-l\mathrm{P}_{l-1}(x)
=0
\end{equation}

\begin{equation}
\mathrm{P}_l(x)
=\mathrm{P}_{l+1}'(x)+\mathrm{P}_{l-1}'(x)-2x\mathrm{P}_l'(x)
\end{equation}

\begin{equation}
(2l+1)\mathrm{P}_l(x)
=\mathrm{P}_{l+1}'(x)-\mathrm{P}_{l-1}'(x)
\end{equation}

\begin{equation}
(l+1)\mathrm{P}_l(x)
=\mathrm{P}_{l+1}'(x)-x\mathrm{P}_{l}'(x)
\end{equation}

\begin{equation}
l\mathrm{P}_{l}(x)
=x\mathrm{P}_{l}'(x)-\mathrm{P}_{l-1}'(x)
\end{equation}

\begin{equation}
\left(x^2-1 \right)\mathrm{P}_{l}'(x)
=lx\mathrm{P}_{l}(x)-l\mathrm{P}_{l-1}(x)
\end{equation}

\subsection{勒让德多项式的正交性}

$\displaystyle{\left\{\sqrt{\frac{2l+1}{2}}\mathrm{P}_l(x),\quad l=0,1,2,\cdots \right\} }$ 构成 $[-1,1]$ 上的正交归一函数系,基函数的正交归一性可表达为:

\begin{equation}
\int_{-1}^{1}\sqrt{\frac{2k+1}{2}}\mathrm{P}_k(x)\cdot\sqrt{\frac{2l+1}{2}}\mathrm{P}_l(x)\mathrm{d}x
=\delta_{kl},\quad k,l=0,1,2,\cdots
\end{equation}

\section{具有极轴对称的拉普拉斯方程的求解}

拉普拉斯方程 $\nabla^2 u = 0 $ 可以由亥姆霍兹方程 $\nabla^2 u + k^2 u = 0 $ 令 $k=0 $ 得到。

我们已经知道,在极轴对称性下,亥姆霍兹方程 $\nabla^2 u(r,\theta) + k^2 u(r,\theta) = 0 $ 经分离变量后,径向部分 $R(r) $ 满足满足球贝塞尔方程:

\begin{equation}
\frac{\mathrm{d}^2 R(r) }{\mathrm{d}r^2 } + \frac{2 }{r } \frac{\mathrm{d}R(r) }{\mathrm{d}r } + \left[k^2 - \frac{l(l+1) }{r^2 }  \right]R(r)
=0,
\end{equation}

极角部分 $\Theta(\theta)=y(x),x=\cos\theta $ 满足勒让德方程:

\begin{equation}
(1-x^2)\frac{\mathrm{d}^2 y }{\mathrm{d}x^2 } - 2x\frac{\mathrm{d}y }{\mathrm{d}x } + l(l+1) y
=0.
\end{equation}

令 $k=0 $,就得到具有极轴对称性的拉普拉斯方程 $\nabla^2 u(r,\theta) = 0 $ 在分离变量 $u(r,\theta)=R(r) \Theta(\theta) $ 后 $R(r),\Theta(\theta)=y(x) $ 分别要满足的方程:

\begin{equation}
\frac{\mathrm{d}^2 R(r) }{\mathrm{d}r^2 } + \frac{2 }{r } \frac{\mathrm{d}R(r) }{\mathrm{d}r } - \frac{l(l+1) }{r^2 } R(r)
=0,
\end{equation}

\begin{equation}
(1-x^2)\frac{\mathrm{d}^2 y }{\mathrm{d}x^2 } - 2x\frac{\mathrm{d}y }{\mathrm{d}x } + l(l+1) y
=0.
\end{equation}

对于极角部分 $\Theta(\theta)=y(x) $,结合自然边界条件 $\left|y(x) \right|<+\infty $,就构成S-L本征值问题。我们知道,这个S-L本征值问题的本征值 $l=0,1,2,\cdots $,相应于本征值 $l $ 的本征函数 $y_l(x) $ 为勒让德多项式:

\begin{equation}
y_l(x)
=\mathrm{P}_l(x),
\end{equation}

\begin{equation}
\Theta_l(\theta)
=y_l(x)
=y_l(\cos\theta)
=\mathrm{P}_l(\cos\theta).
\end{equation}

对于径向部分 $R(r) $,可以证明其相应于本征值 $l $ 的本征函数 $R_l(r) $ 为

\begin{equation}
R_l(r)
=C_l r^l + D_l r^{-(l+1)}
\end{equation}

因此,具有极轴对称性的拉普拉斯方程

\begin{equation}
\nabla^2 u(r,\theta) = 0
\end{equation}

在自然边界条件下的形式解为

\begin{equation}
u(r,\theta)
=\sum_{l=0}^{\infty} \left(C_l r^l + D_l r^{-(l+1)} \right) \mathrm{P}_l(\cos\theta).
\end{equation}

\subsection{例题}

\subsubsection{例1}

\begin{example}
    
在单位球的北极点上放置一电荷量为 $4\pi \varepsilon_0$ 的点电荷,求单位球内任一点 $\vec{r} $ 的电势,并用勒让德多项式表示。

\end{example}

\begin{solution}
    
由余弦定理知,单位球内 $\vec{r}$ 点到 单位球上北极点的距离为:

\begin{equation}
r'
=\sqrt{1+r^2-2r\cos\theta},\quad r<1
\end{equation}

单位球内 $\vec{r}$ 点的电势为:

\begin{equation}
\begin{split}
u(\vec{r})
&=\frac{1}{4\pi\varepsilon_0}\frac{q}{r'} \\
&=\frac{1}{4\pi\varepsilon_0}\frac{4\pi\varepsilon_0}{\sqrt{1+r^2-2r\cos\theta}} \\
&=\frac{1}{\sqrt{1+r^2-2r\cos\theta}},\quad r<1
\end{split}
\end{equation}

这恰好是勒让德多项式的生成函数,其可在 $r=0$ 点展开为:

\begin{equation}
u(\vec{r})
=\sum_{l=0}^{\infty}\mathrm{P}_l(\cos\theta)r^{l},\quad r<1
\end{equation}

\end{solution}

\subsubsection{例2}

\begin{example}
    
在半径 $r=r_0$ 的球内求解 $\nabla^2 u=0$,使其满足边界条件 $\left. u\right|_{r=r_0}=\sin^2\theta$.

\end{example}

\begin{solution}
    
边界条件与方位角 $\varphi $ 无关,因此所求应也与 $\varphi $ 无关:

\begin{equation}
\nabla^2 u(r,\theta) = 0.
\end{equation}

套用结论,轴对称问题的拉普拉斯方程在自然边界条件约束下的形式解为:

\begin{equation}
u(r,\theta)
=\sum_{l=0}^{\infty}\left(A_l r^l+B_l r^{-(l+1)} \right)\mathrm{P}_l(\cos\theta)
\end{equation}

由自然边界条件,球心 $r=0 $ 处场量不应发散:

\begin{equation}
\left.\left|u(r,\theta) \right| \right|_{r=0}<+\infty
\end{equation}

因此 $-r^{(l+1)} $ 项必须舍弃,即:

\begin{equation}
B_l = 0,\quad l=0,1,2,\cdots
\end{equation}

于是:

\begin{equation}
u(r,\theta)
=\sum_{l=0}^{\infty} A_l r^l\mathrm{P}_l(\cos\theta).
\end{equation}

考虑边界条件 $\left. u\right|_{r=r_0}=\sin^2\theta=1-\cos^2\theta$,注意到:

\begin{equation}
\left\{
\begin{aligned}
&\mathrm{P}_0(\cos\theta) = 1 \\
&\mathrm{P}_1(\cos\theta) = \cos\theta \\
&\mathrm{P}_2(\cos\theta)= \frac{1 }{2 } \left(3\cos^2\theta-1 \right)
\end{aligned}
\right.
\Longrightarrow 1-\cos^2\theta = \frac{2 }{3 } \left[\mathrm{P}_0(\cos\theta) -\mathrm{P}_2(\cos\theta)\right]
\end{equation}

因此:

\begin{equation}
\sum_{l=0}^{\infty} A_l r_0^l\mathrm{P}_l(\cos\theta)
=\frac{2 }{3 } \left[\mathrm{P}_0(\cos\theta) -\mathrm{P}_2(\cos\theta)\right]
\end{equation}

把边界条件整理成各阶勒让德多项式的线性叠加的形式:

\begin{equation}
\left(A_0-\frac{2 }{3 }  \right)\mathrm{P}_0(\cos\theta) + A_1r_0\mathrm{P}_1(\cos\theta) + \left(A_2r_0^2 + \frac{2 }{3 }  \right)\mathrm{P}_2(\cos\theta) + \sum_{l=3}^{\infty} A_l r_0^l\mathrm{P}_l(\cos\theta)
=0
\end{equation}

由各阶勒让德多项式的正交性,它们的线性叠加为零,当且仅当所有线性叠加系数为零,即:

\begin{equation}
A_0-\frac{2 }{3 } = 0,
A_1 = 0,
A_2 r_0^2 + \frac{2 }{3 }  = 0,
A_3 = A_4 = \cdots = 0
\end{equation}

即:

\begin{equation}
A_0 = \frac{2 }{3 },
A_1 = 0,
A_2 = -\frac{2 }{3r_0^2 },
A_3 = A_4 = \cdots =0
\end{equation}

于是:

\begin{equation}
\begin{split}
u(r,\theta)
&=\sum_{l=0}^{\infty} A_l r^l\mathrm{P}_l(\cos\theta) \\
&=\frac{2 }{3 } - \frac{2 }{3r_0^2 } r^2\mathrm{P}_2(\cos\theta) \\
&=\frac{2 }{3 } - \frac{r^2 }{3r_0^2 } \left(3\cos^2\theta-1 \right)
\end{split}
\end{equation}

\end{solution}

\subsubsection{例3}

\begin{example}
    
在均匀电场 $\vec{E}_0 $ 中放一半径为 $a$ 的接地导体球,求球外电势、电场、导体球表面面电荷密度分布。

\end{example}

\begin{solution}
    
以球心 $O$ 为坐标原点,选取 $\vec{E}_0 $ 方向为 $z$ 轴正方向,则电势 $u$ 关于 $z$ 轴轴对称。

球外无自由电荷,于是球外电势分布 $u(\vec{r}) $ 满足拉普拉斯方程:

\begin{equation}
\nabla^2 u(\vec{r}) = 0,\quad r>a
\end{equation}

特别地,这里电势 $u$ 关于 $z$ 轴对称,$u$ 与 $\varphi$ 无关,拉普拉斯方程可简化为:

\begin{equation}
\nabla^2 u(r,\theta)
=0,\quad r>a
\end{equation}

导体球接地,得到一个边界条件:

\begin{equation}
\left. u(r,\theta)\right|_{r=a}
=0
\end{equation}

由电势的叠加原理,实际电势 $u(r,\theta)$ 是导体球面上的感应电荷产生的电势和匀强电场 $\vec{E}_0 $ 导致的电势的代数和。把感应电荷在无穷远处产生的电势设为零,则当 $r\to +\infty$,电势只由匀强电场贡献。设匀强电场单独存在时在坐标原点产生的电势为 $u_0$,则:

\begin{equation}
u_0-u(r,\theta)
=E_0 r\cos\theta,\quad r\to +\infty
\end{equation}

定解问题为:

\begin{equation}
\left\{
\begin{aligned}
&\nabla^2 u(r,\theta)
=0 \\
&\left. u(r,\theta)\right|_{r=a}
=0 \\
&u(r,\theta)
=u_0-E_0r\cos\theta ,\quad r\to +\infty
\end{aligned}
\right.
\end{equation}

套用结论,轴对称问题的拉普拉斯方程在自然边界条件约束下的形式解为:

\begin{equation}
u(r,\theta)
=\sum_{l=0}^{\infty}\left(A_l r^l+B_l r^{-(l+1)} \right)\mathrm{P}_l(\cos\theta)
\end{equation}

考虑边界条件 $\displaystyle{\left. u(r,\theta)\right|_{r\to +\infty}=u_0-E_0r\cos\theta  }$,当 $r\to+\infty$, 有 $r^{-(l+1)}\to 0$,于是:

\begin{equation}
\begin{aligned}
u_0-E_0r\cos\theta
&=\sum_{l=0}^{\infty} A_l r^l \mathrm{P}_l(\cos\theta) \\
&=A_0 +A_1r\cos\theta + \cdots
\end{aligned}
\end{equation}

左右两边都看作关于 $r $ 的多项式,对比系数得:

\begin{equation}
A_0
=u_0,\quad
A_1=-E_0,\quad
A_2=A_3=\cdots=0
\end{equation}

于是形式解可写为:

\begin{equation}
\begin{aligned}
u(r,\theta)
&=\sum_{l=0}^{\infty}\left(A_l r^l+B_l r^{-(l+1)} \right)\mathrm{P}_l(\cos\theta) \\
&=u_0-E_0r\cos\theta+\sum_{l=0}^{\infty}B_l r^{-(l+1)}\mathrm{P}_l(\cos\theta)
\end{aligned}
\end{equation}

再考虑边界条件 $\displaystyle{\left. u(r,\theta)\right|_{r=a}=0 }$,将形式解代入边界条件,得:

\begin{equation}
u_0-E_0a\cos\theta+\sum_{l=0}^{\infty}B_l a^{-(l+1)}\mathrm{P}_l(\cos\theta)
=0
\end{equation}

即:

\begin{equation}
u_0\mathrm{P}_0(\cos\theta) - E_0 a\mathrm{P}_1(\cos\theta)+\sum_{l=0}^{\infty}B_l a^{-(l+1)}\mathrm{P}_l(\cos\theta)
=0
\end{equation}

整理成各阶勒让德多项式的线性叠加的形式:

\begin{equation}
\left(u_0+B_0a^{-1} \right)\mathrm{P}_0(\cos\theta) + \left(-E_0 a + B_1a^{-2} \right)\mathrm{P}_1(\cos\theta) + \sum_{l=2}^{\infty} B_la^{-(l+1)}\mathrm{P}_l(\cos\theta)
=0
\end{equation}

由各阶勒让德多项式的正交性,它们的线性叠加为零,当且仅当所有线性叠加系数为零,即:

\begin{equation}
B_0
=-a u_0,\quad
B_1
=a^3 E_0,\quad
B_2=B_3=\cdots=0
\end{equation}

综上,导体球外电势分布为:

\begin{equation}
\begin{aligned}
u(r,\theta)
&=u_0-E_0r\cos\theta+\sum_{l=0}^{\infty}B_l r^{-(l+1)}\mathrm{P}_l(\cos\theta) \\
&=u_0-E_0r\cos\theta-\frac{u_0 a}{r}+E_0a^3\frac{\cos\theta}{r^2},\quad r\geqslant a
\end{aligned}
\end{equation}

其中,$u_0$ 为匀强电场单独存在时在坐标原点产生的电势。

取 $u_0=0 $,则导体球外电势分布为:

\begin{equation}
\boxed{
u(r,\theta)
=-E_0r\cos\theta+E_0a^3\frac{\cos\theta}{r^2}
},\quad r\geqslant a
\end{equation}

球外电场与电势的关系为:

\begin{equation}
\begin{aligned}
\vec{E}(\vec{r})
&=-\nabla u(\vec{r}) \\
&=-\left[\frac{\partial u }{\partial r } \vec{\mathrm{e}}_r  + \frac{1 }{r } \frac{\partial u }{\partial \theta } \vec{\mathrm{e}}_\theta + \frac{1 }{r\sin\theta } \frac{\partial u }{\partial \varphi } \vec{\mathrm{e}}_\varphi \right] \\
&=E_0\cos\theta\left(1+\frac{2a^3 }{r^3 } \right) \vec{\mathrm{e}}_r  + E_0\sin\theta\left(\frac{a^3 }{r^3 } - 1 \right)\vec{\mathrm{e}}_\theta,\quad r\geqslant a
\end{aligned}
\end{equation}

导体表面电场为:

\begin{equation}
\left. \vec{E}(\vec{r})\right|_{r=a}
=3E_0\cos\theta\vec{\mathrm{e}}_r
\end{equation}

利用电磁场中的高斯定理,导体球表面面电荷密度分布为:

\begin{equation}
\left. \sigma(\vec{r})\right|_{r=a}
=\varepsilon_0 \left. \vec{E}(\vec{r})\right|_{r=a}\cdot \vec{\mathrm{e}}_r
=3\varepsilon_0 E_0\cos\theta 
\end{equation}

\end{solution}

\subsubsection{例4}

\begin{example}
    
半径为 $a$ 的导体球接地,在距球心为 $b$ 的地方放置一点电荷,$b>a$,电荷量为 $q$,求导体球外的电势分布。

\end{example}

\begin{solution}
    
选取 $z $ 轴使得点电荷的位矢为 $b\vec{\mathrm{e}}_z $,则球外电势 $u $ 具有 $z $ 轴对称性,即 $u=u(r,\theta) $.

点电荷会在接地导体球表面激发出感应电荷。根据电势叠加原理,导体球外的电势 $u $ 是感应电荷单独存在时产生的电势 $u_r $ 与 点电荷单独存在时的电势 $u_q $ 之和:

\begin{equation}
u = u_r + u_q.
\end{equation}

考虑点电荷单独存在时在球外产生的电势 $u_q $,由余弦定理,场点 $\vec{r} $ 到点电荷 $q $ 的距离 $r' $ 满足:

\begin{equation}
r'
=\sqrt{b^2+r^2-2br\cos\theta}.
\end{equation}

点电荷 $q $ 在 $\vec{r} $ 处产生的电势 $u_q $ 满足:

\begin{equation}
\begin{aligned}
u_q
&=\frac{1 }{4\pi \varepsilon_0 } \frac{q }{r' } \\
&=\frac{1 }{4\pi\varepsilon_0 } \frac{q }{\sqrt{b^2+r^2-2br\cos\theta} } \\
&=\frac{1 }{4\pi\varepsilon_0 b} \frac{q }{\sqrt{1+\left(r/b \right)^2 - 2 \left(r/b \right)\cos\theta} } ,\quad r>a
\end{aligned}
\end{equation}

根据勒让德多项式的母函数的相关知识,

\begin{equation}
\frac{1 }{\sqrt{1+\left(r/b \right)^2 - 2 \left(r/b \right)\cos\theta} }
=\left\{
\begin{aligned}
&\sum_{l=0}^{\infty} \mathrm{P}_l(\cos\theta)\left(\frac{r }{b }  \right)^l&,~~r/b<1,~~r<b \\
&\sum_{l=0}^{\infty} \mathrm{P}_l(\cos\theta)\left(\frac{r }{b }  \right)^{-(l+1)}&,~~r/b>1,~~r>b
\end{aligned}
\right.
\end{equation}

因此点电荷产生的电势分布 $u_q $ 可展为:

\begin{equation}
\boxed{
\begin{aligned}
u_q
&=\frac{1 }{4\pi\varepsilon_0 } \frac{q }{\sqrt{b^2+r^2-2br\cos\theta} } \\
&=\frac{q }{4\pi\varepsilon_0 b } \frac{1 }{\sqrt{1+\left(r/b \right)^2 - 2 \left(r/b \right)\cos\theta} } \\
&=\left\{
\begin{aligned}
&\frac{q }{4\pi\varepsilon_0 b }\sum_{l=0}^{\infty} \mathrm{P}_l(\cos\theta)\left(\frac{r }{b }  \right)^l&&,a<r<b \\
&\frac{q }{4\pi\varepsilon_0 b }\sum_{l=0}^{\infty} \mathrm{P}_l(\cos\theta)\left(\frac{r }{b }  \right)^{-(l+1)}&&,r>b
\end{aligned}
\right.
\end{aligned}
}
\end{equation}

再考虑感应电荷单独存在时在球外产生的电势 $u_r $,此时球外没有电荷,因此球外的电势分布 $u_r $ 满足拉普拉斯方程:

\begin{equation}
\nabla^2 u_r(r,\theta)
=0,\quad r>a.
\end{equation}

套用结论,轴对称问题的拉普拉斯方程在自然边界条件约束下的形式解为:

\begin{equation}
u_r(r,\theta)
=\sum_{l=0}^{\infty}\left(A_l r^l+B_l r^{-(l+1)} \right)\mathrm{P}_l(\cos\theta)
\end{equation}

在无穷远处,电势 $u_r $ 应趋于零:

\begin{equation}
\left. u_r\right|_{r\to +\infty} = 0,
\end{equation}

可得:

\begin{equation}
A_l = 0,\quad l=0,1,2,\cdots
\end{equation}

因此:

\begin{equation}
u_r(r,\theta)
=\sum_{l=0}^{\infty} B_l r^{-(l+1)}\mathrm{P}_l(\cos\theta)
\end{equation}

考虑点电荷和感应电荷产生的总电势 $u(r,\theta) $,形式上可写为:

\begin{equation}
\begin{aligned}
u(r,\theta)
&=u_q(r,\theta) + u_r(r,\theta) \\
&=\left\{
\begin{aligned}
&\frac{q }{4\pi\varepsilon_0 b }\sum_{l=0}^{\infty} \mathrm{P}_l(\cos\theta)\left(\frac{r }{b }  \right)^l + \sum_{l=0}^{\infty} B_l r^{-(l+1)}\mathrm{P}_l(\cos\theta) &&,a<r<b \\
&\frac{q }{4\pi\varepsilon_0 b }\sum_{l=0}^{\infty} \mathrm{P}_l(\cos\theta)\left(\frac{r }{b }  \right)^{-(l+1)} + \sum_{l=0}^{\infty} B_l r^{-(l+1)}\mathrm{P}_l(\cos\theta) &&,r>b
\end{aligned}
\right.
\end{aligned}
\end{equation}

导体球接地给出边界条件:

\begin{equation}
\left. u(r,\theta)\right|_{r=a}
=0
\end{equation}

即:

\begin{equation}
\frac{q }{4\pi\varepsilon_0 b }\sum_{l=0}^{\infty} \mathrm{P}_l(\cos\theta)\left(\frac{a }{b }  \right)^l + \sum_{l=0}^{\infty} B_l a^{-(l+1)}\mathrm{P}_l(\cos\theta)
=0
\end{equation}

整理成以 $\cos\theta $ 为自变量的各阶勒让德多项式 $\mathrm{P}_l(\cos\theta) $ 的线性叠加的形式:

\begin{equation}
\sum_{l=0}^{\infty} \left(\frac{q a^l}{4\pi\varepsilon_0 b^{l+1} } + B_l a^{-(l+1)} \right) \mathrm{P}_l(\cos\theta)
=0
\end{equation}

由各阶勒让德多项式的正交性,可得:

\begin{equation}
\frac{q a^l}{4\pi\varepsilon_0 b^{l+1} } + B_l a^{-(l+1)}
=0,\quad l=0,1,2,\cdots
\end{equation}

解得:

\begin{equation}
B_l
=-\frac{q a^{2l+1}}{4\pi\varepsilon_0 b^{l+1} },\quad l=0,1,2,\cdots
\end{equation}

因此:

\begin{equation}
\begin{aligned}
u_r(r,\theta)
&=\sum_{l=0}^{\infty} B_l r^{-(l+1)}\mathrm{P}_l(\cos\theta) \\
&=-\frac{q }{4\pi\varepsilon_0 } \sum_{l=0}^{\infty} \frac{a^{2l+1} }{b^{l+1} }  r^{-(l+1)} \mathrm{P}_l(\cos\theta) \\
&=-\frac{q}{4\pi\varepsilon_0 a} \sum_{l=0}^{\infty} \left(\frac{b r}{a^2 } \right)^{-(l+1)}\mathrm{P}_l(\cos\theta)
\end{aligned}
\end{equation}

注意到,$r>a,b>a $,于是有:

\begin{equation}
\frac{b r }{a^2 } > 1,\quad \frac{1 }{\sqrt{1+\left(br/a^2 \right)^2 - 2\left(br/a^2 \right)\cos\theta} } 
=\sum_{l=0}^{\infty} \mathrm{P}_l(\cos\theta) \left(\frac{br }{a^2 }  \right)^{-(l+1)}
\end{equation}

因此,感应电荷在导体球外产生的电势 $u_r(r,\theta) $ 实际上可写为:

\begin{equation}
\boxed{
\begin{aligned}
u_r(r,\theta)
&=-\frac{q}{4\pi\varepsilon_0 a} \sum_{l=0}^{\infty} \left(\frac{b r}{a^2 } \right)^{-(l+1)}\mathrm{P}_l(\cos\theta) \\
&=-\frac{q}{4\pi\varepsilon_0 a} \frac{1 }{\sqrt{1+\left(br/a^2 \right)^2 - 2\left(br/a^2 \right)\cos\theta} } ,\quad r>a
\end{aligned}
}
\end{equation}

最终得到导体球外的电势分布 $u(r,\theta) $:

\begin{equation}
\boxed{
\begin{aligned}
u(r,\theta)
&=u_q(r,\theta) + u_r(r,\theta) \\
&=\frac{1 }{4\pi\varepsilon_0 } \frac{q }{\sqrt{b^2+r^2-2br\cos\theta} } - \frac{q}{4\pi\varepsilon_0 a} \frac{1 }{\sqrt{1+\left(br/a^2 \right)^2 - 2\left(br/a^2 \right)\cos\theta} } \\
&=\frac{1 }{4\pi\varepsilon_0 } \frac{q }{\sqrt{b^2+r^2-2br\cos\theta} } + \frac{1}{4\pi\varepsilon_0 } \frac{-aq/b }{\sqrt{\left(a^2/b \right)^2 + r^2 - 2\left(a^2/b \right)r\cos\theta} } \\
\end{aligned}
}
\end{equation}

可以看到,感应电荷在导体球外产生的电势与一个处于 $z $ 轴正半轴距球心 $b'=a^2/b $ 处电荷量为 $Q'=-aq/b $ 的点电荷相同。

\end{solution}

