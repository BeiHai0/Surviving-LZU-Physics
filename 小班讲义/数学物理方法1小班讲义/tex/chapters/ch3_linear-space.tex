\chapter{线性空间}

\section{Hilbert空间}

\subsection{内积空间的定义}

设 $L $ 是一个域 $\mathbb{F} $ 上的线性空间。在 $L $ 上定义一个映射

\begin{equation}
\Braket{\cdot , \cdot } : L\times  L\to \mathbb{F},
\end{equation}

若这个映射满足以下三个条件

(1)共轭对称:

\begin{equation}
\forall \psi,\phi\in L,\quad
\Braket{\psi,\phi } = \Braket{\phi,\psi }^*,
\end{equation}

(2)对第二个元素是线性的,对第一个元素是反线性的:

\begin{equation}
\forall a\in \mathbb{F},\quad \psi,\phi\in L,\quad
\Braket{\psi , a\phi } = a\Braket{\psi , \phi },\quad
\Braket{a \psi , \phi } = a^* \Braket{\psi , \phi },
\end{equation}

(3)非负性:

\begin{equation}
\forall \psi \in L,\quad
\Braket{\psi | \psi } \geqslant 0,
\end{equation}

则称 $L $ 是一个内积空间。映射 $\Braket{\cdot , \cdot } $ 称为内积。

\subsection{Hilbert空间的定义}

完备的内积空间称为 Hilbert 空间。一般用 $\mathcal{H} $ 表示希尔伯特空间。

在量子力学中,一般用 $\Ket{\psi }\in \mathcal{H} $ 表示 Hilbert 空间中的向量。

\section{线性空间上的各种算符}

\subsection{算符的定义}

线性空间 $L $ 上的算符 $O $ 是一个从 $L $ 到 $L $ 的映射

\begin{equation}
O : L \to L.
\end{equation}

因此有

\begin{equation}
\forall \psi \in L,\quad
O \psi \in L.
\end{equation}

\subsection{算符之间的运算}

\subsubsection{算符加法}

\begin{equation}
(A+B) \psi = A \psi + B \psi.
\end{equation}

\subsubsection{算符乘法}

\begin{equation}
(AB) \psi = A(B \psi).
\end{equation}

\subsubsection{算符的对易括号}

\begin{equation}
\left[A , B \right]
\equiv AB - .
\end{equation}

\subsection{对算符的运算}

\subsubsection{线性算符}

若算符 $O $ 满足

\begin{equation}
\forall \psi,\phi \in L,\quad\forall a,b\in \mathbb{F},\quad
O(a \psi + b \phi)
=a O \psi + b O \phi,
\end{equation}

则称 $O $ 为线性算符。

\subsubsection{线性算符的转置}

设 $O $ 是 $L $ 上的线性算符,则 $O $ 的转置 $O^\mathrm{T} $ 由下式定义

\begin{equation}
\forall \psi,\phi \in L,\quad
\Braket{\phi , O^\mathrm{T} \psi }
=\Braket{\psi , O \phi }.
\end{equation}

可以证明

\begin{equation}
\left(A B \right)^\mathrm{T}
=B^\mathrm{T} A^\mathrm{T}.
\end{equation}

\subsubsection{线性算符的复共轭}

设 $O $ 是 $L $ 上的线性算符,则 $O $ 的复共轭算符,记为 $O^* $,由下式定义

\begin{equation}
\forall \psi,\phi\in L,\quad
\Braket{\phi , O^* \psi }
=\Braket{\phi , O \psi }^*.
\end{equation}

\subsubsection{线性算符的伴随算符}

设 $O $ 是 $L $ 上的线性算符,则 $O $ 的伴随算符,记为 $O^\dag $,由下式定义

\begin{equation}
\forall \psi,\phi\in L,\quad
\Braket{\phi , O^\dag \psi }
=\Braket{O \phi , \psi }.
\end{equation}

注意到

\begin{equation}
\Braket{O \phi , \psi }
=\left(\Braket{O \phi , \psi }^* \right)^*
=\Braket{\psi , O \phi }^*
=\Braket{\phi , O^\mathrm{T} \psi }^*
=\Braket{\phi , \left(O^\mathrm{T} \right)^* \psi },
\end{equation}

即

\begin{equation}
\Braket{\phi , O^\dag \psi }
=\Braket{\phi , \left(O^\mathrm{T} \right)^* \psi },
\end{equation}

对比得

\begin{equation}
O^\dag = \left(O^\mathrm{T} \right)^*.
\end{equation}

可以证明

\begin{equation}
\left(A B \right)^\dag
=B^\dag A^\dag.
\end{equation}

\subsection{线性空间上的一些特殊算符}

\subsubsection{对称算符与反对称算符}

对称算符:若线性算符 $O $ 满足

\begin{equation}
O^\mathrm{T} = O,
\end{equation}

则称 $O $ 为对称算符。

反对称算符:若线性算符 $O $ 满足

\begin{equation}
O^\mathrm{T} = -O,
\end{equation}

则称 $O $ 为反对称算符。

\subsubsection{自伴算符(厄米算符)}

若线性算符 $O $ 满足

\begin{equation}
O^\dag = O,
\end{equation}

则称 $O $ 为自伴算符或厄米算符。

\subsubsection{幺正算符}

对于线性算符 $U $,若其满足

\begin{equation}
U^\dag U = U U^\dag = I,
\end{equation}

则称 $U $ 为幺正算符。

\section{线性算符的本征值和本征向量}

形如

\begin{equation}
A \psi = \lambda \psi,\quad \lambda \in \mathbb{F}, \psi \in L,
\end{equation}

的方程称为线性算符 $A $ 的本征方程。$\lambda $ 称为 $A $ 的本征值,$\psi $ 称为 $A $ 的本征向量。

若某个本征值 $\lambda_i $ 对应着 $n $ 个线性独立的 $\psi_{i j},j=1,2,\cdots,n $,则称本征值 $\lambda_i $ 是 $n $ 重简并的。

\section{一些定理}

\begin{theorem}
厄米算符本征值为实数。
\end{theorem}

\begin{proof}

设 $A $ 是厄米算符,本征方程

\begin{equation}
A \psi = \lambda \psi,\quad A^\dag = A.
\end{equation}

一方面

\begin{equation}
\Braket{\psi , A \psi }
=\Braket{\psi , \lambda \psi }
=\lambda \Braket{\psi , \psi },
\end{equation}

另一方面

\begin{equation}
\Braket{\psi , A \psi }
=\Braket{\psi , A^\dag \psi }
=\Braket{A \psi , \psi }
=\Braket{\lambda \psi , \psi }
=\lambda^*\Braket{\psi , \psi },
\end{equation}

对比得

\begin{equation}
\lambda^* = \lambda.
\end{equation}

\end{proof}

\begin{theorem}
属于厄米算符不同本征值的本征向量是正交的。
\end{theorem}

\begin{proof}
    
设厄米算符 $A $ 的本征向量 $\psi_1,\psi_2 $ 分别对应本征值 $\lambda_1,\lambda_2,\lambda_1\ne\lambda_2 $,则有

\begin{equation}
A \psi_1 = \lambda_1 \psi_1,\quad
A \psi_2 = \lambda_2 \psi_2.
\end{equation}

一方面

\begin{equation}
\Braket{\psi_2 , A \psi_1}
=\Braket{\psi_2 , \lambda_1 \psi_1}
=\lambda_1 \Braket{\psi_2 , \psi_1},
\end{equation}

另一方面

\begin{equation}
\Braket{\psi_2 , A \psi_1}
=\Braket{\psi_2 , A^\dag \psi_1}
=\Braket{A \psi_2 , \psi_1}
=\Braket{\lambda_2 \psi_2 , \psi_1}
=\lambda_2^* \Braket{\psi_2 , \psi_1}
=\lambda_2 \Braket{\psi_2 , \psi_1},
\end{equation}

作差得

\begin{equation}
\left(\lambda_1 - \lambda_2 \right) \Braket{\psi_2 , \psi_1}
=0.
\end{equation}

由于 $\lambda_1 \ne \lambda_2 $,于是

\begin{equation}
\Braket{\psi_2 , \psi_1} = 0.
\end{equation}

\end{proof}

\begin{theorem}
若 $U $ 为幺正算符,则有

\begin{equation}
\forall \psi,\phi\in L,\quad
\Braket{U \psi , U\phi }
=\Braket{\psi , \phi } .
\end{equation}
\end{theorem}

\begin{proof}

\begin{equation}
\Braket{U \psi , U\phi }
=\Braket{\psi , U^\dag U\phi }
=\Braket{\psi , I\phi }
=\Braket{\psi , \phi }.
\end{equation}

\end{proof}

\begin{theorem}
复内积空间 $L $ 中的线性算符 $A $ 为厄米算符的充要条件是 $\forall \phi\in L, \Braket{\phi , A \phi}\in \mathbb{R} .$
\end{theorem}

\begin{proof}

若 $A $ 为厄米算符,即 $A^\dag=A $ 则

\begin{equation}
\Braket{\phi , A \phi}
=\Braket{\phi , A^\dag \phi}
=\Braket{A \phi , \phi}
=\Braket{\phi , A \phi}^*.
\end{equation}

前后对比得

\begin{equation}
\Braket{\phi , A \phi}\in \mathbb{R}.
\end{equation}

把上面过程反过来就能证明必要性。

\end{proof}

\section{例题}

\subsection{纯态下可观测量期望值}

\begin{example}

假设体系处在纯态 $\Ket{\psi } $,若对体系的可观测物理量 $O $ 进行测量,证明测量期望值 $\Braket{O} $ 可表达为

\begin{equation}
\Braket{O}
=\Braket{\psi | O | \psi}.
\end{equation}

\end{example}

\begin{proof}

考虑本征值离散情况,本征方程为

\begin{equation}
O \Ket{n } = o_n \Ket{n }.
\end{equation}

利用正交性关系和完备性关系

\begin{equation}
\Braket{n | n' } = \delta_{n,n'},\quad
\sum_n \Ket{n }\Bra{n} = I,
\end{equation}

有

\begin{equation}
\begin{split}
O
&=I O I \\
&=\left(\sum_n \Ket{n }\Bra{n } \right) O \left(\sum_{n'} \Ket{n' }\Bra{n' } \right) \\
&=\sum_{n,n'} \Braket{n | O | n' } \Ket{n }\Bra{n' } \\
&=\sum_{n,n'} \Braket{n | o_{n'} | n' } \Ket{n }\Bra{n' } \\
&=\sum_{n,n'} o_{n'} \Braket{n | n' } \Ket{n }\Bra{n' } \\
&=\sum_{n,n'} o_{n'} \delta_{n,n'} \Ket{n }\Bra{n' } \\
&=\sum_{n} o_{n} \Ket{n }\Bra{n }.
\end{split}
\end{equation}

量子力学告诉我们,在 $\Ket{\psi } $ 态下对可观测量 $O $ 进行测量,测得 $o_n $ 的概率为 $\left|\Braket{n | \psi} \right|^2 $,于是期望值

\begin{equation}
\begin{split}
\Braket{O}
&\equiv \sum_n o_n \left|\Braket{n | \psi} \right|^2 \\
&=\sum_n o_n \Braket{n | \psi}^* \Braket{n | \psi} \\
&=\sum_n o_n \Braket{\psi | n} \Braket{n | \psi} \\
&=\Bra{\psi} \left(\sum_n o_n \ket{n} \bra{n} \right) \Ket{\psi } \\
&=\Braket{\psi | O | \psi}.
\end{split}
\end{equation}

\end{proof}

\subsection{混合态下可观测量期望值}

\begin{example}

已知若体系以 $p_i $ 的概率处于纯态 $\Ket{\psi_i } $,则体系的状态可用密度矩阵

\begin{equation}
\rho
\equiv \sum_i p_i \Ket{\psi_i }\Bra{\psi_i},
\end{equation}

来描述。证明在上述状态下对可观测量 $O $ 进行测量,测量期望值 $\Braket{O} $ 可写为

\begin{equation}
\Braket{O}
=\mathrm{Tr}\left(\rho O \right),
\end{equation}

其中,求迹操作 $\mathrm{Tr}(\cdot) $ 的定义为:设 $\left\{\Ket{j } \right\} $ 是任意一组正交完备基,则

\begin{equation}
\mathrm{Tr}\left(O \right)
\equiv \sum_{j} \Braket{j | O | j}.
\end{equation}

\end{example}

\begin{proof}

考虑本征值离散情况,本征方程

\begin{equation}
O \Ket{n } = o_n \Ket{n }.
\end{equation}

\begin{equation}
\begin{split}
\Braket{O}
&\equiv \sum_i p_i \sum_n o_n \left|\Braket{n | \psi_i } \right|^2 \\
&=\sum_i p_i \sum_n o_n \Braket{n | \psi_i } \Braket{n | \psi_i }^* \\
&=\sum_i p_i \sum_n o_n \Braket{n | \psi_i } \Braket{\psi_i | n } \\
&=\sum_i \sum_n o_n p_i \Braket{n | \psi_i } \Braket{\psi_i | n } \\
&=\sum_n \sum_i o_n p_i \Braket{n | \psi_i } \Braket{\psi_i | n } \\
&=\sum_n o_n \Bra{n} \left(\sum_i p_i \Ket{\psi_i }\Bra{\psi_i} \right) \Ket{n } \\
&=\sum_n o_n \Bra{n} \rho \Ket{n } \\
&=\sum_n \Bra{n} \rho o_n \Ket{n } \\
&=\sum_n \Bra{n} \rho O \Ket{n } \\
&=\mathrm{Tr}\left(\rho O \right).
\end{split}
\end{equation}

\end{proof}

\subsection{不确定性关系}

\begin{example}

利用 Schwarz 不等式

\begin{equation}
\forall \Ket{\alpha},\Ket{\beta} \in \mathcal{H},\quad
\Braket{\alpha | \alpha} \Braket{\beta | \beta} \geqslant \left|\Braket{\alpha | \beta} \right|^2,
\end{equation}

以及不等式

\begin{equation}
\forall z \in \mathbb{C},\quad \left|z \right|^2 \geqslant \left|\mathrm{Im}(z) \right|^2,
\end{equation}

推导不确定性关系

\begin{equation}
\forall \ket{\psi} \in \mathcal{H},\quad
\Delta A \Delta B
\geqslant \frac{1 }{2 } \left|\Braket{\left[A , B \right]} \right|,
\end{equation}

其中 $A,B $ 是厄米算符,

\begin{equation}
\Braket{O} = \Braket{\psi | O | \psi},\quad
\Delta O \equiv \sqrt{\Braket{\left(O - \Braket{O} \right)^2}}.
\end{equation}

\end{example}

\begin{proof}

注意到对于厄米算符 $O $,有

\begin{equation}
\begin{split}
\forall \Ket{\psi}\in \mathcal{H},\quad
\Delta O
&\equiv \sqrt{\Braket{\left(O - \Braket{O} \right)^2}} \\
&=\sqrt{\Braket{ \left(O - \Braket{O} \right) \left(O - \Braket{O} \right) }} \\
&=\sqrt{\Braket{ O^2 - 2\Braket{O} O + \Braket{O}^2 }} \\
&=\sqrt{\Braket{O^2} - 2\Braket{O}\Braket{O} + \Braket{O}^2} \\
&=\sqrt{\Braket{O^2} -\Braket{O}^2},
\end{split}
\end{equation}

\begin{equation}
\begin{split}
\Braket{\psi | \left(O - \Braket{O} \right)^\dag \left(O - \Braket{O} \right) | \psi }
&=\Braket{\psi | \left(O^\dag - \Braket{O}^* \right) \left(O - \Braket{O} \right) | \psi } \\
&=\Braket{\psi | \left(O - \Braket{O} \right) \left(O - \Braket{O} \right) | \psi } \\
&=\Braket{\psi | \left(O - \Braket{O} \right)^2 | \psi },
\end{split}
\end{equation}

令 $\ket{\alpha} = \left(A - \Braket{A} \right) \ket{\psi},\ket{\beta} = \left(B - \Braket{B} \right) \ket{\psi} $,则

\begin{equation}
\begin{split}
\Braket{\alpha | \alpha }
=\Braket{\psi | \left(A - \Braket{A } \right)^2 | \psi }
=\Braket{\left(A - \Braket{A } \right)^2 }
=\left(\Delta A \right)^2 \\
\Braket{\beta | \beta }
=\Braket{\psi | \left(B - \Braket{B } \right)^2 | \psi }
=\Braket{\left(B - \Braket{B } \right)^2 }
=\left(\Delta B \right)^2,
\end{split}
\end{equation}

\begin{equation}
\begin{split}
\Braket{\alpha | \beta }
&=\Braket{\psi | \left(A^\dag-\Braket{A }^* \right) \left(B-\Braket{B } \right) | \psi } \\
&=\Braket{\psi | \left(A-\Braket{A } \right) \left(B-\Braket{B } \right) | \psi } \\
&=\Braket{\psi | AB - \Braket{A }B - \Braket{B }A + \Braket{A }\Braket{B } | \psi } \\
&=\Braket{AB } - \Braket{A }\Braket{B } - \Braket{B }\Braket{A } + \Braket{A }\Braket{B } \\
&=\Braket{AB } - \Braket{A }\Braket{B },
\end{split}
\end{equation}

\begin{equation}
\Braket{\beta | \alpha }
=\Braket{BA } - \Braket{B }\Braket{A },
\end{equation}

\begin{equation}
\begin{split}
\left|\Braket{\alpha | \beta } \right|^2
&\geqslant \left|\mathrm{Im}\Braket{\alpha | \beta } \right|^2 \\
&=\left|\frac{1 }{2\mathrm{i} } \left(\Braket{\alpha | \beta } - \Braket{\alpha | \beta }^* \right) \right|^2 \\
&=\left|\frac{1 }{2\mathrm{i} } \left(\Braket{\alpha | \beta } - \Braket{\beta | \alpha } \right) \right|^2 \\
&=\frac{1 }{4 } \left|\Braket{AB } - \Braket{BA } \right|^2 \\
&=\frac{1 }{4 } \left|\Braket{AB-BA } \right|^2 \\
&=\frac{1 }{4 } \left|\Braket{\left[A , B \right] } \right|^2,
\end{split}
\end{equation}

代入 Schwarz 不等式 $\Braket{\alpha | \alpha} \Braket{\beta | \beta} \geqslant \left|\Braket{\alpha | \beta} \right|^2\geqslant \left|\mathrm{Im}\Braket{\alpha | \beta } \right|^2 $,有

\begin{equation}
\left(\Delta A \right)^2 \left(\Delta B \right)^2
\geqslant \frac{1 }{4 } \left|\Braket{\left[A , B \right] } \right|^2,
\end{equation}

即

\begin{equation}
\Delta A \Delta B
\geqslant \frac{1 }{2 } \left|\Braket{\left[A , B \right]} \right|.
\end{equation}

\end{proof}