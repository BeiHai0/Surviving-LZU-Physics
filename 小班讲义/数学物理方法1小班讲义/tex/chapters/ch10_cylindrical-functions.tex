\chapter{柱函数}

\section{贝塞尔函数}

\subsection{贝塞尔函数作为贝塞尔方程的解}

\subsubsection{非整数阶贝塞尔函数}

我们知道,柱坐标系下的亥姆霍兹方程

\begin{equation}
\nabla^2 u(\rho,\varphi,z) + k^2 u(\rho,\varphi,z) = 0,
\end{equation}

在分离变量 $u(\rho,\varphi,z) = R(\rho)\Phi(\varphi)Z(z) $ 以及变量代换 $x=\sqrt{k^2+\lambda } \rho,y(x)=R(\rho) $ 后得到

\begin{equation}
\Phi''(\varphi) + \nu^2 \Phi(\varphi) = 0,
\end{equation}

\begin{equation}
Z''(z) - \lambda Z(z) = 0,
\end{equation}

\begin{equation}
\frac{\mathrm{d}^2 y(x) }{\mathrm{d}x^2 } + \frac{1 }{x } \frac{\mathrm{d}y(x) }{\mathrm{d}x } + \left(1-\frac{\nu^2 }{x^2 }  \right)y(x)
=0.
\end{equation}

当 $\nu $ \textbf{不是正整数}时,利用级数法可以证明,径向部分 $R(\rho)=y(x) $ 所要满足的 $\nu $ 阶贝塞尔方程

\begin{equation}
\frac{\mathrm{d}^2 y(x) }{\mathrm{d}x^2 } + \frac{1 }{x } \frac{\mathrm{d}y(x) }{\mathrm{d}x } + \left(1-\frac{\nu^2 }{x^2 }  \right)y(x)
=0
\end{equation}

的通解为

\begin{equation}
y(x)
=C_1 \mathrm{J}_\nu(x) + C_2 \mathrm{J}_{-\nu}(x),
\end{equation}

其中,$\nu $ 阶贝塞尔函数(或称为第一类贝塞尔函数) $\mathrm{J}_\nu(x) $ 由下式定义:

\begin{equation}
\mathrm{J}_\nu(x)
\equiv \sum_{k=0}^{\infty} \frac{\left(-1 \right)^k }{k!\Gamma\left(k+\nu+1 \right) }\left(\frac{x }{2 }  \right)^{2k+\nu}.
\end{equation}

当 $\nu $ 不等于正整数时,$\mathrm{J}_\nu(x) $ 与 $\mathrm{J}_{-\nu}(x) $ 线性独立。此时用 $\mathrm{J}_\nu(x) $ 与 $\mathrm{J}_{-\nu}(x) $ 进行如下的线性组合就得到诺伊曼函数(或称为第二类贝塞尔函数) $\mathrm{N}_\nu(x) $:

\begin{equation}
\mathrm{N}_\nu(x)
\equiv \frac{\cos\left(\nu\pi \right)\mathrm{J}_\nu(x)-\mathrm{J}_{-\nu}(x) }{\sin(\nu\pi) }.
\end{equation}

特别地,当 $\nu=m\in \mathbb{Z} $ 为整数时,诺伊曼函数 $\mathrm{N}_m(x) $ 由下面的极限给出(用洛必达法则计算):

\begin{equation}
m\in \mathbb{Z},\quad
\mathrm{N}_m(x)
\equiv \lim_{\nu\to m} \mathrm{N}_\nu(x)
\end{equation}

\subsubsection{整数阶贝塞尔函数}

角度部分 $\Phi(\varphi) $ 所满足的方程

\begin{equation}
\Phi''(\varphi) + \nu^2 \Phi(\varphi) = 0,
\end{equation}

的通解为:

\begin{equation}
\Phi(\varphi) = A_1 \mathrm{e}^{\mathrm{i}\nu\varphi} + A_2 \mathrm{e}^{\mathrm{-\mathrm{i}\nu\varphi}}
\end{equation}

如果我们不限制 $\varphi $ 的取值,转而用周期性边界条件

\begin{equation}
\Phi(\varphi+2\pi) = \Phi(\varphi)
\end{equation}

来保证场量的唯一性,那么有

\begin{equation}
2\pi \nu = 2\pi m,\quad m\in \mathbb{Z}
\end{equation}

也即此时 $\nu $ 取整数。

可以证明,当 $\nu=m,m\in \mathbb{Z} $ 时,$\mathrm{J}_m(x) $ 与 $\mathrm{J}_{-m}(x) $ 线性相关。此时,$m $ 阶贝塞尔方程

\begin{equation}
\frac{\mathrm{d}^2 y(x) }{\mathrm{d}x^2 } + \frac{1 }{x } \frac{\mathrm{d}y(x) }{\mathrm{d}x } + \left(1-\frac{m^2 }{x^2 }  \right)y(x)
=0,\quad m\in \mathbb{Z}
\end{equation}

的的通解为:

\begin{equation}
y(x)
=C_1 \mathrm{J}_m(x) + C_2 \mathrm{N}_m(x)
\end{equation}

\section{贝塞尔函数的性质}

\subsection{整数阶贝塞尔函数的简单性质}

\begin{equation}
\mathrm{J}_m(-x) = \left(-1 \right)^m\mathrm{J}_m(x)
\end{equation}

\begin{equation}
\mathrm{J}_0(0) = 1
\end{equation}

\begin{equation}
\mathrm{J}_m(0) = 0,\quad m=1,2,\cdots
\end{equation}

\begin{equation}
\mathrm{N}_{-m}(x) = \left(-1 \right)^m \mathrm{N}_m(x)
\end{equation}

\subsection{贝塞尔函数的递推关系}

\begin{equation}
\mathrm{J}_{\nu-1}(x) + \mathrm{J}_{\nu+1}(x) = \frac{2\nu }{x } \mathrm{J}_\nu(x)
\end{equation}

\begin{equation}
\mathrm{J}_{\nu-1}(x) - \mathrm{J}_{\nu+1}(x) = 2\mathrm{J}_\nu'(x)
\end{equation}

\begin{equation}
\mathrm{N}_{\nu-1}(x) + \mathrm{N}_{\nu+1}(x) = \frac{2\nu }{x } \mathrm{N}_\nu(x)
\end{equation}

\begin{equation}
\mathrm{N}_{\nu-1}(x) - \mathrm{N}_{\nu+1}(x) = 2\mathrm{N}_\nu'(x)
\end{equation}

\begin{equation}
\mathrm{J}_0'(x) = -\mathrm{J}_1(x)
\end{equation}

\section{柱函数}

若函数 $y_\nu(x) $ 满足:

\begin{equation}
\left\{
\begin{aligned}
&y_{\nu-1}(x)+y_{\nu+1}(x)
=\frac{2\nu}{x} y_\nu(x) \\
&y_{\nu-1}(x)-y_{\nu+1}(x)
=2y_\nu'(x)
\end{aligned}
\right.
\end{equation}

或满足与上两式等价的关系:

\begin{equation}
\left\{
\begin{aligned}
&\frac{\mathrm{d} }{\mathrm{d}x }\left[x^\nu y_\nu(x) \right]
=x^\nu y_{\nu-1}(x) \\
&\frac{\mathrm{d} }{\mathrm{d}x }\left[x^{-\nu}y_{\nu+1}(x) \right]=x^{-\nu} y_{\nu+1}(x)
\end{aligned}
\right.
\end{equation}

则这类函数统称为柱函数。

柱函数必定满足贝塞尔方程。

\section{例题}

\subsection{边缘固定圆形膜本征振动}

\begin{example}
    
求边缘固定半径为 $b $ 的圆形膜的本征振动频率及本征振动模式。

\end{example}

\begin{solution}
    
以圆形膜的圆心为原点建立极坐标,设 $u(\rho,\varphi,t) $ 是 $t $ 时刻 $\rho,\varphi $ 处质点偏离平衡位置的位移,则 $u(\rho,\varphi,t) $ 满足二维波动方程:

\begin{equation}
u_{tt}(\rho,\varphi,t) - a^2 \nabla^2_{(2)} u(\rho,\varphi,t)
=0,
\end{equation}

其中,$\nabla^2_{(2)} $ 是二维拉普拉斯算子:

\begin{equation}
\nabla^2_{(2)}
\equiv \frac{1}{\rho}\frac{\partial}{\partial \rho}\left(\rho\frac{\partial }{\partial \rho}\right)+\frac{1}{\rho^2}\frac{\partial^2}{\partial \varphi^2}.
\end{equation}

设 $u(\rho,\varphi,t) $ 可分离变量为:

\begin{equation}
u(\rho,\varphi,t)
=U(\rho,\varphi) T(t),
\end{equation}

代入二维波动方程可得:

\begin{equation}
U(\rho,\varphi) T''(t) - a^2 T(t)\left[\frac{1}{\rho}\frac{\partial}{\partial \rho}\left(\rho\frac{\partial }{\partial \rho}\right)+\frac{1}{\rho^2}\frac{\partial^2}{\partial \varphi^2} \right]U(\rho,\varphi)
=0.
\end{equation}

上式两边同时除以 $U(\rho,\varphi) T(t) $,得:

\begin{equation}
\frac{T''(t) }{T(t) } - \frac{a^2 }{U(\rho,\varphi) } \left[\frac{1}{\rho}\frac{\partial}{\partial \rho}\left(\rho\frac{\partial }{\partial \rho}\right)+\frac{1}{\rho^2}\frac{\partial^2}{\partial \varphi^2} \right]U(\rho,\varphi)
=0.
\end{equation}

观察可知:

\begin{equation}
\frac{T''(t) }{T(t) }
=-\omega^2,\quad 
\frac{a^2 }{U(\rho,\varphi) } \left[\frac{1}{\rho}\frac{\partial}{\partial \rho}\left(\rho\frac{\partial }{\partial \rho}\right)+\frac{1}{\rho^2}\frac{\partial^2}{\partial \varphi^2} \right]U(\rho,\varphi)
=-\omega^2
\end{equation}

由于要求本征振动频率和本征振动模式,因此只需要关注空间部分 $U(\rho,\varphi) $ 满足的方程和边界条件。

对上式空间部分 $U(\rho,\varphi) $ 满足的方程等号两边同乘 $\displaystyle{\frac{U(\rho,\varphi) }{a^2 }  }$ 并移项,得:

\begin{equation}
\frac{1}{\rho}\frac{\partial}{\partial \rho}\left(\rho\frac{\partial U(\rho,\varphi)}{\partial \rho}\right)+\frac{1}{\rho^2}\frac{\partial^2 U(\rho,\varphi)}{\partial \varphi^2} + \frac{\omega^2 }{a^2 } U(\rho,\varphi)
=0.
\end{equation}

令:

\begin{equation}
k
\equiv \frac{\omega }{a } ,\quad k^2=\frac{\omega^2 }{a^2 }
\end{equation}

则 $U(\rho,\varphi) $ 满足的方程为:

\begin{equation}
\frac{1}{\rho}\frac{\partial}{\partial \rho}\left(\rho\frac{\partial U(\rho,\varphi)}{\partial \rho}\right)+\frac{1}{\rho^2}\frac{\partial^2 U(\rho,\varphi)}{\partial \varphi^2} + k^2 U(\rho,\varphi)
=0
\end{equation}

由于圆形膜边界固定,因此得到一个边界条件:

\begin{equation}
\left. U(\rho,\varphi)\right|_{\rho=b}
=0
\end{equation}

且圆心处质点偏离平衡位置的位移应有限,因此得到一个自然边界条件:

\begin{equation}
\left. \left|U(\rho,\varphi)\right|\right|_{\rho=0} 
<+\infty
\end{equation}

再结合 $\varphi $ 作为角度这一物理量应使得 $U(\rho,\varphi) $ 满足周期性边界条件:

\begin{equation}
U(\rho,\varphi+2\pi)
=U(\rho,\varphi)
\end{equation}

综上,空间部分 $U(\rho,\varphi) $ 要满足的所有条件为:

\begin{equation}
\left\{
\begin{aligned}
&\frac{1}{\rho}\frac{\partial}{\partial \rho}\left(\rho\frac{\partial U(\rho,\varphi)}{\partial \rho}\right)+\frac{1}{\rho^2}\frac{\partial^2 U(\rho,\varphi)}{\partial \varphi^2} + k^2 U(\rho,\varphi)
=0 \\
&\left. U(\rho,\varphi)\right|_{\rho=b}
=0 \\
&\left. \left|U(\rho,\varphi)\right|\right|_{\rho=0} 
<+\infty \\
&U(\rho,\varphi+2\pi)
=U(\rho,\varphi)
\end{aligned}
\right.
\end{equation}

设 $U(\rho,\varphi) $ 可分离变量为:

\begin{equation}
U(\rho,\varphi)
=R(\rho)\Phi(\varphi)
\end{equation}

代入空间部分 $U(\rho,\varphi) $ 要满足的方程,得:

\begin{equation}
\frac{\Phi(\varphi) }{\rho } \frac{\mathrm{d} }{\mathrm{d}\rho }\left(\rho \frac{\mathrm{d}R(\rho) }{\mathrm{d}\rho } \right) + \frac{R(\rho) }{\rho^2 } \frac{\mathrm{d}^2 \Phi(\varphi) }{\mathrm{d}\varphi^2 } + k^2 R(\rho)\Phi(\varphi)
=0
\end{equation}

上式等号两边同乘 $\displaystyle{\frac{\rho^2 }{R(\rho)\Phi(\varphi) }  }$,整理得:

\begin{equation}
\frac{1 }{\Phi(\varphi) } \frac{\mathrm{d}^2\Phi(\varphi) }{\mathrm{d}\varphi^2 } + \left[\frac{\rho }{R(\rho) } \frac{\mathrm{d} }{\mathrm{d}\rho }\left(\rho\frac{\mathrm{d}R(\rho) }{\mathrm{d}\rho }  \right) +k^2\rho^2  \right]
=0
\end{equation}

观察可知:

\begin{equation}
\frac{1 }{\Phi(\varphi) } \frac{\mathrm{d}^2\Phi(\varphi) }{\mathrm{d}\varphi^2 } = -m^2,\quad
\frac{\rho }{R(\rho) } \frac{\mathrm{d} }{\mathrm{d}\rho }\left(\rho\frac{\mathrm{d}R(\rho) }{\mathrm{d}\rho }  \right) +k^2\rho^2 - m^2
=0
\end{equation}

因此,角度部分满足方程:

\begin{equation}
\Phi''(\varphi) + m^2\Phi(\varphi)
=0
\end{equation}

周期性边界条件:

\begin{equation}
U(\rho,\varphi+2\pi)
=U(\rho,\varphi)
\Longrightarrow R(\rho)\Phi(\varphi+2\pi)=R(\rho)\Phi(\varphi)
\Longrightarrow \Phi(\varphi+2\pi)=\Phi(\varphi)
\end{equation}

\begin{equation}
\left\{
\begin{aligned}
&\Phi''(\varphi) + m^2\Phi(\varphi)=0 \\
&\Phi(\varphi+2\pi)=\Phi(\varphi)
\end{aligned}
\right.
\end{equation}

从

\begin{equation}
\Phi''(\varphi) + m^2\Phi(\varphi)
=0
\end{equation}

可以解得:

\begin{equation}
\Phi(\varphi)
=A\cos(m\varphi) + B\sin(m\varphi)
\end{equation}

结合周期性边界条件

\begin{equation}
\Phi(\varphi+2\pi)=\Phi(\varphi)
\end{equation}

可得:

\begin{equation}
m=0,1,2,\cdots
\end{equation}

径向部分 $R(\rho) $ 满足:

\begin{equation}
\frac{\rho }{R(\rho) } \frac{\mathrm{d} }{\mathrm{d}\rho }\left(\rho\frac{\mathrm{d}R(\rho) }{\mathrm{d}\rho }  \right) +k^2\rho^2 - m^2
=0
\end{equation}

可以整理成:

\begin{equation}
\frac{1 }{\rho } \frac{\mathrm{d} }{\mathrm{d}\rho }\left(\rho \frac{\mathrm{d}R(\rho) }{\mathrm{d}\rho } \right) + \left(k^2 - \frac{m^2 }{\rho^2 }  \right)R(\rho)
=0
\end{equation}

令 $x=k\rho,\rho=x/k,R(\rho)=y(x) $,则上面可方程化为 $m $ 阶贝塞尔方程:

\begin{equation}
\frac{\mathrm{d}^2 y }{\mathrm{d}x^2 } + \frac{1 }{x } \frac{\mathrm{d}y }{\mathrm{d}x } + \left(1-\frac{m^2 }{x^2 }  \right)y
=0,\quad m\in \mathbb{Z}
\end{equation}

$$
\left. U(\rho,\varphi)\right|_{\rho=b}
=0
\Longrightarrow \left. R(\rho)\Phi(\varphi) \right|_{\rho=b} = 0
\Longrightarrow \left. R(\rho)\right|_{\rho=b} = 0
$$

\begin{equation}
\left. \left|U(\rho,\varphi)\right|\right|_{\rho=0} 
<+\infty
\Longrightarrow \left. \left|R(\rho)\Phi(\varphi) \right|\right|_{\rho=0}<+\infty
\Longrightarrow \left. \left|R(\rho) \right|\right|_{\rho=0}<+\infty
\end{equation}

\begin{equation}
\left\{
\begin{aligned}
&\frac{\mathrm{d}^2 y }{\mathrm{d}x^2 } + \frac{1 }{x } \frac{\mathrm{d}y }{\mathrm{d}x } + \left(1-\frac{m^2 }{x^2 }  \right)y
=0 \\
&R(\rho)=y(x)=y(k\rho)\\
&\left. R(\rho)\right|_{\rho=b} = 0 \\
&\left. \left|R(\rho) \right|\right|_{\rho=0}<+\infty
\end{aligned}
\right.
\end{equation}

对于 $m $ 阶贝塞尔方程

\begin{equation}
\frac{\mathrm{d}^2 y }{\mathrm{d}x^2 } + \frac{1 }{x } \frac{\mathrm{d}y }{\mathrm{d}x } + \left(1-\frac{m^2 }{x^2 }  \right)y
=0,\quad m\in \mathbb{Z}
\end{equation}

其通解为:

\begin{equation}
y^{(m)}(x)
=C_m \mathrm{J}_m(x) + D_m\mathrm{N}_m(x)
\end{equation}

考虑自然边界条件 $\displaystyle{\left. \left|R(\rho) \right|\right|_{\rho=0}<+\infty }$,可得:

\begin{equation}
D_m=0
\end{equation}

因此:

\begin{equation}
y^{(m)}(x)
=C_m\mathrm{J}_m(x)
\end{equation}

由 $R(\rho)=y(x),x=k\rho $ 有

\begin{equation}
R^{(m)}(\rho)
=y^{(m)}(x)
=C_m\mathrm{J}_m(k\rho)
\end{equation}

设 $m $ 阶贝塞尔函数 $\mathrm{J}_m(x) $ 的第 $n $ 个正零点为 $x_n^{(m)} $,即:

\begin{equation}
\mathrm{J}_m\left(x_n^{(m)} \right)
=0,\quad m=0,1,2,\cdots;\quad n=,1,2,\cdots
\end{equation}

结合边界条件 $\displaystyle{\left. R(\rho)\right|_{\rho=b} = 0 }$,即:

\begin{equation}
C_m\mathrm{J}_m\left(k b \right)
=0
\end{equation}

因此 $k $ 的本征值 $k_n^{(m)} $ 为:

\begin{equation}
k_n^{(m)}
=\frac{x_n^{(m)} }{b } ,\quad m=0,1,2,\cdots;\quad n=1,2,\cdots
\end{equation}

相应的本征振动模式 $R_n^{(m)}(\rho) $ 为:

\begin{equation}
R_n^{(m)}(\rho)
=\mathrm{J}_m\left(k_n^{(m)}\rho \right)
=\mathrm{J}_m\left(\frac{x_n^{(m)} }{b } \rho \right),\quad m=0,1,2,\cdots;\quad n=1,2,\cdots
\end{equation}

再根据 $k\equiv \omega/a $,得到 $\omega $ 的本征值,即圆形膜的本征频率 $\omega_n^{(m)} $ 为:

\begin{equation}
\omega_n^{(m)}
=ak_n^{(m)}
=\frac{x_n^{(m)} }{b }\cdot a,\quad m=0,1,2,\cdots;\quad n=1,2,\cdots
\end{equation}

综上所述,边缘固定半径为 $b $ 的圆形膜的本征振动频率 $\omega_n^{(m)} $ 及本征振动模式 $R_n^{(m)}(\rho) $ 为:

\begin{equation}
\boxed{
\omega_n^{(m)} = \frac{x_n^{(m)} }{b }\cdot a
},\quad m=0,1,2,\cdots;\quad n=1,2,\cdots
\end{equation}

\begin{equation}
\boxed{
R_n^{(m)}(\rho)
=\mathrm{J}_m\left(\frac{x_n^{(m)} }{b } \rho \right)
},\quad m=0,1,2,\cdots;\quad n=1,2,\cdots
\end{equation}

其中,$x_n^{(m)} $ 是 $m $ 阶贝塞尔函数 $\mathrm{J}_m(x) $ 的第 $n $ 个正零点。

\end{solution}
