\chapter{拉普拉斯变换}

\section{拉普拉斯变换的定义}

对于定义在实数 $t\in[0,+\infty)$ 上的实函数或复函数 $f(t)$,$f(t)$ 的拉普拉斯变换,记为 $F(p) $,由下式定义:

\begin{equation}
F(p)
\equiv \int_{t=0}^{t=+\infty} f(t)\mathrm{e}^{-p t} \mathrm{d}t,
\end{equation}

其中,$p=s+\mathrm{i}\sigma,s\in \mathbb{R},\sigma\in \mathbb{R},\mathrm{e}^{-pt}$ 称为拉普拉斯变换核,$F(p) $ 称为像函数,也记为:

\begin{equation}
F(p) \risingdotseq f(t),\quad f(t) \fallingdotseq F(p).
\end{equation}

\section{拉普拉斯变换的基本性质}

\subsection{线性定理}

设 $f_1(t)\fallingdotseq F_1(p),f_2(t)\fallingdotseq F_2(p),\alpha_1,\alpha_2\in \mathbb{C}$,则:

\begin{equation}
\alpha_1 f_1(t)+\alpha_2 f_2(t)
\fallingdotseq \alpha_1F_1(p)+\alpha_2F_2(p).
\end{equation}

\subsection{延迟定理}

定义阶跃函数 $H $:

\begin{equation}
H(t)
\equiv\begin{cases}
1&,t>0 \\
0&,t\leqslant 0
\end{cases},
\end{equation}

设 $f(t)\fallingdotseq F_(p),\tau>0 $,则:

\begin{equation}
f(t-\tau)H(t-\tau)\fallingdotseq \mathrm{e}^{-p\tau}F(p).
\end{equation}

\subsection{位移定理}

设 $f(t)\fallingdotseq F(p),\lambda\in\mathbb{C} $,则:

\begin{equation}
\mathrm{e}^{-\lambda t}f(t)
\fallingdotseq F(p+\lambda).
\end{equation}

\subsection{标度变换定理}

设 $f(t)\fallingdotseq F(p),a>0 $,则:

\begin{equation}
f(at)
\fallingdotseq \frac{1}{a}F\left(\frac{p}{a}\right),\quad a>0.
\end{equation}

\subsection{卷积定理}

定义卷积:

\begin{equation}
f_1(t)*f_2(t)
\equiv \int_{\tau=0}^{\tau=t} f_1(\tau)f_2(t-\tau)\mathrm{d}\tau,
\end{equation}

设 $f_1(t)\fallingdotseq F_1(p),f_2(t)\fallingdotseq F_2(p) $,则:

\begin{equation}
f_1(t)*f_2(t)
\fallingdotseq F_1(p)F_2(p).
\end{equation}

\subsection{微分定理}

设 $f(t)\fallingdotseq F(p) $,则:

\begin{equation}
f^{(n)}(t)
\fallingdotseq p^n F(p)-p^{n-1}f^{(0)}(0)-p^{n-2}f^{(1)}(0)-\cdots-p^{1}f^{(n-2)}(0)-p^{0}f^{(n-1)}(0).
\end{equation}

特别地:

\begin{equation}
f^{(1)}(t)
\fallingdotseq p^1 F(p)-p^0 f^{(0)}(0),
\end{equation}

也即:

\begin{equation}
f'(t)
\fallingdotseq p F(p) - f(0).
\end{equation}

\begin{equation}
f^{(2)}(t)
\fallingdotseq p^2 F(p)-p^1f^{(0)}(0)-p^0f^{(1)}(0),
\end{equation}

也即:

\begin{equation}
f''(t)
\fallingdotseq p^2 F(p) - p f(0) - f'(0).
\end{equation}

\section{常用拉普拉斯变换及反演}

\begin{equation}
\begin{split}
\left\{
\begin{aligned}
&\frac{1}{p}
\risingdotseq 1,\quad
\frac{1}{p^2}
\risingdotseq t,\quad
\frac{n!}{p^{n+1}}
\risingdotseq t^n \\[1mm]
&\frac{1}{p-\alpha}
\risingdotseq \mathrm{e}^{\alpha t},\quad
\frac{n!}{(p-n)^{n+1}}
\risingdotseq t^n\mathrm{e}^{\alpha t} \\[1mm]
&\frac{\alpha}{p^2+\alpha^2}
\risingdotseq \sin\alpha t,\quad
\frac{p}{p^2+\alpha^2}
\risingdotseq \cos\alpha t \\[1mm]
&\frac{\alpha}{p^2-\alpha^2}
\risingdotseq \sinh\alpha t,\quad
\frac{p}{p^2-\alpha^2}
\risingdotseq \cosh\alpha t.
\end{aligned}
\right.
\end{split}
\end{equation}

\section{拉普拉斯变换的应用}

\subsection{例题}

\begin{example}

用拉普拉斯变换解下列 $RL $ 串联电路方程,其中 $L,R,E $ 为常数:

\begin{equation}
\left\{
\begin{aligned}
&L\frac{\mathrm{d}i(t) }{\mathrm{d}t } + R i(t) = E, \\
&i(0) = 0.
\end{aligned}
\right.
\end{equation}

\end{example}

\begin{solution}
    
设 $i(t)\fallingdotseq F(p) $,微分定理给出:

\begin{equation}
\frac{\mathrm{d}i(t) }{\mathrm{d}t }
\fallingdotseq p^1 F(p) - p^0 i^{(0)}(0)
=pF(p) - i(0)
=pF(p),
\end{equation}

常用拉普拉斯变换:

\begin{equation}
1\fallingdotseq \frac{1 }{p },\quad \mathrm{Re}~p>0
\end{equation}

对方程 $\displaystyle{L\frac{\mathrm{d}i(t) }{\mathrm{d}t } = +R i(t) = E }$ 两边同时作拉普拉斯变换,得:

\begin{equation}
L p F(p) + R F(p) = \frac{E }{p }.
\end{equation}

可以解出 $F(p) $:

\begin{equation}
\begin{split}
F(p)
&=\frac{E }{Lp^2 + Rp } \\
&=\frac{E }{L }\frac{1 }{p( p + R/L) } \\
&=\frac{E }{L } \frac{L }{R }  \left(\frac{1 }{p } - \frac{1 }{p + R/L }  \right) \\
&=\frac{E }{R } \left(\frac{1 }{p } - \frac{1 }{p - (-R/L)}  \right).
\end{split}
\end{equation}

常用拉普拉斯变换的反演:

\begin{equation}
\frac{1 }{p-\alpha } \risingdotseq \mathrm{e}^{\alpha t},
\end{equation}

于是:

\begin{equation}
\frac{1 }{p } \risingdotseq 1,\quad \frac{1 }{p - (-R/L) } \risingdotseq \mathrm{e}^{-\frac{R }{L } t}.
\end{equation}

对方程 $\displaystyle{F(p) = \frac{E }{R } \left(\frac{1 }{p } - \frac{1 }{p+R/L } \right) }$ 两边同时作拉普拉斯逆变换,得:

\begin{equation}
i(t)
=\frac{E }{R } \left(1-\mathrm{e}^{-\frac{R }{L } t} \right).
\end{equation}

\end{solution}