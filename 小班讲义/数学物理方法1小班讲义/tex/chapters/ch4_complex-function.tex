\chapter{复变函数}

\section{复变函数的概念}

\subsection{复变函数的定义}

复变函数 $f $ 是黎曼面 $\mathbb{C}^\mathrm{R} $ 到复平面 $\mathbb{C} $ 的映射。

\section{解析函数}

\subsection{复变函数的连续性}

设复变函数 $f(z)$ 在 $z_0$ 点及其邻域内有定义。当自变量 $z$ 以任何路径趋于 $z_0$ 时,都有

\begin{equation}
\lim_{z\to z_0} f(z)
=f(z_0),
\end{equation}

则称 $f(z)$ 在 $z_0$ 点连续。

若 $f(z)$ 在区域 $\Omega$ 内的所有点都连续,则称 $f(z)$ 在 $\Omega$ 内连续。

\subsection{复变函数的导数}

设 $z_0 $ 是复变函数 $f(z) $ 定义域 $\Omega $ 内的一点。当 $z$ 以任何路径趋于 $z_0$ 时,即 $\Delta z=z-z_0$ 以任何方式趋于 $0$ 时,若极限

\begin{equation}
\lim_{\Delta z\to 0}\frac{f(z_0+\Delta z)-f(z_0)}{\Delta z}
\end{equation}

存在且唯一,则称 $f(z)$ 在 $z_0$ 点可导,$f(z)$ 在 $z_0$ 点的导数记为 $f'(z_0).$

\subsection{柯西-黎曼条件}

\begin{theorem}

设复变函数 $f(z)=u(x,y)+\mathrm{i}v(x,y)$,若 $f(z)$ 在 $z$ 点可导,则必定有:

\begin{equation}
\frac{\partial u}{\partial x}
=\frac{\partial v}{\partial y},\quad
\frac{\partial u}{\partial y}
=-\frac{\partial v}{\partial x}.
\end{equation}

上面两条等式称为柯西-黎曼条件(C-R条件)。

\end{theorem}

\begin{proof}

设 $z=x+\mathrm{i} y,f(z)=u(x,y)+\mathrm{i}v(x,y)$,则:

\begin{equation}
\lim_{\Delta z\to 0} \frac{f(z+\Delta z)-f(z)}{\Delta z}
=\lim_{\Delta z\to 0}\frac{\Delta u+\mathrm{i}\Delta v}{\Delta x+\mathrm{i}\Delta y}.
\end{equation}

由于 $f(z)$ 在 $z$ 点可导,故极限

\begin{equation}
\lim_{\Delta z\to 0} \frac{f(z+\Delta z)-f(z)}{\Delta z}
\end{equation}

存在且与 $\Delta z$ 趋于 $0$ 的方式无关。

特别地,

(1)令

\begin{equation}
\mathrm{i}\Delta y=0,\quad \Delta x\to 0,
\end{equation}

此时

\begin{equation}
\lim_{\Delta z\to 0}\frac{\Delta u+\mathrm{i}\Delta v}{\Delta x+\mathrm{i}\Delta y}
=\lim_{\Delta x\to 0}\frac{\Delta u+\mathrm{i}\Delta v}{\Delta x}
=\frac{\partial u}{\partial x}+\mathrm{i}\frac{\partial v}{\partial x}.
\end{equation}

(2)令

\begin{equation}
\Delta x=0,\quad\mathrm{i}\Delta y\to 0,
\end{equation}

此时

\begin{equation}
\lim_{\Delta z\to 0}\frac{\Delta u+\mathrm{i}\Delta v}{\Delta x+\mathrm{i}\Delta y}
=-\mathrm{i}\frac{\partial u}{\partial y}+\frac{\partial v}{\partial y}.
\end{equation}

由于 $f(z)$ 在 $z_0$ 点可导,则这两个导数值应该相等,对比实部和虚部就得到

\begin{equation}
\frac{\partial u}{\partial x}
=\frac{\partial v}{\partial y},\quad
\frac{\partial u}{\partial y}
=-\frac{\partial v}{\partial x}.
\end{equation}

\end{proof}

C-R 条件是 $f(z) $ 在 $z $ 点可导的必要条件,但不是充分条件。也就是说,可导必定满足 C-R 条件,但满足 C-R 条件不一定可导。

\subsection{解析函数的定义}

若复变函数 $f(z)$ 在 $z_0$ 的邻域内每一点都可导,则称 $f(z)$ 在 $z_0$ 点是解析的。

若复变函数 $f(z)$ 在 $\Omega$ 内每一点都可导,则称 $f(z)$ 在 $\Omega$ 内是解析的,或称为全纯的。

\subsection{例题}

\begin{example}
    
已知解析函数 $f(z)=u+\mathrm{i}v $ 的实部 $u=x^3-3xy^2 $,求该解析函数。

\end{example}

\begin{solution}
    
解析函数应满足柯西-黎曼条件:

\begin{equation}
\frac{\partial u}{\partial x}
=\frac{\partial v}{\partial y},\quad
\frac{\partial u}{\partial y}
=-\frac{\partial v}{\partial x},
\end{equation}

\begin{equation}
\frac{\partial v}{\partial y}
=\frac{\partial u }{\partial x } 
=3x^2-3y^2,\quad
\frac{\partial v}{\partial x}
=-\frac{\partial u }{\partial y } 
=6xy,
\end{equation}

\begin{equation}
\mathrm{d}v(x,y)
=\frac{\partial v }{\partial x }\mathrm{d}x + \frac{\partial v }{\partial y } \mathrm{d}y
=6xy\mathrm{d}x+\left(3x^2-3y^2\right)\mathrm{d}y.
\end{equation}

选择积分路径为:$\underbrace{(0,0)\to(x_0,0)}_{C_1},\underbrace{(x_0,0)\to(x_0,y_0)}_{C_2}$,在路径 $C_1 $ 上有 $y=0,\mathrm{d}y=0 $,在路径 $C_2 $ 上有 $x=x_0,\mathrm{d}x=0 $,两边积分:

\begin{equation}
\begin{split}
v(x_0,y_0)-v(0,0)
&=\int\limits_{C_1} 6xy\mathrm{d}x + \left(3x^2-3y^2\right)\mathrm{d}y + \int\limits_{C_2}6xy\mathrm{d}x + \left(3x^2-3y^2\right)\mathrm{d}y \\
&=0+\int_{y=0}^{y=y_0}\left(3x_0^2-3y^2\right)\mathrm{d}y \\
&=3x_0^2 y_0-y_0^3.
\end{split}
\end{equation}

令 $v(0,0)=C$,则:

\begin{equation}
v(x,y)
=3x^2y-y^3+v(0,0)
=3x^2y-y^3+C,
\end{equation}

于是:

\begin{equation}
\begin{split}
f(z)
&=u(x,y)+\mathrm{i}v(x,y) \\
&=x^3-3xy^2+\mathrm{i}\left(3x^2 y-y^3+C\right).
\end{split}
\end{equation}

\end{solution}

\begin{example}
    
已知解析函数 $f(z)=u + \mathrm{i} v $ 的虚部 $\displaystyle{v=\frac{y }{x^2+y^2 }  }$,求该解析函数。

\end{example}

\begin{solution}

先计算偏微分:

\begin{equation}
\frac{\partial v}{\partial x}
=\frac{-2xy}{\left(x^2+y^2\right)^2},\quad
\frac{\partial v}{\partial y}
=\frac{x^2-y^2}{\left(x^2+y^2\right)^2},
\end{equation}

函数解析,故满足 C-R 条件,即满足:

\begin{equation}
\frac{\partial u}{\partial x}
=\frac{\partial v }{\partial y } 
=\frac{x^2-y^2}{\left(x^2+y^2\right)^2},
\end{equation}

\begin{equation}
\frac{\partial u}{\partial y}
=-\frac{\partial v }{\partial x } 
=\frac{2xy}{\left(x^2+y^2\right)^2},
\end{equation}

于是:

\begin{equation}
\mathrm{d}u
=\frac{\partial u }{\partial x } \mathrm{d}x + \frac{\partial u }{\partial y } \mathrm{d}y
=\frac{x^2-y^2}{\left(x^2+y^2\right)^2}\mathrm{d}x+\frac{2xy}{\left(x^2+y^2\right)^2}\mathrm{d}y,
\end{equation}

看到 $\left(x^2+y^2 \right) $,很自然想到极坐标变换:

\begin{equation}
\left\{
\begin{aligned}
&x=\rho\cos\varphi \\
&y=\rho\sin\varphi
\end{aligned}
\right.
\Longrightarrow
\left\{
\begin{aligned}
&\mathrm{d}x = \frac{\partial x }{\partial \rho } \mathrm{d}\rho + \frac{\partial x }{\partial \varphi } \mathrm{d}\varphi = \cos\varphi\mathrm{d}\rho - \rho \sin\varphi \mathrm{d}\varphi \\
&\mathrm{d}y = \frac{\partial y }{\partial \rho } \mathrm{d}\rho + \frac{\partial y }{\partial \varphi } \mathrm{d}\varphi = \sin\varphi \mathrm{d}\rho + \rho\cos\varphi \mathrm{d}\varphi
\end{aligned}
\right. ,
\end{equation}

于是:

\begin{equation}
\begin{split}
\mathrm{d}u
&=\frac{x^2-y^2}{\left(x^2+y^2\right)^2}\mathrm{d}x+\frac{2xy}{\left(x^2+y^2\right)^2}\mathrm{d}y \\
&=\frac{\cos\varphi }{\rho^2 } \mathrm{d}\rho + \frac{\sin\varphi }{\rho } \mathrm{d}\varphi \\
&=\mathrm{d}\left(\frac{-\cos\varphi }{\rho }  \right),
\end{split}
\end{equation}

于是:

\begin{equation}
u
=\frac{-\cos\varphi }{\rho } + C
=-\frac{x }{x^2+y^2 } + C,
\end{equation}

综上,

\begin{equation}
\begin{split}
f(z)
&=u + \mathrm{i} v \\
&=\left(-\frac{x }{x^2+y^2 } + C \right) + \mathrm{i}\left(\frac{y }{x^2+y^2 } \right).
\end{split}
\end{equation}

\end{solution}

\section{复变函数积分}

\subsection{复变函数积分的定义}

\subsection{柯西积分定理}

\subsection{柯西积分公式}

\subsection{解析函数高阶导数的积分表达式}

\section{复变函数的级数展开}

\subsection{复变函数项级数}

\subsection{解析函数的泰勒展开}

\subsection{解析函数的洛朗展开}

\section{留数定理及其在实积分中的应用}

\subsection{留数定理}

\subsection{留数的一般求法}

\subsection{留数定理在实积分中的应用}



