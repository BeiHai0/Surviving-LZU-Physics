\chapter{\texorpdfstring{$\mathbb{R}^3$空间的向量分析}{R3空间的向量分析}} % 不这样写的话会报warning,原因是pdf不允许超链接标题出现 $$ 这样的东西

%----------------------------------------

\section{向量分析基本知识}

\subsection{爱因斯坦求和约定}

爱因斯坦求和约定就是说,在同一代数项中见到两个重复指标 $i$ 就自动进行求和(除非特别指出该重复指标不求和),我们称求和指标 $i$ 为“哑标”。

比如,$\mathbb{R}^3 $ 空间中的向量 $\vec{A}\in \mathbb{R}^3 $ 在直角坐标下可表示为

\begin{equation}
\vec{A}
=A_1\vec{\mathrm{e}}_1+A_2\vec{\mathrm{e}}_2+A_3\vec{\mathrm{e}}_3
\equiv \sum_{i} A_i \vec{\mathrm{e}}_i,
\end{equation}

其中,$\vec{\mathrm{e}}_1,\vec{\mathrm{e}}_2,\vec{\mathrm{e}}_3 $ 分别是 $x,y,z $ 轴正方向上的单位向量。

可利用爱因斯坦求和约定将 $\vec{A}\in \mathbb{R}^3 $ 简写为

\begin{equation}
\vec{A}
=\sum_{i} A_i \vec{\mathrm{e}}_i
\to
\vec{A}
=A_i \vec{\mathrm{e}}_i,
\end{equation}

这样就省去了写求和符号 $\displaystyle{\sum_i }$  的工作。 

\subsection{\texorpdfstring{Kronecher delta 符号 $\delta_{ij} $}{Kronecher delta 符号 }}

\begin{equation}
\delta_{ij}
\equiv\begin{cases}
1&,i=j \\
0&,i\ne j
\end{cases}.
\end{equation}

\subsection{\texorpdfstring{三阶 Levi-Citita 符号 $\varepsilon_{ijk} $}{三阶 Levi-Citita 符号 }}

\begin{equation}
\varepsilon_{ijk}
\equiv\begin{cases}
1&,ijk=123,231,312,\text{即相邻两指标经过偶次对换能还原到}123 \\
-1&,ijk=132,213,321,\text{即相邻两指标经过奇次对换能还原到}123 \\
0&,ijk\text{中有相同指标} \\
\end{cases}.
\end{equation}

可以利用 $\varepsilon_{ijk} $ 表示任何一个三阶行列式:

\begin{equation}
\begin{vmatrix}
a_1 &a_2 &a_3 \\
b_1 &b_2 &b_3 \\
c_1 &c_2 &c_3
\end{vmatrix}
=\varepsilon_{ijk} a_i b_j c_k.
\end{equation}

\subsection{一些简单算例}

\begin{example}
一些简单算例
\begin{itemize}
\item $\vec{\mathrm{e}}_i\cdot\vec{\mathrm{e}}_j=\delta_{ij} $,
\item $\vec{\mathrm{e}}_i\times\vec{\mathrm{e}}_j=\varepsilon_{ijk}\vec{\mathrm{e}}_k $,
\item $A_{i}\delta_{ij}=A_j $,
\item $\vec{A}\cdot\vec{B}=A_i B_i $,
\begin{proof}
\begin{equation}
\vec{A}\cdot\vec{B}
=(A_i\vec{\mathrm{e}}_i)\cdot (B_j\vec{\mathrm{e}}_j)
=A_i B_j\vec{\mathrm{e}}_i\cdot \vec{\mathrm{e}}_j
=A_i B_j \delta_{ij}
=A_i B_i
\end{equation}
\end{proof}

\item $\vec{A}\times\vec{B}=\varepsilon_{ijk}\vec{\mathrm{e}}_i A_j B_k $,
\begin{proof}
\begin{equation}
\vec{A}\times\vec{B}
=\begin{vmatrix}
\vec{\mathrm{e}}_1 &\vec{\mathrm{e}}_2 &\vec{\mathrm{e}}_3 \\
A_1 &A_2 &A_3 \\
B_1 &B_2 &B_3
\end{vmatrix}
=\varepsilon_{ijk} \vec{\mathrm{e}}_i A_j B_k
\end{equation}
\end{proof}
\end{itemize}
\end{example}

\subsection{\texorpdfstring{$\nabla$算子}{nabla算子}}

$\nabla $ 算子(nabla 算子,或 del 算子)定义为

\begin{equation}
\nabla
\equiv \vec{\mathrm{e}}_i\partial_i,
\end{equation}

其中,$\partial_i $ 的定义为

\begin{equation}
\partial_i\equiv \frac{\partial }{\partial x_i}.
\end{equation}

\subsection{标量场的梯度、方向导数、梯度定理}

\subsubsection{标量场梯度的定义}

设 $\psi(\vec{x}) $ 是标量场,$\psi(\vec{x}) $ 的梯度,记为 $\mathrm{grad}~\psi(\vec{x}) $,由下式定义

\begin{equation}
\mathrm{grad}~\psi(\vec{x})\cdot\mathrm{d}\vec{x}
=\mathrm{d}\psi(\vec{x}),
\end{equation}

其中,$\mathrm{d}\vec{x} $ 是位矢 $\vec{x} $ 的任意微小变化,$\mathrm{d}\psi(\vec{x}) $ 是标量场 $\psi(\vec{x}) $ 因位矢 $\vec{x} $ 变化 $\mathrm{d}\vec{x} $ 而引起的相应的变化。具体来说,$\mathrm{d}\psi(\vec{x}) $ 的定义为

\begin{equation}
\mathrm{d}\psi(\vec{x})
\equiv \psi(\vec{x}+\mathrm{d}\vec{x})-\psi(\vec{x}).
\end{equation}

可以证明,标量场的梯度 $\mathrm{grad}~\psi(\vec{x}) $ 可以用 $\nabla $ 算子表达为

\begin{equation}
\mathrm{grad}~\psi(\vec{x})
=\nabla \psi(\vec{x}).
\end{equation}

为了书写方便,以后就用 $\nabla \psi(\vec{x}) $ 指代标量场 $\psi(\vec{x}) $ 的梯度。

\subsubsection{方向导数的定义}

标量场 $\psi $ 在 $\vec{x} $ 点处沿 $\vec{v} $ 方向的方向导数,记为 $\displaystyle{\frac{\partial \psi(\vec{x}) }{\partial l }\bigg|_{\vec{v}}  }$,定义为

\begin{equation}
\left. \frac{\partial \psi(\vec{x}) }{\partial l }\right|_{\vec{v}}
\equiv \lim_{t\to 0^+} \frac{\psi(\vec{x} + t\vec{v}) - \psi(\vec{x}) }{tv }.
\end{equation}

从方向导数的定义可以看出,方向导数描述的是标量场沿某一方向变化的快慢。

特别地,标量场 $\psi $ 在曲面 $\Sigma $ 上的 $\vec{x} $ 点处沿曲面上 $\vec{x} $ 点的外法向的方向导数简记为 $\displaystyle{\left. \frac{\partial\psi(\vec{x}) }{\partial n }\right|_{\Sigma}. }$ 

\subsubsection{标量场梯度与方向导数的关系}

标量场的梯度的定义:

\begin{equation}
\nabla\psi(\vec{x})\cdot\mathrm{d}\vec{x}
=\mathrm{d}\psi(\vec{x}),
\end{equation}

设 $\mathrm{d}\vec{x} = \vec{n} \mathrm{d}x $,其中 $\vec{n} $ 是与 $\mathrm{d}\vec{x} $ 同向的单位向量,则有

\begin{equation}
[\nabla\psi(\vec{x})] \cdot \vec{n} \mathrm{d}x
=\mathrm{d}\psi(\vec{x}),
\end{equation}

即

\begin{equation}
[\nabla\psi(\vec{x})] \cdot \vec{n}
=\frac{\mathrm{d}\psi(\vec{x}) }{\mathrm{d}x }
=\frac{\psi(\vec{x}+\mathrm{d}\vec{x}) - \psi(\vec{x}) }{\mathrm{d}x }
=\frac{\partial \psi(\vec{x}) }{\partial l }\bigg|_{\vec{n}}.
\end{equation}

这就是说,标量场 $\psi(\vec{x}) $ 的梯度 $\nabla\psi(\vec{x}) $ 在某一方向 $\vec{n} $ 上的投影 $[\nabla\psi(\vec{x})]\cdot\vec{n} $ 恰等于标量场沿这一方向 $\vec{n} $ 的方向导数 $\displaystyle{\frac{\partial \psi(\vec{x}) }{\partial l }\bigg|_{\vec{n}}  }$。

\subsubsection{标量场梯度的意义}

考虑标量场梯度与方向导数的关系

\begin{equation}
[\nabla\psi(\vec{x})] \cdot \vec{n}
=\frac{\partial \psi(\vec{x}) }{\partial l }\bigg|_{\vec{n}},
\end{equation}

有:

\begin{equation}
\frac{\partial \psi(\vec{x}) }{\partial l }\bigg|_{\vec{n}}
=[\nabla\psi(\vec{x})] \cdot \vec{n}
=\left|\nabla\psi(\vec{x}) \right| \left|\vec{n} \right| \cos\Braket{\nabla\psi(\vec{x}) , \vec{n} }
=\left|\nabla\psi(\vec{x}) \right| \cos\Braket{\nabla\psi(\vec{x}) , \vec{n} },
\end{equation}

上式中,$\vec{n} $ 为方向任意的单位向量。

对于确定的场点 $\vec{x} $,$\psi(\vec{x}) $ 和 $\nabla\psi(\vec{x}) $ 也是确定的,则 $\left|\nabla \psi(\vec{x}) \right| $ 是确定的。

现在我们想看看 $\psi(\vec{x}) $ 沿哪个方向的变化速度最快,也就是看 $\psi(\vec{x}) $ 在哪个方向上的方向导数最大。

显然,在固定场点 $\vec{x} $ 的情况下,当 $\vec{n} $ 与 $\nabla \psi(\vec{x}) $ 同向时,也即 $\vec{n}=\nabla\psi(\vec{x})/\left|\nabla\psi(\vec{x}) \right| $ 时,$\psi(\vec{x}) $ 在 $\vec{n} $ 方向上的方向导数 $\displaystyle{\frac{\partial \psi(\vec{x}) }{\partial l }\bigg|_{\vec{n}} }$ 最大,这个最大的方向导数为

$$
\frac{\partial \psi(\vec{x}) }{\partial l }\bigg|_{\vec{n}=\nabla\psi(\vec{x})/\left|\nabla\psi(\vec{x}) \right|}
=\left|\nabla\psi(\vec{x}) \right| \cos\Braket{\nabla\psi(\vec{x}) , \vec{n} }
=\left|\nabla\psi(\vec{x}) \right|.
$$

也就是说,标量场 $\psi(\vec{x}) $ 的梯度 $\nabla \psi(\vec{x}) $ 的方向就是标量场 $\psi(\vec{x}) $ 方向导数最大的方向;标量场梯度 $\nabla \psi(\vec{x}) $ 的大小 $\left|\nabla \psi(\vec{x}) \right| $ 就是最大方向导数。

\subsubsection{梯度定理}

\begin{theorem}
设 $\psi(\vec{x}) $ 是标量场,$C $ 是连结 $\vec{p},\vec{q}\in \mathbb{R}^3 $ 的任一曲线,则有

\begin{equation}
\psi\left(\vec{p} \right) - \psi\left(\vec{q} \right)
=\int\limits_{\vec{x}\in C[\vec{q}\to \vec{p}]} \nabla \psi(\vec{x}) \cdot \mathrm{d}\vec{x}.
\end{equation}
\end{theorem}

证明思路也很简单,把曲线 $C $ 分成很多小的有向线元,对每一段有向线元都使用梯度的定义,最后把结果加起来就得证。

\subsection{矢量场的散度、高斯定理}

\subsubsection{矢量场散度的定义}

矢量场 $\vec{A} $ 的散度,记为 $\mathrm{div}~\vec{A} $,定义为

\begin{equation}
\mathrm{div}~\vec{A}
\equiv \lim_{V\to 0^+} \frac{1 }{V } \oint\limits_{\partial V^+} \vec{A}\cdot\mathrm{d}\vec{S}.
\end{equation}

可以证明,矢量场 $\vec{A} $ 的散度 $\mathrm{div}~\vec{A} $ 可以用 $\nabla $ 算子表达为

\begin{equation}
\mathrm{div}~\vec{A}
=\nabla \cdot \vec{A}
\end{equation}

为了书写方便,以后就用 $\nabla \cdot \vec{A} $ 指代矢量场 $\vec{A} $ 的散度。

\subsubsection{高斯定理}

\begin{theorem}
设 $\vec{A}(\vec{x}) $ 是矢量场,$V $ 是 $\mathbb{R}^3 $ 中的封闭体,则有
\begin{equation}
\oint\limits_{\partial V^+} \vec{A}\cdot\mathrm{d}\vec{S}
=\int\limits_{V} \left(\nabla\cdot\vec{A}\right)\mathrm{d}V.
\end{equation}
\end{theorem}

证明的思路也很简单,把区域 $V $ 分成很多小体积元,对每个体积元都使用矢量场散度的定义,最后把结果加起来就得证。

\subsection{矢量场的旋度、斯托克斯定理}

\subsubsection{矢量场的旋度}

矢量场 $\vec{A} $ 的旋度,记为 $\mathrm{curl}~\vec{A} $,由下式定义:

\begin{equation}
\left(\mathrm{curl}~\vec{A} \right)\cdot \vec{n}
=\lim_{\sigma\to 0^+} \frac{1 }{\sigma } \oint\limits_{\partial \sigma^+} \vec{A}\cdot\mathrm{d}\vec{l},
\end{equation}

其中,$\sigma $ 是与 $\vec{n} $ 垂直的面元。$\vec{n} $ 与 $\partial\sigma $ 的正绕行方向满足右手定则。

可以证明,矢量场 $\vec{A} $ 的旋度 $\mathrm{curl}~\vec{A} $ 可以用 $\nabla $ 算子表达为

\begin{equation}
\mathrm{curl}~\vec{A}
=\nabla \times \vec{A}.
\end{equation}

为了书写方便,以后就用 $\nabla \times \vec{A} $ 指代矢量场 $\vec{A} $ 的旋度。

\subsubsection{斯托克斯定理}

\begin{theorem}
设 $\vec{A}(\vec{x}) $ 是矢量场,$\Sigma $ 是 $\mathbb{R}^3 $ 中的封闭曲面,则有

\begin{equation}
\oint\limits_{\partial\Sigma^+}\vec{A}\cdot\mathrm{d}\vec{l}
=\int\limits_{\Sigma} \left(\nabla\times\vec{A}\right)\cdot\mathrm{d}\vec{S},
\end{equation}

其中,曲面 $\Sigma $ 的取向与 $\partial\Sigma $ 的正绕行方向满足右手定则。

\end{theorem}

证明的思路也很简单,把曲面 $\Sigma $ 分成很多小面元,对每个面元都使用矢量场旋度的定义,最后把结果加起来就得证。

%----------------------------------------

\section{向量分析常用公式}

\subsection{分析工具}

\begin{equation}
\left\{
\begin{aligned}
&\vec{\mathrm{e}}_i\cdot\vec{\mathrm{e}}_j=\delta_{ij} \\
&\vec{\mathrm{e}}_i\times\vec{\mathrm{e}}_j=\varepsilon_{ijk}\vec{\mathrm{e}}_k \\
&\vec{A} = A_i\vec{\mathrm{e}}_i \\
&A_{i}\delta_{ij}=A_j \\
&\vec{A}\cdot\vec{B}=A_i B_i \\
&\vec{A}\times\vec{B}=\varepsilon_{ijk}\vec{\mathrm{e}}_i A_j B_k \\
&\left(\vec{A}\times\vec{B}\right)_l = \varepsilon_{ljk} A_j B_k \\
&\nabla = \vec{\mathrm{e}}_i\partial_i \\
&\nabla\cdot\vec{A}=\partial_i A_i \\
&\nabla\times\vec{A} = \varepsilon_{ijk} \vec{\mathrm{e}}_i \partial_j A_k \\
&\partial_i \psi = (\nabla\psi)_i \\
&\nabla^2 \equiv \nabla\cdot \nabla = \partial_i \partial_i \\
&\nabla^2\psi\equiv \nabla\cdot(\nabla\psi)=\partial_i\partial_i \psi \\
&\nabla^2\vec{A}\equiv \left(\nabla^2 A_i\right)\vec{\mathrm{e}}_i \\
&\varepsilon_{ijk} = \varepsilon_{jki} = \varepsilon_{kij} \\
&\varepsilon_{\textcolor{red}{i}jk}\varepsilon_{\textcolor{red}{i}lm} = \delta_{jl}\delta_{km} - \delta_{jm}\delta_{kl} \\
&\partial_i x_j = \delta_{ij}
\end{aligned}
\right.
\end{equation}

\subsection{\texorpdfstring{$\mathbb{R}^3 $空间重要微分恒等式及其证明}{R3空间重要微分恒等式及其证明}}

\subsubsection{\texorpdfstring{与$\vec{x}$有关的公式}{与x有关的公式}}

\begin{example}
\begin{equation}
\nabla\cdot\vec{x}
=3.
\end{equation}
\end{example}

\begin{proof}
\begin{equation}
\nabla\cdot\vec{x}
=\partial_i x_i
=\delta_{ii}
=3.
\end{equation}
\end{proof}

\begin{example}
\begin{equation}
\nabla\times\vec{x}
=\vec{0}.
\end{equation}
\end{example}

\begin{proof}
\begin{equation}
\nabla\times\vec{x}
=\varepsilon_{ijk}\vec{\mathrm{e}}_i\partial_j x_k
=\varepsilon_{ijk}\vec{\mathrm{e}}_i \delta_{jk}
=\varepsilon_{ikk}\vec{\mathrm{e}}_i
=\vec{0}.
\end{equation}
\end{proof}



\subsubsection{从左往右证的公式}

\begin{example}
\begin{equation}
\nabla(\varphi\psi)
=\varphi\nabla\psi+\psi\nabla\varphi.
\end{equation}
\end{example}

\begin{proof}
\begin{equation}
\begin{split}
\nabla(\varphi\psi)
&=\vec{\mathrm{e}}_i\partial_i(\varphi\psi) \\
&=\vec{\mathrm{e}}_i\varphi\partial_i\psi + \vec{\mathrm{e}}_i\psi\partial_i\varphi \\
&=\varphi\vec{\mathrm{e}}_i\partial_i\psi + \psi\vec{\mathrm{e}}_i\partial_i\varphi \\
&=\varphi\nabla\psi + \psi\nabla\varphi.
\end{split}
\end{equation}
\end{proof}

\begin{example}
\begin{equation}
\nabla\cdot\left(\varphi\vec{A}\right)
=\vec{A}\cdot(\nabla\varphi) + \varphi\nabla\cdot\vec{A}.
\end{equation}
\end{example}

\begin{proof}
\begin{equation}
\begin{split}
\nabla\cdot\left(\varphi\vec{A}\right)
&=\partial_i\left(\varphi \vec{A}\right)_i \\
&=\partial_i\left(\varphi A_i\right) \\
&=A_i\partial_i \varphi + \varphi\partial_i A_i \\
&=\left(\vec{A}\cdot\nabla\right)\varphi + \varphi\nabla\cdot\vec{A}.
\end{split}
\end{equation}
\end{proof}

\begin{example}
\begin{equation}
\nabla\times\left(\varphi\vec{A}\right)
=(\nabla\varphi)\times\vec{A} + \varphi \nabla\times\vec{A}.
\end{equation}
\end{example}

\begin{proof}
\begin{equation}
\begin{split}
\nabla\times(\varphi\vec{A})
&=\varepsilon_{ijk}\vec{\mathrm{e}}_i\partial_j \left(\varphi\vec{A}\right)_k \\
&=\varepsilon_{ijk}\vec{\mathrm{e}}_i \partial_j(\varphi A_k) \\
&=\varepsilon_{ijk}\vec{\mathrm{e}}_i (A_k\partial_j \varphi + \varphi\partial_j A_k) \\
&=\varepsilon_{ijk} \vec{\mathrm{e}}_i(\nabla\varphi)_j A_k + \varphi \varepsilon_{ijk} \vec{\mathrm{e}}_i \partial_j A_k \\
&=(\nabla\varphi)\times \vec{A} + \varphi \nabla\times\vec{A} .
\end{split}
\end{equation}
\end{proof}

\begin{example}
\begin{equation}
\nabla\cdot \left(\vec{A}\times\vec{B} \right)
=\vec{B}\cdot\left(\nabla\times\vec{A}\right)-\vec{A}\cdot\left(\nabla\times\vec{B}\right).
\end{equation}
\end{example}

\begin{proof}
\begin{equation}
\begin{split}
\nabla\cdot \left(\vec{A}\times\vec{B} \right)
&=\partial_i \left(\vec{A}\times\vec{B}\right)_i \\
&=\partial_i (\varepsilon_{ijk} A_j B_k) \\
&=\varepsilon_{ijk}\partial_i ( A_j B_k) \\
&=\varepsilon_{ijk} B_k\partial_i A_j + \varepsilon_{ijk} A_j \partial_i B_k \\
&=B_k \varepsilon_{kij} \partial_i A_j - A_j \varepsilon_{jik} \partial_i B_k \\
&=B_k \left(\nabla\times \vec{A}\right)_k - A_j \left(\nabla\times \vec{B}\right)_j \\
&=\vec{B}\cdot\left(\nabla\times \vec{A}\right) - \vec{A}\cdot\left(\nabla\times\vec{B}\right).
\end{split}
\end{equation}
\end{proof}

\begin{example}
\begin{equation}
\nabla\times\left(\vec{A}\times\vec{B}\right)
=\left(\vec{B}\cdot\nabla\right)\vec{A} - \left(\vec{A}\cdot\nabla\right) \vec{B} + \vec{A}\left(\nabla\cdot\vec{B}\right) - \vec{B}\left(\nabla\cdot\vec{A}\right).
\end{equation}
\end{example}

\begin{proof}
\begin{equation}
\begin{split}
\nabla\times\left(\vec{A}\times\vec{B}\right)
&=\varepsilon_{ijk} \vec{\mathrm{e}}_i \partial_j \left(\vec{A}\times\vec{B}\right)_k \\
&=\varepsilon_{ijk} \vec{\mathrm{e}}_i \partial_j \varepsilon_{klm} A_l B_m \\
&=\varepsilon_{kij} \varepsilon_{klm} \vec{\mathrm{e}}_i \partial_j (A_l B_m) \\
&=(\delta_{il}\delta_{jm}-\delta_{im}\delta_{jl}) \vec{\mathrm{e}}_i (B_m\partial_j A_l + A_l\partial_j B_m) \\
&=\vec{\mathrm{e}}_l B_j\partial_j A_l + \vec{\mathrm{e}}_l A_l \partial_m B_m - \vec{\mathrm{e}}_m B_m \partial_l A_l - \vec{\mathrm{e}}_m A_j\partial_j B_m \\
&= B_j\partial_j A_l\vec{\mathrm{e}}_l + \vec{\mathrm{e}}_l A_l \partial_m B_m - \vec{\mathrm{e}}_m B_m \partial_l A_l -  A_j\partial_j B_m \vec{\mathrm{e}}_m \\
&=\left(\vec{B}\cdot\nabla\right)\vec{A} + \vec{A}\left(\nabla\cdot\vec{B}\right) - \vec{B}\left(\nabla\cdot\vec{A}\right) - \left(\vec{A}\cdot\nabla\right) \vec{B} \\
&=\left(\vec{B}\cdot\nabla\right)\vec{A} - \left(\vec{A}\cdot\nabla\right) \vec{B} + \vec{A}\left(\nabla\cdot\vec{B}\right) - \vec{B}\left(\nabla\cdot\vec{A}\right).
\end{split}
\end{equation}
\end{proof}

\begin{example}
\begin{equation}
\nabla\times\left(\nabla\times\vec{A}\right)
=\nabla\left(\nabla\cdot\vec{A}\right)-\nabla^2 \vec{A}.
\end{equation}
\end{example}

\begin{proof}
\begin{equation}
\begin{split}
\nabla\times\left(\nabla\times\vec{A}\right)
&=\varepsilon_{ijk} \vec{\mathrm{e}}_i\partial_j \left(\nabla\times\vec{A}\right)_k \\
&=\varepsilon_{ijk} \vec{\mathrm{e}}_i \partial_j \varepsilon_{klm} \partial_l A_m \\
&=\varepsilon_{kij}\varepsilon_{klm} \vec{\mathrm{e}}_i\partial_j\partial_l A_m \\
&=(\delta_{il}\delta_{jm}-\delta_{im}\delta_{jl}) \vec{\mathrm{e}}_i \partial_j \partial_l A_m \\
&=\vec{\mathrm{e}}_l\partial_m \partial_l A_m - \vec{\mathrm{e}}_m\partial_l \partial_l A_m \\
&=\vec{\mathrm{e}}_l\partial_l\partial_m A_m -\partial_l\partial_l A_m\vec{\mathrm{e}}_m \\
&=\nabla\left(\nabla\cdot\vec{A}\right) - \nabla^2\vec{A}.
\end{split}
\end{equation}
\end{proof}

\subsubsection{需要注意力的公式}

\begin{example}
\begin{equation}
\nabla\times(\nabla \varphi)
=\vec{0}.
\end{equation}
\end{example}

\begin{proof}

\begin{equation}
\begin{split}
\nabla\times(\nabla \varphi)
&=\varepsilon_{ijk} \vec{\mathrm{e}}_i\partial_j (\nabla\varphi)_k \\
&=\vec{\mathrm{e}}_i\varepsilon_{ijk} \partial_j\partial_k\varphi,
\end{split}
\end{equation}

由于我们只考虑性质比较好的函数,于是 $\partial_j \partial_k\varphi=\partial_k\partial_j\varphi $,再结合 $\varepsilon_{ijk}=-\varepsilon_{ikj} $,有

\begin{equation}
\begin{split}
\vec{\mathrm{e}}_i\varepsilon_{ijk}\partial_j\partial_k\varphi
&=-\vec{\mathrm{e}}_i\varepsilon_{ikj}\partial_k\partial_j\varphi \\
&=-\vec{\mathrm{e}}_i\varepsilon_{ijk}\partial_j\partial_k\varphi.
\end{split}
\end{equation}

最后一步是因为 $j,k $ 都是用于求和的哑标,因此可以作替换 $j \leftrightarrow k $ 。上式说明:

\begin{equation}
\vec{\mathrm{e}}_i\varepsilon_{ijk}\partial_j\partial_k\varphi
=\vec{0}.
\end{equation}

于是

\begin{equation}
\begin{split}
\nabla\times(\nabla \varphi)
=\vec{\mathrm{e}}_i\varepsilon_{ijk}\partial_j\partial_k\varphi
=\vec{0}.
\end{split}
\end{equation}

\end{proof}

\begin{example}
\begin{equation}
\nabla\cdot\left(\nabla\times \vec{A}\right)
=0.
\end{equation}
\end{example}

\begin{proof}

\begin{equation}
\begin{split}
\nabla\cdot\left(\nabla\times \vec{A}\right)
&=\partial_i \left(\nabla\times\vec{A}\right)_i \\
&=\partial_i \varepsilon_{ijk} \partial_j A_k \\
&=\varepsilon_{ijk}\partial_i\partial_j A_k,
\end{split}
\end{equation}

注意到

\begin{equation}
\begin{split}
\varepsilon_{ijk}\partial_i\partial_j A_k
&=-\varepsilon_{jik}\partial_j\partial_i A_k \\
&=-\varepsilon_{ijk}\partial_i\partial_j A_k,
\end{split}
\end{equation}

于是

\begin{equation}
\varepsilon_{ijk}\partial_i\partial_j A_k
=0,
\end{equation}

这就是说:

\begin{equation}
\nabla\cdot\left(\nabla\times \vec{A}\right)
=\varepsilon_{ijk}\partial_i\partial_j A_k
=0.
\end{equation}
\end{proof}

\subsubsection{从右往左证的公式}

\begin{example}
\begin{equation}
\nabla\left(\vec{A}\cdot\vec{B}\right)
=\left(\vec{B}\cdot \nabla\right)\vec{A}+\left(\vec{A}\cdot\nabla\right)\vec{B}+\vec{B}\times\left(\nabla\times\vec{A}\right)+\vec{A}\times\left(\nabla\times\vec{B}\right).
\end{equation}
\end{example}

\begin{proof}
\begin{equation}
\begin{split}
&\nabla\left(\vec{A}\cdot\vec{B}\right)=\left(\vec{B}\cdot \nabla\right)\vec{A}+\left(\vec{A}\cdot\nabla\right)\vec{B}+\vec{B}\times\left(\nabla\times\vec{A}\right)+\vec{A}\times\left(\nabla\times\vec{B}\right) \\
=&B_i\partial_i A_j \vec{\mathrm{e}}_j + A_i\partial_i B_j\vec{\mathrm{e}}_j + \varepsilon_{ijk}\vec{\mathrm{e}}_iB_j\left(\nabla\times\vec{A}\right)_k + \varepsilon_{ijk} \vec{\mathrm{e}}_i A_j\left(\nabla\times\vec{B}\right)_k \\
=&B_i\partial_i A_j \vec{\mathrm{e}}_j + A_i\partial_i B_j\vec{\mathrm{e}}_j + \varepsilon_{ijk}\vec{\mathrm{e}}_iB_j\varepsilon_{klm}\partial_l A_m + \varepsilon_{ijk} \vec{\mathrm{e}}_i A_j\varepsilon_{klm}\partial_l B_m \\
=&B_i\partial_i A_j \vec{\mathrm{e}}_j + A_i\partial_i B_j\vec{\mathrm{e}}_j + \varepsilon_{kij}\varepsilon_{klm}\vec{\mathrm{e}}_iB_j\partial_l A_m + \varepsilon_{kij}\varepsilon_{klm} \vec{\mathrm{e}}_i A_j\partial_l B_m \\
=&B_i\partial_i A_j \vec{\mathrm{e}}_j + A_i\partial_i B_j\vec{\mathrm{e}}_j + (\delta_{il}\delta_{jm}-\delta_{im}\delta_{jl})\vec{\mathrm{e}}_iB_j\partial_l A_m + (\delta_{il}\delta_{jm}-\delta_{im}\delta_{jl}) \vec{\mathrm{e}}_i A_j\partial_l B_m \\
=&B_i\partial_i A_j \vec{\mathrm{e}}_j + A_i\partial_i B_j\vec{\mathrm{e}}_j + \vec{\mathrm{e}}_l B_m\partial_l A_m -\vec{\mathrm{e}}_m B_l\partial_l A_m + \vec{\mathrm{e}}_l A_m\partial_l B_m - \vec{\mathrm{e}}_m A_l\partial_l B_m \\
=&B_i\partial_i A_j \vec{\mathrm{e}}_j + A_i\partial_i B_j\vec{\mathrm{e}}_j +  B_m\vec{\mathrm{e}}_l\partial_l A_m -B_l\partial_l A_m \vec{\mathrm{e}}_m + A_m\vec{\mathrm{e}}_l \partial_l B_m -  A_l\partial_l B_m \vec{\mathrm{e}}_m \\
=&B_m\vec{\mathrm{e}}_l\partial_l A_m + A_m\vec{\mathrm{e}}_l \partial_l B_m \\
=&B_m\nabla A_m + A_m \nabla B_m \\
=&\nabla(A_m B_m) \\
=&\nabla\left(\vec{A}\cdot\vec{B}\right)
\end{split}
\end{equation}
\end{proof}

\subsection{\texorpdfstring{$\mathbb{R}^3 $空间重要积分恒等式及其证明}{R3空间重要积分恒等式及其证明}}

\subsubsection{高斯定理}

\begin{theorem}
设 $\vec{A}(\vec{x}) $ 是矢量场,$V $ 是 $\mathbb{R}^3 $ 中的封闭体,则有
\begin{equation}
\oint\limits_{\partial V^+} \vec{A}\cdot\mathrm{d}\vec{S}
=\int\limits_{V} \left(\nabla\cdot\vec{A}\right)\mathrm{d}V.
\end{equation}
\end{theorem}

\subsubsection{斯托克斯定理}

\begin{theorem}
设 $\vec{A}(\vec{x}) $ 是矢量场,$\Sigma $ 是 $\mathbb{R}^3 $ 中的封闭曲面,则有

\begin{equation}
\oint\limits_{\partial\Sigma^+}\vec{A}\cdot\mathrm{d}\vec{l}
=\int\limits_{\Sigma} \left(\nabla\times\vec{A}\right)\cdot\mathrm{d}\vec{S},
\end{equation}

其中,曲面 $\Sigma $ 的取向与 $\partial\Sigma $ 的正绕行方向满足右手定则。

\end{theorem}

\subsubsection{格林第一恒等式}

\begin{example}
\begin{equation}
\oint\limits_{\partial\Omega^+} \psi\nabla\phi\cdot\mathrm{d}\vec{S}
=\int\limits_{\Omega} \left(\psi\nabla^2\phi + \nabla\phi\cdot \nabla\psi \right)\mathrm{d}V.
\end{equation}
\end{example}

\begin{proof}

注意到

\begin{equation}
\begin{split}
\nabla\cdot(\psi\nabla\phi)
&=\partial_i(\psi\nabla\phi)_i \\
&=\partial_i(\psi\partial_i\phi) \\
&=(\partial_i \phi)(\partial_i\psi) + \psi\partial_i\partial_i\phi \\
&=(\nabla\phi)_i(\nabla\psi)_i + \psi\nabla^2\phi \\
&=(\nabla\phi)\cdot(\nabla\psi) + \psi\nabla^2\phi,
\end{split}
\end{equation}

于是由高斯定理,有

\begin{equation}
\begin{split}
\oint\limits_{\partial\Omega^+} \psi\nabla\phi\cdot\mathrm{d}\vec{S}
&=\int\limits_{\Omega} \nabla\cdot(\psi\nabla\phi)\mathrm{d}V \\
&=\int\limits_{\Omega} \left[(\nabla\phi)\cdot(\nabla\psi) + \psi\nabla^2\phi \right]\mathrm{d}V \\
&=\int\limits_{\Omega} \left[\psi\nabla^2\phi + (\nabla\phi)\cdot(\nabla\psi) \right]\mathrm{d}V .
\end{split}
\end{equation}

\end{proof}

\subsubsection{格林第二恒等式}

\begin{example}
\begin{equation}
\oint\limits_{\partial\Omega^+}\left(\psi\nabla\phi-\phi\nabla\psi \right)\cdot\mathrm{d}\vec{S} =\int\limits_{\Omega} \left(\psi\nabla^2\phi-\phi\nabla^2\psi\right)\mathrm{d}V.
\end{equation}
\end{example}

\begin{proof}

利用 $\nabla\cdot\left(\varphi\vec{A}\right)=\vec{A}\cdot(\nabla\varphi) + \varphi\nabla\cdot\vec{A} $ 有

\begin{equation}
\begin{split}
\nabla\cdot(\psi\nabla\phi-\phi\nabla\psi)
&=\nabla\phi\cdot\nabla\psi + \psi\nabla\cdot(\nabla\phi) - \left(\nabla\psi\cdot\nabla\phi + \phi\nabla\cdot(\nabla\psi) \right) \\
&=\psi\nabla^2\phi-\phi\nabla^2\psi,
\end{split}
\end{equation}

于是由高斯定理可得

\begin{equation}
\begin{split}
\oint\limits_{\partial\Omega^+}\left(\psi\nabla\phi-\phi\nabla\psi \right)\cdot\mathrm{d}\vec{S}
&=\int\limits_{\Omega} \nabla\cdot(\psi\nabla\phi-\phi\nabla\psi) \mathrm{d}V \\
&=\int\limits_{\Omega} \left(\psi\nabla^2\phi-\phi\nabla^2\psi\right) \mathrm{d}V.
\end{split}
\end{equation}

\end{proof}

\subsubsection{高斯定理的一个推论}

\begin{example}
\begin{equation}
\oint\limits_{\partial V^+} \psi \mathrm{d}\vec{S}
=\int\limits_{V} \nabla \psi \mathrm{d}V.
\end{equation}
\end{example}

\begin{proof}

对任意标量场 $\psi(\vec{x}) $ 和任意常矢量 $\vec{a} $,构造矢量场

\begin{equation}
\vec{A}(\vec{x})
\equiv \psi (\vec{x}) \vec{a},
\end{equation}

这个特殊的矢量场 $\vec{A} $ 应当满足高斯定理:

\begin{equation}
\oint\limits_{\partial V^+} \vec{A}\cdot\mathrm{d}\vec{S}
=\int\limits_{V} (\nabla\cdot\vec{A})\mathrm{d}V.
\end{equation}

等式左边

\begin{equation}
\begin{split}
\oint\limits_{\partial V^+} \vec{A}\cdot\mathrm{d}\vec{S}
=\oint\limits_{\partial V^+} \left(\psi \vec{a} \right) \cdot \mathrm{d}\vec{S}
=\vec{a} \cdot \oint\limits_{\partial V^+} \psi \mathrm{d}\vec{S}.
\end{split}
\end{equation}

等式右边

\begin{equation}
\begin{split}
\int\limits_{V} \left(\nabla\cdot\vec{A}\right)\mathrm{d}V
&=\int\limits_{V} \left[\nabla\cdot\left(\psi \vec{a} \right)\right]\mathrm{d}V \\
&=\int\limits_{V} \left[ \left(\nabla \psi \right) \cdot \vec{a} + \psi \nabla \cdot \vec{a} \right] \mathrm{d}V \\
&=\int\limits_{V} \left(\nabla \psi \right) \cdot \vec{a} \mathrm{d}V \\
&=\vec{a} \cdot \int\limits_{V} \nabla \psi \mathrm{d}V,
\end{split}
\end{equation}

于是

\begin{equation}
\vec{a} \cdot \oint\limits_{\partial V^+} \psi \mathrm{d}\vec{S}
=\vec{a} \cdot \int\limits_{V} \nabla \psi \mathrm{d}V,
\end{equation}

由 $\vec{a} $ 的任意性就得到

\begin{equation}
\oint\limits_{\partial V^+} \psi \mathrm{d}\vec{S} = \int\limits_{V} \nabla \psi \mathrm{d}V.
\end{equation}

\end{proof}

\subsubsection{斯托克斯定理的一个推论}

\begin{example}
\begin{equation}
\oint\limits_{\partial S } \psi \mathrm{d}\vec{l}
=-\int\limits_{S} \nabla\psi\times\mathrm{d}\vec{S}.
\end{equation}
\end{example}

\begin{proof}

对任意标量场 $\psi(\vec{x}) $ 和任意常矢量 $\vec{a} $,构造矢量场

\begin{equation}
\vec{A}(\vec{x})
\equiv \psi (\vec{x}) \vec{a},
\end{equation}

这个特殊的矢量场 $\vec{A} $ 应当满足斯托克斯定理:

\begin{equation}
\oint\limits_{\partial S} \vec{A}\cdot\mathrm{d}\vec{l}
=\int\limits_{S} \left(\nabla \times \vec{A} \right) \cdot \mathrm{d}\vec{S}.
\end{equation}

等式左边

\begin{equation}
\begin{split}
\oint\limits_{\partial S} \vec{A}\cdot\mathrm{d}\vec{l}
&=\oint\limits_{\partial S} \left(\psi \vec{a} \right)\cdot\mathrm{d}\vec{l} \\
&=\vec{a} \cdot \oint\limits_{\partial S} \psi \mathrm{d}\vec{l},
\end{split}
\end{equation}

等式右边

\begin{equation}
\begin{split}
\int\limits_{S} \left(\nabla \times \vec{A} \right) \cdot \mathrm{d}\vec{S}
&=\int\limits_{S} \left[\nabla \times \left(\psi \vec{a} \right) \right] \cdot \mathrm{d}\vec{S} \\
&=\int\limits_{S} \left[\left(\nabla \psi \right) \times \vec{a} + \psi \nabla \times \vec{a} \right] \cdot \mathrm{d}\vec{S} \\
&=\int\limits_{S} \left[\left(\nabla \psi \right) \times \vec{a} \right] \cdot \mathrm{d}\vec{S} \\
&=\int\limits_{S} \left[\mathrm{d}\vec{S} \times \left(\nabla \psi \right) \right] \cdot \vec{a} \\
&=-\vec{a} \cdot \int\limits_{S} \nabla \psi \times \mathrm{d}\vec{S},
\end{split}
\end{equation}

于是

\begin{equation}
\vec{a} \cdot \oint\limits_{\partial S} \psi \mathrm{d}\vec{l}
=-\vec{a} \cdot \int\limits_{S} \nabla \psi \times \mathrm{d}\vec{S},
\end{equation}

由 $\vec{a} $ 的任意性就得到

\begin{equation}
\oint\limits_{\partial S} \psi \mathrm{d}\vec{l}
=-\int\limits_{S} \nabla \psi \times \mathrm{d}\vec{S}.
\end{equation}

\end{proof}



