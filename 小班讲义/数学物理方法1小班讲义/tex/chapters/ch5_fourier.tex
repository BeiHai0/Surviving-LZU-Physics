\chapter{傅里叶变换}

\section{傅里叶级数}

\subsection{周期函数作为Hilbert空间的元素}

设 $\mathcal{H}$ 是一个希尔伯特空间,其元素是周期为 $2l$ 的单变量函数,$\forall f_1,f_2\in \mathcal{H}$,$\mathcal{H}$ 上两个元素的内积,记为 $\braket{f_1,f_2} $,定义为:

\begin{equation}
\Braket{f_1 , f_2 }
\equiv \int_{x=-l}^{x=l} f_1^*(x)f_2(x)\mathrm{d}x,
\end{equation}

其中,$x $ 是参数,而内积与参数无关。为了指明参数,也可将内积写为:

\begin{equation}
\braket{f_1(x),f_2(x)}
\equiv \int_{x=-l}^{x=l} f_1^*(x)f_2(x)\mathrm{d}x.
\end{equation}

若 $f(x) $ 是实函数,则内积可简化为:

\begin{equation}
\Braket{f_1 , f_2 }
\equiv \int_{x=-l}^{x=l} f_1(x)f_2(x)\mathrm{d}x.
\end{equation}

\subsection{周期函数的三角函数基傅里叶级数}

可以验证以下结论:

\begin{equation}
\left\{
\begin{aligned}
&\int_{x=-l}^{x=l} \frac{1}{\sqrt{2l}}\cdot\frac{1}{\sqrt{2l}}\mathrm{d}x
=1 \\
&\int_{x=-l}^{x=l} \frac{1}{\sqrt{2l}}\cdot\frac{1}{\sqrt{l}}\sin \frac{n\pi}{l}x\mathrm{d}x
=0 \\
&\int_{x=-l}^{x=l} \frac{1}{\sqrt{2l}}\cdot\frac{1}{\sqrt{l}}\cos \frac{m\pi}{l} x\mathrm{d}x
=0 \\
&\int_{x=-l}^{x=l}\frac{1}{\sqrt{l}} \sin\frac{n\pi}{l} x\cdot\frac{1}{\sqrt{l}} \sin\frac{m\pi}{l} x\mathrm{d}x
=\delta_{n,m},\quad n=1,2,\cdots,m=0,1,2,\cdots \\
&\int_{x=-l}^{x=l}\frac{1}{\sqrt{l}} \sin\frac{n\pi}{l} x\cdot\frac{1}{\sqrt{l}} \cos\frac{m\pi}{l} x\mathrm{d}x
=0,\quad n=1,2,\cdots,m=0,1,2,\cdots \\
&\int_{x=-l}^{x=l}\frac{1}{\sqrt{l}} \cos\frac{n\pi}{l} x\cdot\frac{1}{\sqrt{l}} \cos\frac{m\pi}{l} x\mathrm{d}x
=\delta_{n,m},\quad n,m=0,1,2,\cdots
\end{aligned}
\right.
\end{equation}

可以证明,函数系 $\displaystyle{\left\{\frac{1}{\sqrt{2l}},\quad \frac{1}{\sqrt{l}}\sin\frac{n\pi}{l} x,\quad \frac{1}{\sqrt{l}}\cos\frac{m\pi}{l} x;\quad n,m=1,2,\cdots\right\} }$  是一个完备的正交归一函数族,任意一个周期为 $2l$ 的,满足狄利克雷条件的函数 $f(x)$ 可写成这些基函数的线性组合,即 $f(x) $ 可展成三角函数基的傅里叶级数:

\begin{equation}
f(x)
=a_0\cdot\frac{1}{\sqrt{2l}}+\sum_{k=1}^{\infty} \left(a_k\cdot\frac{1}{\sqrt{l}}\cos\frac{k\pi}{l} x+b_k\cdot\frac{1}{\sqrt{l}}\sin\frac{k\pi}{l} x \right)
\end{equation}

为求出线性组合的系数,只需要利用“这组基是正交归一完备的”这一性质,比如:

\begin{equation}
\begin{split}
\Braket{\frac{1 }{\sqrt{l} } \cos\frac{k'\pi }{l } x , f(x)}
&=\Braket{\frac{1 }{\sqrt{l} } \cos\frac{k'\pi }{l } x , a_0\cdot\frac{1}{\sqrt{2l}}+\sum_{k=1}^{\infty} \left(a_k\cdot\frac{1}{\sqrt{l}}\cos\frac{k\pi}{l} x+b_k\cdot\frac{1}{\sqrt{l}}\sin\frac{k\pi}{l} x \right)} \\
&=\sum_{k=1}^{\infty} a_k\Braket{\frac{1 }{\sqrt{l} } \cos\frac{k'\pi }{l } x , \frac{1}{\sqrt{l}}\cos\frac{k\pi}{l} x} \\
&=\sum_{k=1}^{\infty} a_k \delta_{k',k} \\
&=a_{k'}
\end{split}
\end{equation}

总之:

\begin{equation}
a_0
=\Braket{\frac{1}{\sqrt{2l}},f(x)}
=\int_{-l}^{l} \frac{1}{\sqrt{2l}}\cdot f(x)\mathrm{d}x
=\frac{1}{\sqrt{2l}}\int_{-l}^{l} f(x)\mathrm{d}x,
\end{equation}

\begin{equation}
a_k
=\Braket{\frac{1}{\sqrt{l}}\cos\frac{k\pi}{l} x,f(x)}
=\int_{x=-l}^{x=l}\frac{1}{\sqrt{l}}\cos\frac{k\pi}{l} x\cdot f(x)\mathrm{d}x
=\frac{1}{\sqrt{l}}\int_{x=-l}^{x=l} f(x)\cos\frac{k\pi}{l} x\mathrm{d}x,
\end{equation}

\begin{equation}
b_k
=\Braket{\frac{1}{\sqrt{l}}\sin\frac{k\pi}{l} x,f(x)}
=\int_{x=-l}^{x=l}\frac{1}{\sqrt{l}}\sin\frac{k\pi}{l} x\cdot f(x)\mathrm{d}x
=\frac{1}{\sqrt{l}}\int_{x=-l}^{x=l} f(x)\sin\frac{k\pi}{l} x\mathrm{d}x.
\end{equation}

\subsection{周期函数的 e 指数基傅里叶级数展开}

注意到:

\begin{equation}
\frac{1}{2\pi}\int_{-\pi}^{\pi} \mathrm{e}^{\mathrm{i}(m-n)x}\mathrm{d}x
=\delta_{m,n},
\end{equation}

可以证明,函数系 $\displaystyle{\left\{\frac{1}{\sqrt{2\pi}}\mathrm{e}^{\mathrm{i}m x},m\in \mathbb{Z} \right\} }$  可作为以 $2\pi$ 为周期的函数为元素的希尔伯特空间 $\mathcal{H}$ 中的一组正交完备归一基矢,以 $2\pi$ 为周期的函数 $f$ 在这组基矢上的展开式为:

\begin{equation}
f(x)
=\sum_{m=-\infty}^{\infty} C_m\cdot\frac{1}{\sqrt{2\pi}}\mathrm{e}^{\mathrm{i}mx} .
\end{equation}

利用正交归一条件

\begin{equation}
\begin{split}
\Braket{\frac{1 }{\sqrt{2\pi} }\mathrm{e}^{\mathrm{i}n x}, \frac{1 }{\sqrt{2\pi} } \mathrm{e}^{\mathrm{i}m x} }
&\equiv \int_{-\infty}^{+\infty} \left(\frac{1 }{\sqrt{2\pi} }\mathrm{e}^{\mathrm{i}n x} \right)^* \cdot \left(\frac{1 }{\sqrt{2\pi} } \mathrm{e}^{\mathrm{i}m x} \right)\mathrm{d}x \\
&=\frac{1 }{2\pi } \int_{-\infty}^{+\infty} \mathrm{e}^{\mathrm{i}(m-n)x}\mathrm{d}x \\
&=\delta_{m,n},
\end{split}
\end{equation}

内积可得系数:

\begin{equation}
\begin{split}
\Braket{\frac{1 }{\sqrt{2\pi} } \mathrm{e}^{\mathrm{i}n x}, f(x)}
&=\Braket{\frac{1 }{\sqrt{2\pi} } \mathrm{e}^{\mathrm{i}n x}, \sum_{m=-\infty}^{\infty} C_m\cdot\frac{1}{\sqrt{2\pi}}\mathrm{e}^{\mathrm{i}m x} } \\
&=\sum_{m=-\infty}^{\infty} C_m \Braket{\frac{1 }{\sqrt{2\pi} }\mathrm{e}^{\mathrm{i}n x}, \frac{1 }{\sqrt{2\pi} } \mathrm{e}^{\mathrm{i}m x} } \\
&=\sum_{m=-\infty}^{\infty} C_m\delta_{m,n} \\
&=C_n,
\end{split}
\end{equation}

即系数 $C_m $ 可通过内积求得:

\begin{equation}
\begin{split}
C_m
=\Braket{\frac{1}{\sqrt{2\pi}} \mathrm{e}^{\mathrm{i}mx},f(x)}
=\int_{-\pi}^{\pi}\left(\frac{1}{\sqrt{2\pi}} \mathrm{e} ^{\mathrm{i}mx}\right)^*\cdot f(x)\mathrm{d}x
=\frac{1}{\sqrt{2\pi}} \int_{-\pi}^{\pi} f(x) \mathrm{e}^{-\mathrm{i}mx}\mathrm{d}x.
\end{split}
\end{equation}

一般地,周期为 $2l $ 的周期函数 $f(x) $ 可在非归一的 $\mathrm{e} $ 指数基 $\left\{\mathrm{e}^{\mathrm{i}m \pi x / l},m\in \mathbb{Z} \right\} $ 上展开:

\begin{equation}
f(x)
=\sum_{m=-\infty}^{\infty} C_m \mathrm{e}^{\mathrm{i} m \pi x / l },
\end{equation}

为了求出展开系数,上式两边同乘 $\mathrm{e}^{-\mathrm{i} m' \pi x / l } $,并对 $x $ 从 $-l $ 到 $l $ 积分:

\begin{equation}
\begin{split}
\int_{-l}^{l} f(x) \mathrm{e}^{-\mathrm{i} m' \pi x / l } \mathrm{d}x
&=\int_{-l}^{l} \left(\sum_{m=-\infty}^{\infty} C_m \mathrm{e}^{\mathrm{i} m \pi x / l } \right) \mathrm{e}^{-\mathrm{i} m' \pi x / l } \mathrm{d}x \\
&=\sum_{m=-\infty}^{\infty} C_m \int_{-l}^{l} \mathrm{e}^{\mathrm{i}\left(m - m' \right)\pi x/l} \mathrm{d}x \\
&=\sum_{m=-\infty}^{\infty} C_m \cdot 2l \delta_{m,m'} \\
&=2l C_{m'}.
\end{split}
\end{equation}

于是得到系数 $C_{m} $ 的表达式:

\begin{equation}
C_{m}
=\frac{1 }{2l } \int_{-l}^{l} f(x) \mathrm{e}^{-\mathrm{i} m \pi x / l } \mathrm{d}x.
\end{equation}

\section{傅里叶变换}

对于一个非周期函数 $f(x) $,我们可以认为其为周期无穷大的周期函数。

先考虑一个周期为 $2l $ 的周期函数,其可在 $\mathrm{e} $ 指数基下展成如下的傅里叶级数:

\begin{equation}
\begin{split}
f(x)
&=\sum_{m=-\infty}^{\infty} C_m \mathrm{e}^{\mathrm{i}m\pi x/l} \\
&=\sum_{m=-\infty}^{\infty} \left(\frac{1 }{2l } \int_{x'=-l}^{x'=l} f(x') \mathrm{e}^{-\mathrm{i} m \pi x' / l } \mathrm{d}x' \right) \mathrm{e}^{\mathrm{i}m\pi x/l} \\
\end{split}
\end{equation}

令 $k=m\pi /l $,则 $m $ 遍历 $\mathbb{Z}\equiv \left\{0,\pm 1,\pm 2,\cdots \right\} $ 等价于 $k $ 遍历 $\frac{\pi }{l } \mathbb{Z}\equiv \left\{0,\pm \frac{\pi }{l }, \pm \frac{2\pi }{l },\cdots   \right\}  $,且 $\Delta m =1 $ 对应 $\Delta k= \frac{\pi }{l }  $ 于是上式对 $m $ 的求和可改写为:

\begin{equation}
\begin{split}
f(x)
&=\sum_{m=-\infty}^{\infty} \left(\frac{1 }{2l } \int_{x'=-l}^{x'=l} f(x') \mathrm{e}^{-\mathrm{i} m \pi x' / l } \mathrm{d}x' \right) \mathrm{e}^{\mathrm{i}m\pi x/l} \\
&=\sum_{k\in \frac{\pi }{l } \mathbb{Z} } \left(\frac{1 }{2l } \int_{x'=-l}^{x'=l} f(x') \mathrm{e}^{-\mathrm{i} k x' } \mathrm{d}x' \right) \mathrm{e}^{\mathrm{i} k x} \\
&=\sum_{k\in \frac{\pi }{l } \mathbb{Z} } \left(\frac{1 }{2l\cdot \Delta k } \int_{x'=-l}^{x'=l} f(x') \mathrm{e}^{-\mathrm{i} k x' } \mathrm{d}x' \right) \mathrm{e}^{\mathrm{i} k x} \Delta k \\
&=\frac{1 }{2\pi }  \sum_{k\in \frac{\pi }{l } \mathbb{Z} } \left( \int_{x'=-l}^{x'=l} f(x') \mathrm{e}^{-\mathrm{i} k x' } \mathrm{d}x' \right) \mathrm{e}^{\mathrm{i} k x} \Delta k \\
\end{split}
\end{equation}

现在令 $l\to +\infty $,相当于说 $f(x) $ 不是周期函数。此时 $\Delta k \to 0 $,

\begin{equation}
\lim_{l\to +\infty} \left( \int_{x'=-l}^{x'=l} f(x') \mathrm{e}^{-\mathrm{i} k x' } \mathrm{d}x' \right)
=\int_{-\infty}^{+\infty} f(x') \mathrm{e}^{-\mathrm{i} k x' } \mathrm{d}x'
\end{equation}

定义函数 $f(x) $ 的傅里叶变换 $\tilde{f}(k) $ 为:

\begin{equation}
\tilde{f}(k)
\equiv \int_{-\infty}^{+\infty} f(x) \mathrm{e}^{-\mathrm{i} k x } \mathrm{d}x
\end{equation}

进而有

\begin{equation}
\begin{split}
f(x)
&=\lim_{l\to +\infty} \left[\frac{1 }{2\pi }  \sum_{k\in \frac{\pi }{l } \mathbb{Z} } \left( \int_{x'=-l}^{x'=l} f(x') \mathrm{e}^{-\mathrm{i} k x' } \mathrm{d}x' \right) \mathrm{e}^{\mathrm{i} k x} \Delta k \right] \\
&=\lim_{l\to +\infty} \left[\frac{1 }{2\pi }  \sum_{k\in \frac{\pi }{l } \mathbb{Z} } \tilde{f}(k) \mathrm{e}^{\mathrm{i} k x} \Delta k \right] \\
&=\frac{1 }{2\pi } \int_{k=-\infty}^{k=+\infty} \tilde{f}(k) \mathrm{e}^{\mathrm{i}kx}\mathrm{d}k
\end{split}
\end{equation}

上面这种把 $f(x) $ 展开成不同频率的平面波 $\mathrm{e}^{\mathrm{i} kx} $ 的方法称为傅里叶积分展开。波矢为 $k $ 的平面波前面的展开系数 $\tilde{f}(k) $ 就称为 $f(x) $ 的傅里叶变换。


