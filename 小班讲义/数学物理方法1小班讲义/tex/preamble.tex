% ===== 字体和版面 =====
% ctexbook 已经设置好了字体

\usepackage[a4paper, margin=2.5cm, headheight=15pt]{geometry}  % "a4paper"--A4纸大小;"margin=2.5cm"--上下左右统一边距 2.5 厘米;"headheight=15pt" 页眉高度
\linespread{1.3} % 1.3倍行距

% ===== 数学宏包 =====
\usepackage{amsmath, amssymb, amsthm, mathtools, bm, physics}

% ===== 图片和浮动体 =====
\usepackage{graphicx, subcaption, float}

% ===== 超链接 =====
\usepackage{hyperref}
\hypersetup{
    colorlinks=true,
    linkcolor=black, % 标题超链接黑色
    citecolor=blue, % 引用超链接蓝色
    urlcolor=magenta % 网址超链接洋红色
}

% ===== 目录 =====
\setcounter{tocdepth}{3} % 控制目录显示到 subsubsection 层级
\setcounter{secnumdepth}{3} % 编号显示到到 subsubsection 层级

% ===== 章节标题 =====
\usepackage{titlesec}
\titleformat{\chapter}[hang]{\bfseries\Huge}{第\,\thechapter\,章}{1em}{}
\titleformat{\section}[hang]{\bfseries\Large}{\thesection}{1em}{}

% ===== 页眉页脚 =====
\usepackage{fancyhdr}
\pagestyle{fancy}
\fancyhf{}
\fancyhead[LE,RO]{\thepage} % 左页左边、右页右边显示页码
\fancyhead[RE]{\leftmark} % 偶数页页眉显示章名
\fancyhead[LO]{\rightmark} % 奇数页页眉显示节名

% ===== 定理和例题环境 =====
\newtheorem{theorem}{定理}[chapter]
\newtheorem{definition}{定义}[chapter]
\newtheorem{corollary}[theorem]{推论}
\newtheorem{example}{例}[chapter]
\newtheorem{exercise}{习题}[chapter]

% ===== 彩色框 =====
\usepackage[most]{tcolorbox}
\tcolorboxenvironment{example}{
  colback=blue!5!white,
  colframe=blue!40!black,
  boxrule=0.5pt,arc=3pt,left=6pt,right=6pt,top=3pt,bottom=3pt
}
\tcolorboxenvironment{exercise}{
  colback=yellow!5!white,
  colframe=orange!60!black,
  boxrule=0.5pt,arc=3pt,left=6pt,right=6pt,top=3pt,bottom=3pt
}

% ===== 边注 =====
\usepackage{marginnote}

