\documentclass[10pt]{article}
\usepackage[margin=1in]{geometry}
\usepackage{dsfont}
\usepackage{amsmath, amsthm, amssymb, mathrsfs}
\usepackage[abbrev,non-sorted-cites]{amsrefs}
\usepackage[all]{xy}
\usepackage{CJK}
\usepackage{grffile}
\usepackage{graphicx}
\ProvidesPackage{stackrel}
\newtheorem{thm}{Theorem}[section]
\newtheorem{cor}[thm]{Corollary}
\newtheorem{lem}[thm]{Lemma}
\newtheorem{ex}[thm]{Example}
\newtheorem{definition}[thm]{Definition}
\newtheorem{rmk}[thm]{Remark}
\newtheorem{obs}[thm]{Observation}
\newtheorem{prop}[thm]{Proposition}
\newtheorem{conj}[thm]{Conjecture}
\newtheorem{question}[thm]{Question}
\newtheorem{prob}[thm]{Problem}
\renewcommand{\thefootnote}{\arabic{footnote}}

\newcommand{\de}{\partial}
\newcommand{\db}{\overline{\partial}}
\newcommand{\ov}{\overline}
\newcommand{\ddbar}{\sqrt{-1}\de\db}
\newcommand{\vp}{\varphi}
\newcommand{\ti}{\tilde}
\newcommand{\ve}{\varepsilon}
\newcommand{\Ric}{\mathrm{Ric}}
\newcommand{\tr}[2]{\textrm{tr}_{#1} #2}
\renewcommand{\leq}{\leqslant}
\renewcommand{\geq}{\geqslant}



\begin{document}

%\title{Geometry and Topology Problems}

%\maketitle

%\section*{Individual}

\begin{center}
S.-T. Yau College Student Mathematics Contests 2019

\vspace{0.1cm}

\Large {\bf Geometry and Topology}

\vspace{0.1cm}

\large {\bf Individual (5 problems)}

\vspace{0.1cm}
\end{center}

\begin{enumerate}
\item [1)]  Let $\text{Conf}_n$ be the following submanifold of $\mathbb C^n$:
$$
\text{Conf}_n=\{(z_1,z_2, \cdots, z_n)\in \mathbb C^n| z_i\neq z_j \ \text{for any}\  i\neq j\}.
$$
For every pair $(i,j)$ with $i\neq j$, we define the complex valued 1-form
$$
\omega_{ij}:= {dz_i-dz_j\over z_i-z_j}.
$$
\begin{enumerate}
\item [(a)] Show that for any $i\neq j$, $\omega_{ij}$ represents a non-zero de Rham cohomology class in $H^1(\text{Conf}_n, \mathbb C)$.

\item [(b)] Show that for any pair-wise distinct indices $i,j,k$,
$$
  \omega_{ij}\wedge \omega_{jk}+\omega_{jk}\wedge \omega_{ki}+\omega_{ki}\wedge \omega_{ij}=0.
$$
\end{enumerate}





\item [2)] Let $M$ be a compact oriented manifold of (real) dimension $4$. Consider the following symmetric bilinear form on  $H^2(M)$
$$H^2(M)\times H^2(M)\to\mathbb{R},\quad ([\alpha],[\beta])\mapsto\int_M\alpha\wedge\beta.
$$
Let $\tau(M)$ be the signature of this bilinear form, i.e. the number of positive eigenvalues minus the number of negative eigenvalues. Compute $\tau(M)$ for $M=S^4, \mathbb{CP}^2$ and $S^2\times S^2$.

\item [3)] Let $X=\mathbb{R}^4/\sim$, where
      \begin{align*}
         (x_1,x_2,x_3,x_4)&\sim (x_1,x_2+1,x_3,x_4)\\
         (x_1,x_2,x_3,x_4)&\sim (x_1,x_2,x_3,x_4+1)\\
         (x_1,x_2,x_3,x_4)&\sim (x_1+1,x_2,x_3,x_4)\\
         (x_1,x_2,x_3,x_4)&\sim (x_1,x_2+x_4,x_3+1,x_4)\\
      \end{align*} Compute $H_1(X, \mathbb Z)$.


\item [4)]  Let $E$ be a vector bundle over a smooth manifold
$M$. Let $\nabla^E$ be a connection $E$ and $R^E\in \Omega^2(M, \text{End}(E))$ be its curvature tensor. For any polynomial $f(x)=a_0+a_1x+a_2 x^2+\cdots+a_nx^n$, we denote
$$
f(R^E)=a_0+a_1 R^E+ a_2 (R^E)^2\cdots+a_n (R^E)^n \in \Omega^*(M, \text{End}(E)).
$$
Here $(R^E)^k\in \Omega^{2k}(M, \text{End}(E))$ is the $k$-th wedge product on forms combined with matrix multiplications on $\text{End}(E)$.
\begin{enumerate}
\item [(a)] Show that the differential form $\text{tr}\left[ f(R^E)\right]\in \Omega^{*}(M)$ is closed
$$
  d \text{tr}\left[ f(R^E)\right]=0.
$$
Here $\text{tr}$ is the trace on $\text{End}(E)$.
\item [(b)] Let $\nabla^E$, $\widetilde\nabla^E$ be two connections on $E$ and $R^E$, $\widetilde R^E$ be the corresponding
curvature tensors.  Show that  there exists a differential form
$\omega\in\Omega^*(M)$ such that
$$\text{tr}\left[ f(R^E)\right]-\text{tr} \left[f(\tilde R^E)\right]=d\omega.$$
\end{enumerate}
\item [5)]
\begin{enumerate}
\item [(a)] Let $u$ be a smooth function over a Riemannian manifold $(M,g)$.  Prove the following Bochner's formula
$$\frac{1}{2}\Delta|\nabla u|^2=|\nabla\nabla u|^2+\text{Ric}(\nabla u,\nabla u)+g(\nabla\Delta u,\nabla u)$$
where $\Delta$ is the Laplacian and $|\bullet|^2=g(\bullet,
\bullet)$.

\item [(b)] Let $(S^2,g)$ be the standard unit sphere and $E$ be a constant. Show that the only smooth positive solution to
$$\Delta \ln f+Ef^2=1$$ is $f=\frac{1}{A+\phi}$ where $A$ is a
constant and $\phi$ is some first eigenfunction of $S^2$.
\end{enumerate}


\end{enumerate}






\end{document}
